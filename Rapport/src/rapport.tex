
% Le type de votre document
\documentclass[a4paper,10pt]{article}

% Quelques packages pour le francais, vous pouvez saisir du texte accentué.
\usepackage[utf8]{inputenc}
\usepackage[frenchb]{babel}
\usepackage{verbatim}
% Des trucs biens pour le pdf.
\usepackage{ae}
\usepackage{aeguill}
\usepackage[bookmarks=true,colorlinks,linkcolor=blue]{hyperref}
\usepackage[french, figure, boxed, onelanguage,]{algorithm2e}
\usepackage{amsmath,amssymb,mathrsfs}
\usepackage{makeidx}
\usepackage{slashbox}
\usepackage{multirow}
\usepackage{array}
\usepackage{float}
\usepackage[table]{xcolor}


% Pour inclure des graphiques.
\usepackage{graphicx}


\usepackage{tikz}
\usepackage{pgfplots}


% Pour faire déborder les cases d'un tableau sur plusieurs lignes.
\usepackage{multirow}

\usepackage[left=2.5cm,right=2.5cm,top=2.5cm,bottom=2.5cm]{geometry}

\usepackage[final]{pdfpages}

\hypersetup{
    colorlinks=true,                         
    linkcolor=blue, % Couleur des liens internes
    citecolor=red, % Couleur des numéros de la biblio dans le corps
    urlcolor=blue  } % Couleur des url


\newtheorem{definition}{Définition}
\newtheorem{proposition}{Proposition}
\newtheorem{exemple}{Exemple}
\newtheorem{remarque}{Remarque}
\newtheorem{preuve}{Preuve}

\SetKwFor{PourTous}{pour tout}{faire}{fin~pour}

\begin{document}
%
\includepdf[pages=1]{couverture/couverture.pdf}
\newpage
\thispagestyle{empty}
\null
\newpage
\includepdf[pages=1]{couverture/nonPlagiat.pdf}
\newpage
\thispagestyle{empty}
\null

\newpage
\thispagestyle{empty}
\section*{\begin{center}Remerciements\end{center}}
\par Tout d'abord, je tiens à remercier Frédéric LARDEUX et Frédéric SAUBION pour leur écoute, leurs disponibilités, leurs explications et pour la liberté qu'ils m'ont laissée lors de la réalisation de ce projet.
\par Je remercie également les membres du LERIA\footnote{Laboratoire d'études et de recherche en informatique d'Angers}, pour m'avoir accordé leur confiance en m'attribuant ce stage.
\par Je remercie l'équipe enseignante du DAEU\footnote{Diplôme d'accès aux études universitaire} d'Angers, et particulièrement Eric SCHRAFSTETTER qui m'a mis le pied à l'étrier pour intégrer la faculté des sciences d'Angers.
\par Enfin, je remercie ma famille, qui m'a toujours soutenue dans la reprise de mes études, et tout particulièrement Lucie, Marius et Paulin qui partagent mes humeurs et mon quotidien.
\newpage
\thispagestyle{empty}
\null
\newpage

\newpage
\thispagestyle{empty}
\null
\begin{flushright}
\vspace{\fill}
\textbf{À Marius et Paulin.}
\vspace{\fill}
\end{flushright}

\newpage
\thispagestyle{empty}
\null
\newpage

\newpage
\section{Introduction}
\subsection*{Sujet du stage}
\par La plupart des bactéries appartenant au genre Xanthomonas sont responsables de pathologies sur
une large gamme de cultures économiquement importantes, induisant notamment des pertes
de rendement et diminuant ainsi la valeur marchande des semences. Quelques graines
contaminées suffisent à générer une source d'inoculation primaire et à occasionner ainsi une
dissémination ultérieure plus large. En particulier, le pathovar phaseoli de Xanthomonas axonopodis
(Xap) qui regroupe toutes les souches identifiées comme pathogènes sur le haricot n'est pas
endémique en Europe mais pour limiter son introduction, il est inscrit sur la liste des agents
pathogènes de quarantaine. Une approche possible pour l'identification des souches bactériennes
consiste à utiliser un répertoire de gènes de virulence. Il s'agit ainsi de trouver la plus petite
combinaison de gènes de virulence spécifiques. Cette combinaison peut ainsi être utilisée pour
concevoir un test d'identification. Des travaux préliminaires\cite{Chhel2013} montrent que la combinaison des
tests moléculaires ainsi obtenus fournit une technique rapide pour l'identification de toutes les
souches de Xanthomonas pathogènes sur les haricots.
\par Avec les possibilités accrues d'acquisition de données génomiques – par exemple le séquençage à
haut débit – mais également phénotypiques, le problème de la caractérisation de données
biologiques devrait rapidement devenir l'un des verrous essentiel de l'exploitation effective des
grandes bases de données qui sont en cours de constitution, et constituera donc un centre d'intérêt
commun aux biologistes des domaines du végétal ou de la santé. La caractérisation telle que nous
l'entendons permet d'identifier les caractères propres, éventuellement hétérogènes, d'un groupe
d’individus partageant des spécificités fonctionnelles communes (par exemple pathologiques).
\par Du point de vue informatique, ce problème est abordé comme la recherche d'un ensemble de
formules propositionnelles (variables booléennes) permettant de caractériser de manière exacte les
groupes de pathogènes. Les algorithmes mis en jeu reposent sur des explorations arborescentes
(Branch \& Bound) et des algorithmes heuristiques (recherche locale).
\par L'objectif de ce stage est double :
\begin{itemize}
\item D'une part il s'agit de constituer de nouveaux jeux de données, en dialogue avec nos collègues
biologistes. Ceci requiert la définition de formats et l'exploration de base de données pour la collecte
d'informations pertinentes. Ce travail sera effectué en lien étroit avec les laboratoires de l'INRA
Angers.
\item D'autre part, il s'agit également d'améliorer les algorithmes existants et de proposer de nouvelles
approches pour traiter des instances de grande taille. Cette phase s'inspire des algorithmes de
résolution de problèmes combinatoires (SAT-CSP).
\end{itemize}

\subsection*{Plan du rapport}
Dans la première partie de ce mémoire, nous dressons un état de l'art concernant l'étude du MIN-PCM. Celui-ci reprend les grands axes présentés dans l'article \cite{Chhel2013}. Ensuite nous décrivons les contributions que nous apportons à l'étude de ce problème: nous définissons la notion d'instance difficile, nous proposons des heuristiques de résolution (exactes ou approchées) et nous fournissons les résultats ainsi obtenus. Enfin, nous tirons les conclusions de nos travaux et nous discutons les perspectives de recherche envisagées.

\newpage
\thispagestyle{empty}
\null
\newpage

% La table des matières
\newpage
\setcounter{tocdepth}{3}
\tableofcontents
\newpage

\newpage
\thispagestyle{empty}
\null
\newpage

\section{État de l'art}

%\subsection{Notations}
%Dans ce mémoire, nous utilisons les notations suivantes:
%\begin{itemize}
%\item $\mathcal{X}$ représente l'ensemble des gènes d'une instance non redondante.
%\item $\mathcal{G}$ représente l'ensemble des gènes d'une instance non redondante.
%\item $\mathcal{E}$ représente l'ensemble des gènes d'une instance non redondante.
%\item $|\mathcal{X}|$ représente la cardinalité de l'ensemble $\mathcal{X}$.
%\item $|\mathcal{G}|$ représente la cardinalité l'ensemble $\mathcal{G}$.
%\item $|\mathcal{E}|$ représente la cardinalité l'ensemble $\mathcal{E}$.
%\end{itemize}



Nos travaux faisant suite à ceux présenté dans l'article\cite{Chhel2013} \emph{Application du problème de caractérisation multiple à la conception de test de diagnostic pour la biologie végétale} de F.Chhel, F.Lardeux, F.Saubion et B.Zanuttini parut en 2013 dans la \emph{Revue d'intelligence artificielle}  et dans un souci de cohérence, nous reprenons dans cette section, avec l'accord des auteurs, les éléments clefs de l'article afin de présenter un état de l'art du problème de caractérisation multiple.
 
\subsection{Présentation générale du problème}
La plupart des bactéries appartenant au genre Xanthomonas sont responsables de pathologies sur une large gamme de cultures économiquement importantes,  induisant notamment  des pertes de rendement et diminuant ainsi la valeur marchande des semences. Quelques graines contaminées sont suffisantes pour générer une source d'inoculation primaire et occasionner ainsi une dissémination ultérieure plus large. En particulier, le pathovar\footnote{La notion de pathovar correspond à une subdivision du genre ayant des caractéristiques pathologiques observées communes.} phaseoli de  Xanthomonas Axonopodis (Xap) qui regroupe toutes les souches identifiées comme pathogènes sur le haricot \cite{Vauterin1995}, n'est pas endémique en Europe mais pour limiter son introduction, il est inscrit sur la liste des agents pathogènes de quarantaine.

La taxonomie du genre Xanthomonas n'est pas encore pleinement résolue, et la délimitation de certaines espèces dans ce genre fait encore débat \cite{Schaad2005}~; les approches phylogénétiques ne sont alors pas réellement applicables. Une approche possible pour l'identification des souches bactériennes consiste à utiliser un répertoire de gènes de virulence. Il s'agit ainsi de trouver la plus petite combinaison de gènes de virulence spécifiques au Xap. Cette combinaison peut être utilisée pour concevoir un test PCR multiplex pour l'identification de Xap \cite{Boureau2013,Boureau2012} dont le coût est directement lié au nombre de gènes à tester. Les résultats obtenus
montrent que la combinaison des tests moléculaires ainsi obtenus fournit une technique rapide pour l'identification de toutes les souches de Xanthomonas pathogènes sur les haricots.

Plus formellement, considérons un ensemble d'entités (les souches bactériennes) regroupées en groupes (les pathovars). Chaque entité est définie par la présence ou l'absence d'un ensemble de caractères (les gènes). Au regard de la représentation binaire qui est utilisée, une entité est considérée comme une interprétation booléenne sur les caractères, qui seront donc les variables booléennes du problème. Ainsi, pour chaque groupe, l'ensemble des entités fournit une table de vérité partielle d'une fonction booléenne vraie pour les interprétations correspondant aux entités du groupe et fausse pour toutes les autres entités des autres groupes. Une telle fonction sera appelée caractérisation d'un groupe. Ces fonctions booléennes seront représentées par des formules propositionnelles construites sur un langage fixé.



Le problème de caractérisation multiple consiste ainsi à trouver un ensemble de
formules booléennes de sorte que chaque formule soit une caractérisation. Le terme
multiple désigne le fait qu'il faut considéroer un ensemble de groupes dont les
caractérisations sont dépendantes des entités contenues dans ceux-ci mais aussi
des entités appartenant aux autres groupes.

\begin{figure}[H]
\begin{center}
\begin{tabular}{|c||c|c|c|c|}
\hline
\multirow{2}{*}{Souches}&\multirow{2}{*}{Groupes}&\multicolumn{3}{c|}{Caractères
}\\
&&$a$&$b$&$c$\\
\hline
\hline
$e_1$&\multirow{2}{*}{$g_1$}&\cellcolor{lightgray}0&\cellcolor{lightgray}0&0\\
\cline{1-1} \cline{3-5}
$e_2$&&\cellcolor{lightgray}0& \cellcolor{lightgray}0&1\\
\hline
\hline
$e_3$&$g_2$&1&\cellcolor{lightgray}1&\cellcolor{lightgray}1\\
\hline
\hline
$e_4$&\multirow{2}{*}{$g_3$}&1&\cellcolor{lightgray}1&\cellcolor{lightgray}0\\
\cline{1-1} \cline{3-5}
$e_5$&&0&\cellcolor{lightgray}1&\cellcolor{lightgray}0\\
\hline
\end{tabular}
\end{center}
\caption{Exemple de problème de caractérisation multiple}
\label{CD}
\end{figure}

Dans la figure \ref{CD} il y a 5 entités réparties dans 3 groupes et dont
la description se base sur un ensemble de 3 caractères.
Résoudre ce problème revient à caractériser chaque groupe. Il faut donc, pour
chaque groupe, trouver une combinaison de  variables permettant de construire une formule vraie pour toutes les
entités du groupe et fausse pour les autres entités des autres
groupes. Dans l'exemple de la figure \ref{CD}, le groupe 1 est caractérisé par
la négation des variables $a$ et $b$  alors que le groupe 2 est caractérisé par
les variables $b$ et $c$. Les souches du groupe 3 ont toutes en commun la
négation de la variable $c$ tout comme l'entité $e_1$ du groupe 1. Il faut donc
ajouter une autre variable ($b$ par exemple) pour être sûr de caractériser le
groupe.
\subsection{Problème de caractérisation multiple}
Nous présentons dans cette section le problème de caractérisation multiple ainsi que les diverses méthodes qui ont été proposés dans \cite{Chhel2012,Chhel2013} et qui permettent de résoudre le MIN-PCM.

\subsubsection{Présentation du problème}

\begin{definition}[Instance du PCM]
Une instance du problème de caractérisation multiple est définie par un n-uplet
$({\cal X}, {\cal E}, {\cal G})$ où $\cal X$ est l'ensemble des
variables propositionnelles, ${\cal E}$ est l'ensemble des entités définies sur
$\cal X$ et ${\cal G} \subseteq 2^{\cal E}$.
\end{definition}

Chaque entité représente une affectation booléenne, ou interprétation, définit ainsi  $e : {\cal X} \to \{0,1\}$, où  $0$ et $1$ sont respectivement les valeurs de vérité fausse et vraie. $e(x)$ correspond donc à la valeur de vérité affectée à $x$ dans l'interprétation $e$. Pour une formule propositionnelle  $\phi$ quelconque sur ${\cal X}$, nous notons $e \models \phi $ le fait que l'interprétation $e \in {\cal E}$ satisfait la formule $\phi$. Une entité $e \in {\cal E}$ sera classiquement représentée par un n-uplet de valeur booléennes (un élément de  $ \{0,1\}^n$).   Ainsi, une instance $({\cal X}, {\cal E}, {\cal G})$  peut être vue comme une matrice booléenne dont les $n$ colonnes correspondent aux variables booléennes de ${\cal X}$ et les $m$ lignes aux entités de ${\cal E}$.  Chaque variable $x_j$ correspond à la colonne $j \in \{1, \ldots, n \} $. Chaque entité $e_i$ correspond à une ligne $i \in \{1, \ldots, m \}$.

Nous pouvons appliquer des prétraitements pour réduire la taille de la matrice en réduisant le nombre de variables réellement utiles et/ou le nombre d'entités. Nous définissons ainsi la notion d'instance non redondante.

\begin{definition}[Instance non redondante]
Une instance est non redondante ssi :
\begin{itemize}
\item Il n'existe pas de colonnes dont toutes valeurs sont identiques :

$\nexists j \in \{1,\ldots, n \},\forall i \in \{1, \ldots, m \} $ tel que $a_{ij}=1$ (resp $a_{ij}=0$);
\item Chaque colonne est unique :

$\nexists j \in \{1, \ldots, n \},\forall k \in \{1, \ldots, n \} \setminus \{j\}, \forall i \in \{1, \ldots, m \} $ tel que $a_{ij}=a_{ik}$;
\item Chaque entité est unique : \\$ \nexists i \in \{1, \ldots, m \},
\forall l \in \{1, \ldots, m \} \setminus \{i\},\forall j \in \{1, \ldots, n \},$ tel que $e_i(x_j)=e_l(x_j)$.
\end{itemize}
\end{definition}

L'application d'un algorithme d'élimination de la redondance s'effectue au pire des cas en ${\mathcal O}({|\cal X}|^2+|{\cal E}|^2)$ et peut conduire à une instance mal-formée dont toutes les colonnes ont été supprimées ou possédant des groupes vides.

Une fois l'instance réduite, nous pouvons préciser la notion de solution d'un problème de caractérisation multiple.

\begin{definition}[Caractérisation d'un groupe]
Pour une instance $({\cal X}, {\cal E}, {\cal G})$, une formule $\phi_g$
caractérise un groupe $g\in {\cal G}$ ssi :
$\forall e \in g, e \models \phi_g$ (acceptation des entités du groupe) et
$\forall g' \in {\cal G} \setminus \{g\},\forall e' \in g',  e' \not \models
\phi_g$ (rejet des entités des autres groupes).
\end{definition}

Par extension, nous notons $g \models \phi_g$ le fait que $\phi_g$ caractérise $g$ selon la définition précédente. $Sol(g)$ représente l'ensemble des solutions d'un groupe $g$. $Sol(g)=\{\phi_g|g \models \phi_g\}$.

\begin{definition}[Solution d'un PCM ]
Pour une instance $({\cal X}, {\cal E}, {\cal G})$, une solution
admissible du PCM  est un $|{\cal G}|$-uplet de formules
$\Phi=(\phi_1,\cdots,\phi_{{\cal G}})$ tel que  $\forall i\in \{1,\ldots,|G|\}, g_i \in
{\cal G}$, $g_i \models \phi_i$.
\end{definition}

Soit ${\cal I} = ({\cal X}, {\cal E}, {\cal G})$, $ SOL({\cal I})$ est
l'ensemble de toutes les solutions multiples pour tous les groupes. $SOL({\cal
I})=Sol(g_1) \times \cdots \times Sol(g_{|{\cal G}|})$, où $\times$ désigne le produit cartésien. Pour un ensemble de
groupes ${\cal G}$ et un $|G|$-uplet de formules $\Phi=(\phi_1,\cdots,\phi_{|{\cal G}|})$, nous notons par extension ${\cal G} \models \Phi$ le fait que
$\forall i\in \{1,\ldots,|G|\}, g_i \in {\cal G}$, $g_i \models \phi_i$.

\begin{definition}[Satisfiabilité d'un PCM]
Une instance $({\cal X}, {\cal E}, {\cal G})$ est satisfiable (resp. insatisfiable) ssi
$\forall g \in {\cal G}, Sol(g) \neq \emptyset$ (resp. $\exists g \in {\cal G},
Sol(g) = \emptyset$).
\end{definition}

Il est évident que toute instance possédant deux interprétations identiques dans deux groupes différents n'est pas satisfiable d'où :

\begin{proposition}
Une instance $({\cal X},{\cal E},{\cal G})$ est satisfiable $\Leftrightarrow
\forall g,g' \in {\cal G}, g\neq g', g
\cap g' =\emptyset$.
\end{proposition}

Une conséquence immédiate est qu'une instance non redondante et bien formée
est satisfiable puisque chaque entité a une interprétation unique.
En pratique, le test de satisfiabilité est effectué implicitement lors de
l'étape de prétraitement et à charge de l'utilisateur de modifier ou supprimer
les entités mis en cause dans l'échec de ce test.

\begin{definition}[Taille d'un n-uplet de formule]
\label{Flength}
Pour un n-uplet $\Phi=(\phi_1,\cdots,\phi_n)$ nous avons $|\Phi| =
|\bigcup_{\phi_i} var(\phi_{i})|$, où $var({\phi})$ retourne l'ensemble des
variables de
$\phi$.
\end{definition}

\begin{definition}[k-PCM (problème de décision)]
Soit une instance ${\cal I}$ une
caractérisation multiple minimale $k$ est un ensemble de formules
$\Phi \in Sol({\cal I})$ vérifiant $|\Phi| \leq k$ avec $ k\in \mathbb{N}^{+}$.
\end{definition}

Le problème de caractérisation multiple minimale pour une taille $k$ ne
correspond pas nécessairement à une solution minimale de
$(\phi_1,\cdots,\phi_n)$ telle que chaque $\phi_i$ est un élément minimal de
$Sol(g_i)$. Nous définissons le problème d'optimisation comme suit.

\begin{definition}[MIN-PCM (problème d'optimisation)]
Pour une instance ${\cal I}$, une
caractérisation optimale multiple minimale est  un ensemble de formules
$\Phi^* \in Sol({\cal I})$ vérifiant $|\Phi^*| \leq |\Phi|$ avec $\forall \Phi
\in Sol({\cal I})$
\end{definition}

\subsubsection*{Minimisation du problème de caractérisation multiple}
La minimisation du problème de caractérisation multiple consiste à définir le plus petit nombre de gène pouvant caractériser une instance PCM.


\subsection{Complexité}
Le problème SET-COVER appartient à la classe de complexité W[2]-complet. Il a été montré dans \cite{Chhel2013} qu'une instance PCM pouvait être réduite en temps polynômial en une instance SET-COVER. Il en résulte que PCM appartient à la classe de complexité W[2]-complet \footnote{En admettant l'hypothèse que la WEFT-hiérarchie proposé par \cite{DowneyF95} soit correcte.}. Dès lors, il a été prouvé que le MIN-PCM appartient à la classe de complexité W[2]-difficile. L'impact direct de l'appartenance de MIN-PCM à cette classe de complexité est que l'\textbf{unique} possibilité d'améliorier significativement la résolution complète\footnote{Recherche exacte permettant de prouver l'optimalité d'une solution.} d'une instance est de \textbf{travailler sur des heuristiques de choix de variables}(gènes). 

\subsection{Méthodes de résolutions}
Toutes les méthodes présentées dans cette sous-section sont issus de \cite{Chhel2013}.

\subsubsection{Une résolution basée sur les fonctions booléenes partiellement définis}

Les fonctions booléennes partiellement définies, {\em partially
defined Boolean formula - pdBf},  \cite{Iba99} permettent de proposer un
cadre de résolution intéressant.

Une pdBf est vue comme une fonction booléenne pour laquelle  certaines interprétations ne
sont pas définies.
La classe ${C}^{+}$ (resp. ${C}^{-}$)  désigne l'ensemble des exemples positifs
(resp. négatifs).
À partir de toute fonction booléenne, nous pouvons calculer une formule
caractérisant l'ensemble des interprétations, appelée extension. Il en est de
même pour les pdBf où une DNF (formule en forme normale disjonctive) est une
extension facilement calculable caractérisant la classe $\mathcal{C}^+$.

En ce qui concerne le PCM, nous construisons un ensemble de pdBf emboîtées où
chaque classe ${C}^{-}$ est l'union des classes ${C}^{+}$ des autres groupes.
Nous nous appuyons sur la notion de projection pour calculer de nouvelles
solutions.

\begin{definition}[Projection]
Une projection $\pi$ d'une instance ${\cal I}$ de PCM est la donnée d'un
sous-ensemble ${\cal X'} \subseteq {\cal X}$, définissant implicitement
l'instance ${\cal I'} = ({\cal X'},\{\pi(e) \mid e \in  {\cal E} \},\{ \pi(g)
\mid g \in {\cal G}\})$, où $\pi(e)$ est la restriction de $e$ aux variables de
${\cal X'}$, et $\pi(g)$ est $\{\pi(e)\mid e\in g\}$. On appelle
\emph{dimension} d'une projection $\pi$  le cardinal de ${\cal X'}$.
\end{definition}

La minimisation du nombre de colonnes pour le problème PCM revient à chercher la
projection satisfiable de plus petite dimension associée aux pdBf de chaque
groupe.

La principale difficulté du PCM ne dépend pas de la structure des formules de
l'ensemble solution mais du choix des variables présentes dans celui-ci. La
satisfiabilité d'un ensemble de formules de taille $k$  pour une instance du PCM est équivalente au fait que nous cherchons un sous-ensemble de pdBf consistantes, issues d'une projection satisfiable de dimension $k$ sur les pdBf initiales.

%\subsection{Application : test de diagnostic en biologie végétale}
%
%Nous présentons dans cette section l'utilisation du PCM dans le contexte de la biologie végétale.
%
%\subsubsection{La caractérisation de pathovars}
%
%Comme mentionné en introduction, la bactérie {\em Xanthomonas}  cause de gros
%dégâts sur certaines cultures. Les différentes souches de cette bactérie sont
%réparties en groupes et leurs caractéristiques (génotypiques ou phénotypiques)
%sont connues. L'objectif est de créer un test de diagnostic permettant de
%détecter rapidement à quel groupe appartient une souche donnée.
%
%La caractérisation précise des collections de souches bactériennes est un enjeu
%scientifique majeur, car ces bactéries sont responsables d'importantes
%pathologies végétales, et donc soumises à des procédures de contrôles officiels
%(par exemple, en Europe, la directive 2000/29/CE). Le développement de tests de
%diagnostic est donc crucial pour identifier systématiquement les
%souches de ces espèces. Dans ce contexte, le problème de caractérisation
%correspond à l'identification d'un groupe de souches vis-à-vis d'autres groupes,
%basée sur la présence ou l'absence de certains caractères particuliers
%\cite{plos}. Une souche peut donc être vue comme un vecteur de valeurs binaires
%qui reflète la présence (valeur 1) ou l'absence (valeur 0) de ces caractères. De
%plus, les tests de diagnostic étant basés sur des puces à ADN, il est nécessaire
%de chercher à minimiser les solutions (le nombre de tests associés pour chaque
%identification) afin de réduire le coût prohibitif de ces tests. Notons également que pour chaque caractère le spot de test doit être dupliqué sur la puce, ce qui réduit d'autant plus le nombre de caractères utilisables pour un même kit de diagnostic. Ce problème correspond exactement au MIN-PCM.

\subsubsection{Résolutions complètes}
\paragraph{\emph{Exact-Proj-Car}}
Le solveur \emph{Exact-Proj-Car}, proposé dans \cite{sac12}, est basé sur une approche complète consistant à valider ou non la présence d'une caractérisation de taille $k$. Deux types d'exploration sont possibles :
\begin{itemize}
 \item Une exploration en largeur consistant à montrer qu'il n'existe aucune caractérisation valide de taille $k$ avant de tester celles de taille $k+1$. En commençant avec $k= \lceil
log_2(|\cal{G}|)\rceil$ il est donc garanti de trouver la caractérisation valide optimale. En effet, il est impossible de distinguer plus de $2^k$ classes avec une projection de dimension $k$.
 \item Une exploration en profondeur partant de la caractérisation maximale (toutes les variables) et recherchant une caractérisation valide de taille $k-1$ dès qu'une caractérisation valide de taille $k$ est trouvée.
\end{itemize}

Ces deux approches garantissent toutes les deux de trouver une caractérisation valide optimale mais en cas d'arrêt de la recherche (temps limite atteint, ...), seule l'exploration en profondeur est capable de fournir une caractérisation valide. De plus, en termes de nombre de caractérisations testées, l'exploration en largeur semble la plus coûteuse (hormis pour les solutions de très petite taille). En effet, il est évident que l'exploration en largeur traitera au moins $\binom{n}{k}$ caractérisations ($n$ étant le nombre de variables totales) alors que pour l'exploration en profondeur il est impossible de prévoir ce nombre (au mieux $n-k$ et au pire $\sum_{i=n}^{i=k}\binom{n}{i}$). Remarquons que l'utilisation conjointe de ces deux méthodes permet de borner la caractérisation optimale.

Ces explorations ont toutes les deux des choix de variables à faire: une heuristique de branchement basée sur un classement des variables de manière statique au début de la recherche. L'ordre utilisé est un calcul d'entropie inspirée par la technique proposée dans \cite{DesVer81} qui privilégie les variables permettant de séparer un groupe par rapport aux autres.

\paragraph{Reformulation en programmation linéaire}

Il existe une reformulation du PCM en programmation linéaire. Nous nous intéressons plus
particulièrement à la minimisation du PCM (MIN-PCM) qui consiste à minimiser la taille  $k$ de la solution. Cette reformulation permet d'obtenir de nouveaux résultats de complexité pour le PCM.


La modélisation du
MIN-PCM peut se faire en programmation entière 0/1 (pseudo-booléen). Le MIN-PCM est reformulé sous la forme d'un problème MIN-ONES. Le problème MIN-ONES est un problème d'optimisation défini comme suit:

\begin{definition}[MIN-ONES]
Soit $ \Phi $ une collection de formules booléennes $\phi_i$ (contraintes)
définie sur  $\cal X$. Le problème consiste à trouver une affectation booléenne  sur $\cal X$ telle que
chaque contrainte soit satisfaite tout en minimisant le nombre de variable vraie dans cette affectation.
\end{definition}

Une formule booléenne est construite avec chaque paire d'entités de groupes différents.

Soit $({\cal X}, {\cal E}, {\cal G})$ une instance du PCM.
Pour toute paire d'entités $\{e_i,e_j\}\in {\cal E}^2$ telle que $e\in g,e'\in
g',g\neq g'$, nous construisons une formule $\phi$ de la manière suivante :\\
$\phi=\bigvee_{x_k \in \{x|x \in {\cal X}, e(x) \neq e'(x)\}} x_k $.

Nous observons que $\phi$ est une clause composée uniquement de littéraux positifs. L'ensemble $\Phi$ de ces formules permet de modéliser entièrement le PCM de manière à le résoudre par programmation entière 0/1.

\begin{center}
\[\begin{array}{l}
Domaine : y_i \in \{0,1\}\\
min : \sum_i y_i\\
s.t.\\
  y_{1}+\cdots+y_{n} \geq 1, \forall(x_{1}\vee\cdots\vee x_{n}) \in
\Phi\\
\end{array}\]
\end{center}

Nous constatons que la transformation d'une instance est bornée par $|{\cal E}|^2$
%\begin{exemple}[Transformation vers MIN-ONES en programmation linéaire]
%\hspace{0.1cm} 
%
%Considérons l'instance :
%\begin{center}
%\begin{tabular}{|c|c||c|c|c|c|}
%\hline
%\multirow{2}{*}{Entités}&\multirow{2}{*}{Groupes}&\multicolumn{4}{c|}{Caractères}
%\\
%&&x1&x2&x3&x4\\
%\hline
%\hline
%e1&g1&1&1&1&0\\
%\hline
%e2&g1&1&1&1&1\\
%\hline
%e3&g2&0&0&1&0\\
%\hline
%e4&g2&0&1&1&1\\
%\hline
%e5&g3&1&1&0&0\\
%\hline
%\end{tabular}
%
%\par La transformation en programmation entière 0/1 est la suivante :
%
%\[\begin{array}{l m{2cm} r}
%\multicolumn{3}{l}{min : \sum_{i=1}^{4} y_i}\\
%\multicolumn{3}{l}{s.t.} \\
%  y_1+y_2\geq 1 &&  c_{\{e1,e3\}}\\
%  y_1+y_4\geq 1&&  c_{\{e1,e4\}}\\
%  y_3 = 1&& c_{\{e1,e5\}}\\
%  y_1+y_2+y_4 \geq 1&&  c_{\{e2,e3\}}\\
%  y_1 = 1&& c_{\{e2,e4\}}\\
%  y_3 + y_4 \geq 1&&  c_{\{e2,e5\}}\\
%  y_1+y_2+y_3\geq 1&& c_{\{e3,e5\}}\\
%  y_1+y_3 + y_4 \geq 1&&  c_{\{e4,e5\}}
%\end{array}\]
%
%\end{center}
%où chaque contrainte ($c_{\{e_i,e_j\}}$) correspond au traitement d'une paire
%d'entités.
%\end{exemple}
%
%Avec la transformation proposée, nous pouvons constater qu'une même contrainte
%ne peut apparaître plusieurs fois quel que soit le nombre de groupes du départ si
%l'instance est d'une part non redondante pour les entités et d'autre part
%satisfiable. Bien évidement cela impose de construire l'ensemble des contraintes
%en imposant un ordre sur les entités.
%
%Dans \cite{creignou2001complexity}, les auteurs établissent que  MIN-ONES en
%présence de clauses ne possédant que des littéraux positifs est contenu dans la
%classe de complexité APX \cite{PapadimitriouY91}, ce qui laisse penser
%que PCM est approximable\cite{Chhel2013}.

%\subparagraph{Réduction d'une instance du problème en programmation linéaire}
%\label{sec::reduc}
%La transformation du PCM afin de le résoudre à l'aide de programmation linéaire
%produit un nombre extrêmement important de clauses. Afin de réduire ce nombre, il
%est envisageable d'utiliser la détection de subsomptions \cite{sub} comme cela
%est pour le problème SAT. Ce mécanisme peut s'appliquer à
%la simplification d'une instance en programmation entière 0/1 car la définition
%particulière des contraintes sur lesquelles nous travaillons le permet
%(clauses ne contenant que des littéraux positifs). Si
%$var(c_{\{e_i,e_j\}})$, avec $i \neq j$, désigne l'ensemble des variables de
%décisions d'une contrainte  produite par la paire d'entités $\{e_i,e_j\}$, nous
%pouvons supprimer toutes contraintes ayant un sous-ensemble de variables
%correspondant à l'ensemble de variables d'une contrainte. Plus formellement,
%nous supprimons toute contrainte $c_{\{e_i,e_j\}}$ telle que $\forall
%c_{\{e_k,e_l\}}$,avec $k \neq l$, nous avons  $var(c_{\{e_k,e_l\}})\subset
%var(c_{\{e_i,e_j\}})$. Comme pour la
%subsomption, il est facile de remarquer que n'importe quelle affectation
%validant la contrainte $c_{\{e_k,e_l\}}$ implique la validité de
%$c_{\{e_i,e_j\}}$.
%
%Dans l'exemple précédent, nous calculons une solution $y_1=1$ et $y_3=1$
%en remarquant que $c_{\{e2,e4\}}$ {resp $c_{\{e1,e5\}}$}  permet de supprimer
%$c_{\{e1,e3\}}$, $c_{\{e1,e4\}}$, $c_{\{e2,e3\}}$, $c_{\{e3,e5\}}$ et
%$c_{\{e4,e5\}}$ (resp. $c_{\{e2,e5\}}$).
%
%Toutefois, il faut remarquer qu'une subsomption n'est présente que lorsque les conditions suivantes sont vérifiées :
%\begin{itemize}
%\item deux entités $e_1$ et $e_2$ d'un même groupe possèdent $x$ variables avec la même valeur~;
%\item au moins une entité d'un autre groupe a des valeurs identiques pour les $n-x$ variables non identiques de  $e_1$ ou de $e_2$ ($n$ étant le nombre total de variables).
%\end{itemize}
%
%La probabilité de réduire une instance en programmation linéaire est donc assez faible ($\frac{1}{2^{2n}}<p<\frac{1}{2^{n}}$) mais pour des instances venant de problèmes réels les conditions de subsomption sont souvent réunies.

\subsubsection{Résolution par recherche locale }
\paragraph{\emph{LS-Proj-Car}}
Utiliser une approche complète peut s'avérer inefficace dans le cas de PCM de grande taille. Une alternative est donc de résoudre le problème à l'aide d'un algorithme de recherche locale \cite{Hoos2004}. L'optimalité des résultats n'est plus garantie mais une solution de bonne qualité est généralement trouvée assez rapidement.

Le principe de l'approche appelée \emph{LS-Proj-Car} est de parcourir l'espace des projections valides en utilisant une liste tabou \cite{Glover1997} afin de limiter les risques de cycles durant la recherche. La fonction de voisinage est définie par l'ensemble des projections ayant une variable de plus ou de moins que la projection actuelle. Le passage d'un voisin à un autre se fait à l'aide de deux opérateurs : $ajout\_var$ qui ajoute aléatoirement une variable à la projection et $supprime\_var$ qui supprime la première variable permettant d'atteindre une projection valide. La recherche commence avec la projection de dimension maximale (toutes les variables) et applique $supprime\_var$ à chaque fois que cela est possible. La liste tabou permet d'éviter d'ajouter et de supprimer une variable plusieurs fois de suite.

Afin de permettre une exploration plus large de l'espace de recherche, plusieurs redémarrages de la recherche sont effectués. Pour ne pas reparcourir les mêmes projections, un mécanisme d'apprentissage permet de garantir un début de recherche différent pour chaque relance. Son principe est de donner un poids inversement proportionnel à l'ordre de sélection lors des recherches précédentes, pour chacune des variables, afin de pouvoir initier les relances suivantes en sélectionnant des variables ayant un poids faible. Ainsi, chaque relance oriente la recherche vers des zones de l'espace de recherche non explorées.
\subsection{Résultats expérimentaux}



%Le tableau \ref{tab:results}, présente plusieurs informations sur les instances ainsi que sur leur résolution par différents algorithmes sont présentées. Les colonnes "entités", "groupes" et "caractères" donnent les propriétés des instances. La colonne "contraintes" fournit le nombre de contraintes nécessaires pour reformuler chaque instance sous forme de programme linéaire, la colonne $r$ désigne le ratio entre le nombre de contraintes et le nombre d'entités et la colonne réduction indique le nombre de contraintes après réduction ("non" si aucune). Les trois dernières colonnes fournissent les résultats obtenus (nombre de variables dans la caractérisation trouvée) pour la résolution en programmation linéaire en utilisant \emph{cplex}  \footnote{\texttt{http://www.ibm.com/software/integration/optimization/cplex-optimizer/}} sur le problème réduit("PL"), pour l'exécution d'\emph{Exact-Proj-Car} ("EPJ") et enfin pour l'application de l'algorithme de recherche locale \emph{LS-Proj-Car} ("LSPC"). 











Afin d'avoir un moyen de comparaison pour nos contributions, nous reproduisons ici les résultats fournis dans \cite{Chhel2013}\footnote{Nous nous sommes servi du code mis à disposition sur \url{http://forge.info.univ-angers.fr/~gh/Idas/Ccd/mcps/}}. Les expérimentations sont faites sur une machine composée d'un processeur Intel Core\up{tm} i7-2620M CPU à 2.70GHz (deux cœurs) avec 4 Go Ram tournant sous Linux 64-bits. L'option de compilation -Ofast est activée pour obtenir notre exécutable.
\begin{itemize}
\item Les instances réelles (raphv, raphy, rarep, rch8 et rch10) sont tirées de problèmes de caractérisation provenant de l'API BioMérieux basée sur des propriétés bio-chimiques de l'espèce Ralstonia et sur des expressions de gènes de virulences de l'espèce Xanthomonas.
\item Les instances aléatoires sont générées en deux étapes, pour un nombre de variables et de groupes fixés. Tout d'abord, chaque entité est construite de manière aléatoire. Ensuite, une certaine proportion des entités est répartie de manière équitable entre tous les groupes et les entités
restantes sont affectées une à une aléatoirement aux groupes. La proportion
d'entités affectées aléatoirement est donnée en pourcentage et est appelée
$bruit$. Une instance aléatoire est notée sous la forme : s$graine$-$bruit$.
\end{itemize}
La colonne PL présente les résultats obtenus par reformulation en programmation linéaire sur le solveur IBM \textit{cplex}\footnote{\url{http://www.ibm.com/software/integration/optimization/cplex-optimizer}}. Les instances marquées par "-" n'ont pas pu être chargées en mémoire car elles sollicitaient plus de 32 Go de RAM. SR présente les tailles des instances après suppression des redondances. Les résultats indiqués dans DIFF sont les résultats affichés dans l'article, différents de ceux obtenus sur notre machine. SH présente les résultats obtenus sans heuristique(heuristique par défaut) avec notre programme sur notre machine. Le temps autorisé pour chaque exécution est de 10 minutes. Les résultats en gras indiquent que les solutions sont optimales.
\begin{center}
\begin{tabular}{|c|c|c|c|c|c|c|}
\hline 
Instances & Entités(SR) & Groupes & Gènes(SR) & PL & EPC(DIFF)(SH) & LSPC(DIFF) \\ 
\hline 
s301-0 & 500 & 30 & 400 & - & 13 & 14 \\ 
\hline 
s326-0 & 500 & 10 & 500 & - & 13 & 14 \\ 
\hline 
s413-30 & 500 & 20 & 600 & - & 13 & \textcolor{blue}{13} (14) \\ 
\hline 
s555-20 & 800 & 20 & 800 & - & 13 & \textcolor{blue}{13} (14) \\ 
\hline 
s625-20 & 500 & 5 & 1000 & - & 13 & \textcolor{blue}{13} (14) \\ 
\hline 
s754-10 & 600 & 10 & 200 & - & 13 & 14 \\ 
\hline 
s882-20 & 600 & 10 & 400 & - & 13 & 14 \\ 
\hline 
s2501-70 & 800 & 10 & 800 & - & \textcolor{blue}{15} (14) & 15 \\ 
\hline 
s31294-50 & 200 & 15 & 1000 & 10 & 10 & 11 \\ 
\hline 
s3836-0 & 1000 & 15 & 1000 & - & 16 & 16 \\ 
\hline 
raphv & 109 (108) & 8 & 155 (68) & \textbf{6} & \textbf{6} & 9 \\ 
\hline 
raphy & 113 (112) & 4 & 155 (70) & \textbf{6} & \textbf{6} & 8 \\ 
\hline 
rarep & 112 & 7 & 155 (72) & \textbf{12} & 95 (59) (\textcolor{blue}{39}) & 14 \\ 
\hline 
rch8 & 132 (56) & 21 & 37 (27) & \textbf{9} & \textbf{9} & 9 \\ 
\hline 
rch10 & 173 (112) & 27 & 98 (86) & \textbf{10} & 27 (15) (\textcolor{blue}{25}) & 15 \\ 
\hline 
\end{tabular} 
\end{center}



Les différences significatives sur les instances rarep et rch10 sont dues au fait que ces résultats ont été obtenu en partant d'une borne supérieur égale aux nombre de gènes divisé par deux. De fait , il apparaît clairement que l'instance rarep n'a pas bénéficié de la suppression de ses redondances\footnote{$72/2<59$}.

Pour les comparaisons à venir sur les résultats qui diffèrent, nous prendrons en compte les résultats en bleu.

Pour l'instance rch10, le résultat sans heuristique étant meilleur que celui avec l'heuristique de EPC \footnote{Ce qui peut s'expliquer par une différence de structure et d'implémentation du code source.} sur notre machine, nous sommes en mesure de penser que le résultat indiqué dans l'article est erroné, donc nous choisissons de conserver le résultat sans heuristique.


%Nous constatons que le nombre de contraintes est beaucoup plus important que le
%nombre d'entités de départ mais il reste faible par rapport à la borne théorique
%exprimée plus haut. Comme nous l'avons mentionné dans la section \ref{sec::reduc}, la réduction pour les instances aléatoires n'a aucun effet car les conditions nécessaires n'ont qu'une très faible probabilité d'être vérifiées. Toutefois, il est intéressant d'observer que pour les instances réelles, le taux de réduction est assez élevé, ce qui est cohérent car les groupes sont constitués d'entités très similaires.
%Le prétraitement nécessaire à la reformulation du problème sous forme de programmation linéaire peut nécessiter de quelques secondes pour les plus petites instances à plusieurs minutes pour les plus grosses.
%
%\'Etant limités en mémoire (32Go sur une machine 64 bits), nous n'avons pu
%lire (chargé en mémoire) les instances aléatoires (noté par "-") avec le solveur \emph{cplex}. Nous remarquons que pour l'instance s31294, cette instance est lu par \emph{cplex} mais malheureusement nous n'avons pu résoudre le problème de manière optimale également pour cause mémoire insuffisante.
%Pour les problèmes réels \emph{cplex}
%est  rapide (quelques secondes) et optimal. Il est d'ailleurs bien plus
%rapide que le solveur \emph{Exact-Proj-Car} (l'ordre de la minute).
%Cependant \emph{Exact-Proj-Car} fournit une borne supérieure en utilisant très peu de
%mémoire (environ 20 Mo) pour les instances aléatoires. \emph{LS-Proj-Car} permet de garantir en toute circonstance une borne optimale facilement calculable avec peu de mémoire et avec une exécution de l'ordre de la seconde, mais il n'obtient jamais la solution optimale.
%
%Les méthodes de résolution que nous avons proposées permettent de traiter des problèmes réels de taille conséquente. Des caractérisations dans le domaine de la biologie végétale ont donc été réalisées et ont conduit au dépôt d'un brevet pour le dépistage du pathovar phaseoli de Xanthomonas Axonopodis \cite{Boureau2012}. Ce test vient d'être validé au niveau européen comme un des tests officiels de dépistage de ce pathovar. Nous avons de plus mis à disposition une page web \footnote{\texttt{http://forge.info.univ-angers.fr/$\sim$gh/Idas/Ccd/mcps/}} permettant aux biologistes d'obtenir aisément une caractérisation de données correspondant au PCM.


%\subsection{Conclusion}
%
%Dans cet article, nous avons défini formellement le problème de caractérisation multiple (PCM) consistant à trouver une formule booléenne qui caractérise chaque groupe d'entités représentées par un ensemble d'interprétations booléennes. Nous avons proposé deux méthodes exactes et une méthode de recherche locale afin de résoudre le PCM.
%
%Une étude de la complexité de ce problème a permis de montrer qu'il n'existait pas d'algorithme FPT pour le résoudre. Une réduction vers le problème SET-COVER amène à la conclusion que le PCM est W[2]-Complet. Cette complexité permet de savoir que les informations apprises lors de la recherche d'une caractérisation de taille $k$ par une méthode exacte sont difficilement utilisables pour la recherche d'une caractérisation de taille $k-1$ ou $k+1$.
%
%Nous avons aussi proposé un codage du problème en programmation linéaire à l'aide d'une transformation vers le problème MIN-ONES. Ceci nous a permis de comparer différentes méthodes de résolutions (méthode exacte, recherche locale, programmation linéaire) pour des instances aléatoires et réelles du PCM. Les instances aléatoires semblent plus difficiles à résoudre que les instances réelles et un travail futur sera d'étudier la génération d'instances afin de définir ce qui rend une instance difficile (nombre d'entités, nombre de groupes, diversité intra et intergroupe). Pour les instances réelles, de bons résultats ont été obtenus. Ils ont pu être valorisés par le dépôt d'un brevet pour le dépistage du pathovar phaseoli de Xanthomonas Axonopodis ainsi que par une validation au niveau européen comme un des tests officiels de dépistage de ce pathovar. 

%\newpage
\section{Contributions}

\subsection{Introduction} 
Cette section présente les démarches de recherche qui ont été effectuées durant le stage.\\
Dans un premier temps, nous proposerons et définirons des critères qui permettent d'identifier si une instance est difficile ou non.\\
Ensuite, nous aborderons la résolution du MIN-PCM avec deux approches différentes :
\begin{itemize}
\item Une recherche exacte qui a la possibilité de prouver la borne minimum du MIN-PCM sur des instances de tailles raisonnables.
\item Une recherche approchée qui à la possibilité de trouver des solutions de bonne qualité mais non nécessairement optimales, en un temps polynômial sur des instances de grandes tailles.
\end{itemize} 
%\textit{TODO: Enfin, nous générons des instances pseudo-aléatoire de différents degrés de difficultés que nous soumettons à nos algorithmes.}


\subsection{Définition d'une instance difficile}
\label{subsectionInstanceDifficile}
Prenons deux instances: une réelle (rch10) et une aléatoire (s3836-0), voici leurs caractéristiques:
\begin{center}
	\begin{tabular}{|c|c|c|c|c|}
	\hline 
	Instances & Entités & Gènes & Résolution PL & Résolution EPC \\ 
	\hline 
	s3836-0 & 1000 & 1000 & - & 16 \\ 
	\hline
	rch10 & 173 & 98 & \textbf{10} & 14 \\ 
	\hline
	\end{tabular} 
\end{center}
La colonne PL présente les résultats obtenus par reformulation en programmation linéaire sur le solveur IBM \textit{cplex}\footnote{\url{http://www.ibm.com/software/integration/optimization/cplex-optimizer}}. Les instances marquées par "-" n'ont pas pu être chargées en mémoire car elles sollicitaient plus de 32 Go de RAM. La colonne EPC présente les résultats du solveur \emph{Exact-Proj-Car} fournis dans \cite{Chhel2013}. Les résultats en gras indiquent que les solutions sont optimales.
\vspace{7mm}

A priori, on peut supposer que l'instance aléatoire est plus difficile à résoudre: elle est bien plus volumineuse que l'instance réelle à tel point qu'elle nécessite plus de 32 Go de RAM pour une résolution en programmation linéaire.

Observons leurs résolutions avec notre algorithme sans heuristique présenté dans la figure \ref{algoMinPCM} page \pageref{algoMinPCM}
\begin{figure}[H]
\centering
	\begin{minipage}[c]{0.49\linewidth}
	\centering
	\begin{tikzpicture}[scale=0.8]
\begin{axis}[
legend entries={rch10,s3836-0},
%legend style={at={(0.5,1.03)},anchor=south},legend columns=3
xlabel={Caractérisation de taille k},
ylabel={Nombre de comparaisons d'entités},
xmin={14},
xmax={40}
]
\addplot +[mark=none] table[x=k,y=nbComp]{./resultats/sh_rch10.dat};
\addplot +[mark=none] table[x=k,y=nbComp]{./resultats/sh_s3836.dat};
\end{axis}
\end{tikzpicture}


	\end{minipage}
	\begin{minipage}[c]{0.49\linewidth}
	\centering
	\input{./figure/sh_rch10_s3836_temps.tex}
	\end{minipage}
\caption{Résolution sans heuristique de rch10 et s3836-0}
\end{figure} 

\begin{remarque}
La lecture des graphiques se fait de la droite vers la gauche. En effet, nous minimisons la taille de la caractérisation au cours de sa résolution.
\end{remarque}
Nous apercevons que l'instance aléatoire est facilement résolue jusqu'à une caractérisation de taille 15. Ce n'est pas le cas de l'instance réelle qui ne peut plus caractériser en un temps raisonnable à partir d'une caractérisation de taille 25. Ce type d'observation étant \textbf{systématique} quelque soit les caractéristiques des instances réelles ou aléatoires comparées, nous pouvons alors affirmer que la taille d'une instance ne suffit pas à elle seule pour définir sa difficulté. Dès lors, nous nous posons les deux questions suivantes:\\

\begin{itemize}
\item \textbf{Qu'est ce qui peut bien être à l'origine de cette différence de résolution entre une instance aléatoire et une instance réelle?}
\item \textbf{Existe il une méthode permettant de définir si une instance est difficile à résoudre ou non ?}\\
\end{itemize}
Afin de répondre à ces questions, nous définissons les notions suivantes:

\begin{definition}
Le \textbf{masque $M$ d'un groupe $g$} correspond à la moyenne des présences/absences des gènes pour chaque entité du groupe.\\
Formellement, soit $M_g$ le masque d'un groupe $g$, $g \in \mathcal{G}$, $M_g[i]$ la valeur de $M_g$ en position $i$, $i \in [1,\ldots,|\mathcal{X}|]$,
% $|\mathcal{X}|$ étant le nombre de gènes de l'instance après la suppression des redondances,
$$\forall i \in  [1, |\mathcal{X}|], M_g[i]= \frac{\sum_{i=1}^{|\mathcal{G}|}e_i}{|\mathcal{G}|} $$
\end{definition}

\begin{definition}
Le \textbf{ratio $r$ d'un masque $M$} correspond au pourcentage de valeur entière (0/1) présente dans le masque.\\
Formellement, soit $M_g$ le masque d'un groupe $g$, $r_g(I)$ le ratio du groupe $g$ dans l'image $I$, $g \in \mathcal{G}$,
\begin{center}
$$ r_g(I)=\frac{|\{i \in [1,\ldots,|\mathcal{X}|] / M_g[i]=1 \lor M_g[i]=0 \}|}{|\mathcal{X}|}$$
\end{center}
\end{definition}

%\subsubsection*{Exemple :}
\begin{exemple}{Masque et ratio d'un groupe\\}
\begin{center}
\begin{tabular}{|c|c|c|c|c|c|c|c|c|c|c|}
\hline 
\backslashbox{Entités}{Gènes} & g0 & g1 & g2 & g3 & g4 & g5 & g6 & g7 & g8 & g9 \\ 
\hline 
e1 & 1 & 1 & 0 & 1 & 1 & 1 & 0 & 0 & 0 & 1 \\ 
\hline 
e2 & 1 & 1 & 0 & 1 & 1 & 1 & 0 & 1 & 0 & 1 \\ 
\hline 
e3 & 1 & 1 & 0 & 0 & 0 & 1 & 0 & 0 & 0 & 0 \\ 
\hline 
e4 & 1 & 1 & 0 & 1 & 0 & 1 & 0 & 0 & 0 & 0 \\ 
\hline 
e5 & 1 & 1 & 0 & 1 & 1 & 1 & 0 & 1 & 0 & 0 \\ 
\hline 
\hline
Masque & 1 & 1 & 0 & 0.8 & 0.6 & 1 & 0 & 0.4 & 0 & 0.4 \\
\hline
\end{tabular}
\end{center}
Le ratio $r$ de ce groupe est : \\
$r=6/10$\\
soit  $r=0.6$
\end{exemple}

\begin{definition}
L'\textbf{image $I$ d'une instance} est une matrice en deux dimensions de taille $|\mathcal{G}|*|\mathcal{X}|$ où chaque ligne correspond au masque de chacun des groupes de l'instance.\\
Formellement, soit $I_g$ la ligne g de la matrice $I$ correspondant à l'image de l'instance $\mathcal{I}$, $M_g$ le masque du groupe $g$ de l'instance $\mathcal{I}$,
$$\forall g \in [1,|\mathcal{G}|], I_g=M_g$$
\end{definition}


\begin{definition}
Le \textbf{taux de similarité $\mathcal{T}_j$ d'un gène $j$} correspond à la moyenne des valeurs de la colonne $j$ sur l'image $I$ d'une instance $\mathcal{I}$.\\
Formellement, soit $I$ l'image d'une instance $\mathcal{I}$, $i$ la $i$\up{ème} ligne de $I$, $j$ la $j$\up{ème} colonne de $I$, $I_{ij}$ est la valeur dans $I$ en ligne $i$ et en colonne $j$, $\mathcal{T}_j(I)$ le taux de similarité du gène $j$ dans l'image $I$,
$$ \text{Soit } X=\frac{\sum_{i=1}^{|\mathcal{G}|} I_{ij}}{\mathcal{G}} $$ 
$$\text{Si } X<0.5 \text{ alors } \mathcal{T}_j(I)=(0.5-X)*2 $$
$$\text{sinon }\mathcal{T}_j(I)=(0.5-(1-X))*2$$ 
\end{definition}
Ainsi formulé, $\mathcal{T}_j(I) \in [0,1]$, et, plus le taux de similarité d'un gène est élevé, plus sa présence(resp. abscence) dans l'instance est forte.

\begin{definition}
Le \textbf{coefficient de difficulté $\rho(\mathcal{I})$ d'une instance $\mathcal{I}$}, correspond à la moyenne des taux de similarité $\mathcal{T}$ d'une instance.\\
Formellement, soit $I$ l'image d'une instance $\mathcal{I}$, $j$ la $j$\up{ème} colonne de $I$, $\mathcal{T}_j(I)$ le taux de similarité globale du gène $j$ dans l'image $I$,
$$ \rho(\mathcal{I})=\frac{\sum_{j=1}^{|\mathcal{X}|}\tau_j(I)}{|\mathcal{X}|} $$
\end{definition}

\begin{definition}
Le \textbf{coefficient de difficulté $\sigma(\mathcal{I})$ d'une instance $\mathcal{I}$}, correspond au complémentaire de la moyenne des ratios des masques d'une instance\footnote{La moyenne des ratios des masques d'une instance $\in [0,1]$ }. Nous utilisons le complémentaire de façon à ce que l'interprétation du coefficient $\sigma$ soit calqué sur celui de $\rho$.
Formellement, soit $I$ l'image d'une instance $\mathcal{I}$, $i$ la $i$\up{ème} ligne de $I$, $r_i(I)$ le ratio du groupe $i$ dans l'image $I$,
$$ \sigma(\mathcal{I})=1-\frac{\sum_{i=1}^{|\mathcal{G}|}r_i(I)} {|\mathcal{G}|} $$
\end{definition}

\begin{definition}
\textbf{Le coefficient $\Delta\cal{T}(\mathcal{I})$ d'une instance $\mathcal{I}$} correspond à l'écart type des taux de similarité $\cal{T}$ des gènes.
Formellement, soit $I$ l'image d'une instance $\mathcal{I}$, $i$ la $i$\up{ème} ligne de $I$, $r_i(I)$ le ratio du groupe $i$ dans l'image $I$, le coefficient de difficulté $\rho(\mathcal{I})$ de l'instance $\cal{I}$, 
$$\Delta\mathcal{T}(\mathcal{I})=\sqrt{\frac{\sum_{j=1}^{|\mathcal{X}|} (\rho(\mathcal{I})-\mathcal{T}_j)^2}{|\mathcal{X}|}}$$
$$\Delta\mathcal{T}(\mathcal{I}) \in [0,\frac{1}{2}]$$
\end{definition}

\begin{remarque}
	Si $\rho(\mathcal{I}) \simeq 0$ (resp. 1) alors $\Delta\cal{T}(\mathcal{I}) \simeq $ 0 car $\rho(\mathcal{I})$ est la moyenne des taux de similarité $\cal{T}$ d'une instance et si cette moyenne est proche de 0 (resp. 1), cela signifie que la majorité des $\cal{T}$ sont proches de 0 (resp. 1) et donc que leur écart type ($\Delta\cal{T}$) est proche de 0.
	\label{remDeltaTau}
\end{remarque}


Reprenons nos deux instances rch10 et s3836-0 et calculons leurs coefficients de difficultés:

\begin{center}
\begin{tabular}{|c|c|c|c|}
\hline 
Instances & $\Delta\cal{T} $ & $\rho$ & $\sigma$ \\ 
\hline 
s3836-0 & 0.019 & 0.024 & 1 \\ 
\hline
rch10 & 0.238 & 0.626 & 0.094 \\ 
\hline
\end{tabular} 
\end{center}

Nous observons que le coefficient de difficulté $\rho$ semble plus significatif que $\sigma$ pour déterminer la difficulté d'une instance, mais nous ne sommes pas en mesure d'indiquer dans quel proportion. Cependant les travaux de \cite{Chhel2013} nous indiquent que seule une heuristique sur le choix des variables est en mesure de pouvoir améliorer un algorithme de recherche exacte. Cela nous conforte dans l'idée que $\rho$ a plus d'influence que $\sigma$ sur la difficulté d'une instance. Nous pouvons ainsi formuler les deux observations suivantes:

\begin{observation}
Une instance dont le coefficient de difficulté $\rho$ est proche de 1 est une \textbf{instance difficile} à résoudre.
\end{observation}

\begin{observation}
Une instance dont le coefficient de difficulté $\rho$ est proche de 1 et dont le coefficient de difficulté $\sigma$ est proche de 1 est une \textbf{instance très difficile} à résoudre.
\end{observation}

Dès lors, nous pourrions penser qu'il suffit de trouver une heuristique basée sur le choix des gènes en fonction de leurs taux de similarité $\cal{T}$ pour outrepasser la difficulté émise par le coefficient $\rho$. Ce n'est pas le cas. En effet, si l'écart type entre les différents taux $\cal{T}$ d'une instance est faible, alors le critere $\cal{T}$ ne permet pas de différencier significativement les gènes. Nous faisons donc la proposition suivante: 

\begin{proposition}
Une heuristique basée sur le choix des gènes en fonction de leurs taux de similarité $\cal{T}$ n'a aucune influence sur des instances dont le coefficient $\Delta\cal{T}$ est proche de 0.
\label{propDelta}
\end{proposition}

Il y a encore au moins un indicateur sur la difficulté des instances: il s'agit de la taille de celles-ci. Nous rejoignons l'idée émise dans \cite{Chhel2013}, selon laquelle la taille d'une instance est caractérisée par son nombre de gènes et d'entités mais pas par son nombre de groupe. Cependant, lorsque le coefficient de difficulté $\sigma$ est proche de 0, nous devrions pouvoir être en mesure d'outrepasser cette difficulté puisque nous pouvons réduire le parcours d'un très grand nombre d'entités par l'utilisation des masques, nous présentons ce cas de figure dans l'exemple \ref{exempleDifficulté}. Nous faisons donc la proposition suivante:



\begin{proposition}
La taille d'une instance est une information sur la difficulté de sa résolution. Il n'existe pas d'heuristique permettant d'outrepasser cette difficulté lorsque le coefficient $\sigma$ de l'instance est proche de 1.
\end{proposition}

\begin{exemple}[Difficultés d'une instance]
\label{exempleDifficulté}
Supposons qu'il existe une instance composé de 3 groupes de 100 entités que l'on souhaite caractériser avec un ensemble de $n$ gènes. Supposons également que cette instance ait pour coefficients de difficulté $\sigma \simeq 0$, $\rho \simeq 0.7$ et $\Delta\cal{T} \simeq$ 0. D'après notre proposition \ref{propDelta}, il n'existe pas d'heuristique efficace pour outrepasser la difficulté émise par $\rho$.

Une recherche standard pour une caractérisation de taille $k$ effectue au maximum environ $ (100*200 + 100*100) * C_n^k * k$ comparaisons. Comme $\sigma \simeq 0$ , nous pouvons exploiter les masques(voir paragraphe \ref{heuristiqueTabou} page \pageref{heuristiqueTabou}). En les utilisant, nous effectuons au maximum environ $ (1\up{+} * 2\up{+} + 1\up{+}*1\up{+}) * C_n^k * k $ comparaisons.

On remarque que dans ce cas précis, l'utilisation des masques à une influence sur la résolution du problème.

Supposons maintenant que $\Delta\cal{T} \simeq $ 0.5, nous présumons dès lors que l'efficacité d'une heuristique basée sur l'utilisation des masques serait bien dérisoire face à une heuristique basée sur le choix des gènes en fonction de leurs taux de similarité $\cal{T}$\footnote{Quoique cette assertion dépend de la valeur de $n$}. En effet, une heuristique basée sur le choix des gènes en fonction de leurs taux de similarité $\cal{T}$ travaille à réduire le facteur à caractère exponentiel $C_n^k$ alors que l'utilisation des masques vise à réduire un facteur à caractère polynômial. Cela confirme une fois de plus que $\rho$ à plus d'influence que $\sigma$ pour caractériser une instance.
\end{exemple}


%Les observations sur notre jeu de 15 instances nous permettent de conclure que $\sigma$ en particulier, et $\rho$ dans une moindre mesure, nous permettent de définir ce qu'est une instance difficile. 

\subsubsection{Expérimentations}
La colonne PL présente les résultats obtenus par reformulation en programmation linéaire sur le solveur IBM \textit{cplex}\footnote{\url{http://www.ibm.com/software/integration/optimization/cplex-optimizer}}. Les instances marquées par "-" n'ont pas pu être chargées en mémoire car elles sollicitaient plus de 32 Go de RAM. La colonne EPC présente les résultats du solveur \emph{Exact-Proj-Car} fournis dans \cite{Chhel2013}. La colonne LSPC présente les résultats de la recherche locale de\emph{Local-Search-Proj-Car} fournis dans \cite{Chhel2013}. Les expérimentations sont faites sur une machine composée d'un processeur Intel Core\up{tm} i7-2620M CPU à 2.70GHz (deux cœurs) avec 4 Go Ram tournant sous Linux 64-bits. L'option de compilation -Ofast est activée pour obtenir notre exécutable. Le temps autorisé pour chaque exécution est de 10 minutes. Les résultats en gras indiquent que les solutions sont optimales.

Nous présentons ici les instances avec leurs coefficients de difficultés respectifs:
\begin{center}
\begin{tabular}{|c|c|c|c|c|c|c|c|c|}
\hline 
Instances & Entités & Gènes & $\Delta\cal{T}$ & $\rho$ & $\sigma$ & PL & EPC & LSPC \\ 
\hline 
s301-0 & 500 & 400 & 0.025 & 0.034 & 0.999 & - & 13 & 14 \\ 
\hline 
s326-0 & 500 & 500 & 0.026 & 0.033 & 1 & - & 13 & 14 \\ 
\hline 
s413-30 & 500 & 600 & 0.027 & 0.035 & 1 & - & 13 & 13 \\ 
\hline 
s555-20 & 800 & 800 & 0.029 & 0.039 & 0.999 & - & 13 & 13 \\ 
\hline 
s625-20 & 500 & 1000 & 0.027 & 0.035 & 1 & - & 13 & 13 \\ 
\hline 
s754-10 & 600 & 200 & 0.027 & 0.034 & 1 & - & 13 & 14 \\ 
\hline 
s882-20 & 600 & 400 & 0.024 & 0.032 & 1 & - & 13 & 14 \\ 
\hline 
s2501-70 & 800 & 800 & 0.024 & 0.033 & 1 & - & 15 & 15 \\ 
\hline 
s31294-50 & 200 & 1000 & 0.049 & 0.065 & 0.993 & 10 & 10 & 11 \\ 
\hline 
s3836-0 & 1000 & 1000 & 0.019 & 0.024 & 1 & - & 16 & 16 \\ 
\hline 
raphv & 108 & 68 & 0.302 & 0.588 & 0.419 & \textbf{6} & \textbf{6} & 9 \\ 
\hline 
raphy & 112 & 70 & 0.294 & 0.609 & 0.668 & \textbf{6} & \textbf{6} & 8 \\ 
\hline 
rarep & 112 & 72 & 0.295 & 0.651 & 0.502 & \textbf{12} & 39 & 14 \\ 
\hline 
rch8 & 56 & 27 & 0.339 & 0.569 & 0.067 & \textbf{9} & \textbf{9} & 9 \\ 
\hline 
rch10 & 112 & 86 & 0.238 & 0.626 & 0.094 & \textbf{10} & 25 & 15 \\ 
\hline 
\end{tabular} 
\end{center}

Toute les instances aléatoires ont un coefficient $\rho$ proche de 0 et donc un coefficient $\Delta\cal{T}$ proche de 0 (voir remarque \ref{remDeltaTau}). Mais elles ont pour difficulté leurs coefficients $\sigma$ qui est proche de 1, cela signifie que nous ne pourrons pas réduire de façon significative leurs temps de résolution. L'instance rch8 a un coefficient $\sigma$ proche de 0 ainsi qu'un coefficient $\rho$ moyennement élevé, de plus, elle est de faible taille, c'est l'instance qui semble la plus facile à résoudre. L'instance raphv et raphy ont un ordre de difficulté similaire.  L'instance rch10 a un très faible coefficient $\sigma$, elle doit être plus facile à résoudre que l'instance rarep pour qui le coefficient $\sigma$ est moyennement élevé. Ces deux dernières semblent être les instances les plus difficiles à résoudre. Cette série d'observations et de raisonnement est corroborée par les résultats obtenus par \cite{Chhel2013}.

\subsection{Recherche exacte}
\label{rechercheExacte}
\subsubsection{Introduction}
Nous présentons les heuristiques ayant été mises en place pour la résolution d'instance MIN-PCM. Chaque heuristique est abordée de façon indépendante. Une comparaison entre toutes les heuristiques est présentée dans la sous section \ref{sectCompar} page \pageref{sectCompar}. Les comparaisons se font sur l'instance réelle rch10 et l'instance pseudo-aléatoire s3836-0.

Une comparaison entre la meilleure combinaison d'heuristique obtenue et l'heuristique "CCD" proposée par \cite{Chhel2013} avec le solveur \textit{Exact-Proj-Car} est présentée à la fin de la cette section page \pageref{sectionCompare}.
\subsubsection{Algorithmes généraux de résolution}

Dans cette sous section, nous présentons une méthode générale qui permet de résoudre le MIN-PCM.

\begin{algorithm}
	\textbf{booléen minimise\_caractérisation\_instance ($Groupes$, $n$)}\\
	\tcp
	{
		$Groupes$ est un ensemble contenant tous les groupes de l'instance\\
		$n$ correspond aux nombres de gènes de l'instance
	}
	$caracterise \leftarrow$ VRAI\\
	$k \leftarrow n$ \tcp{k correspond au nombre de gènes pouvant caractériser l'instance}
	\Tq {$caracterise =$ VRAI}
	{
		$k \leftarrow k-1$\\
		$caracterise \leftarrow $caractérise\_instance ($Groupes$, $k$, $n$)\\
	}
	afficher("La caractérisation minimale est de taille " $k+1$)\\
	\caption{Algorithme de minimisation du problème de caractérisation multiple}
	\label{algoMinPCM}
\end{algorithm}

\begin{algorithm}
	\textbf{booléen caractérise\_instance ($Groupes$, $k$, $n$)}\\
	\tcp
	{
		$Groupes$ est un ensemble contenant tous les groupes de l'instance\\
		$k$ correspond au nombre de gène souhaité pour la caractérisation\\
		$n$ correspond aux nombres de gènes de l'instance
	}	
	\PourTous {$combinaison$ de $C_n^k$ \tcp{Boucle 1}}
	{
		$DejaVus \leftarrow \{ \}$\\
		$caracteriseTemp \leftarrow$ VRAI\\
		\PourTous {$gRef \in Groupes$ \tcp{Boucle 2}}
		{
			$DejaVus \leftarrow DejaVus \cup \{gRef\}$\\
			\PourTous {$gComp \in Groupes \backslash DejaVus$ \tcp{Boucle 3}}
			{
				\Si {$\neg$ caractérise\_groupes($combinaison$,$gRef$,$gComp$)}
					{
					$caracteriseTemp \leftarrow$ FAUX\\
					Sortir de la boucle
					}
			}
			\Si {$caracteriseTemp=$FAUX} {Sortir de la boucle}
		}
		\Si {$caracteriseTemp=$ VRAI} {\Retour {VRAI}}
		\tcp{Sinon combinaison suivante}
	}
	afficher("Il n'existe pas de caractérisation de taille " $k$)\\
	\Retour{FAUX}
	\caption{Algorithme de caractérisation d'une instance MIN-PCM pour une taille $k$}
	\label{algoCaractInstance}
\end{algorithm}

\begin{algorithm}
	\textbf{booléen caractérise\_groupes ($combinaison$,$g1$,$g2$)}\\
	\tcp{$e1$ et $e2$ sont les entités des groupes $g1$ et $g2$ }				
	\PourTous {$e1 \in g1$ }
	{
		\PourTous {$e2 \in g2$}
		{
			$temp \leftarrow$ FAUX\\
			\PourTous {$i \in combinaison$}
			{
				\Si {$e1[i] \neq e2[i]$} 
					{
						$temp \leftarrow$ VRAI\\
						Sortir de la boucle
					}
			}
			\Si {$temp = $FAUX \tcp{$g1$ et $g2$ ne sont pas caractérisable avec $combinaison$}} {\Retour {FAUX}}
		}
	}
	\Retour {VRAI}	
	\caption{Algorithme de caractérisation de groupes}
	\label{algoCaractGroupes}
\end{algorithm}

L'algorithme présenté dans la figure \ref{algoMinPCM} cherche la plus petite caractérisation possible pour une instance donnée. Il prend en entrée un ensemble de groupe $Groupes$ qui contient tous les groupes de l'instance et un entier $n$ qui correspond aux nombres de gènes de l'instance. Au début de l'algorithme, une variable booléenne $caracterise$ est instanciée à $VRAI$ et un entier $k$, qui correspondra au nombre de gènes pouvant caractériser une instance, est instancié à $n$. En effet, la caractérisation la plus évidente et immédiate est la caractérisation de taille $n$. Nous entrons ensuite dans une boucle qui va avoir pour rôle de minimiser le plus possible la variable $k$. On sort de cette boucle lorsque plus aucune caractérisation n'est possible, c'est à dire lorsque nous avons trouvé la borne minimale du MIN-PCM. Pour cela, il faut que la variable $caracterise$ soit instanciée à faux, ceci ne peut se faire que par l'appel à l'algorithme \emph{caractérise\_instance} présenté dans la figure \ref{algoCaractInstance}.


Cet algorithme prend en entrée un ensemble de groupe $Groupes$ qui contient tous les groupes de l'instance, un entier $k$, qui correspond au nombre de gènes souhaités pour la caractérisation et un entier $n$ qui correspond au nombre de gènes de l'instance. Nous entrons dans une première boucle (boucle 1) qui génère les combinaisons $combinaison$ de $C_n^k$. La variable $DejaVus$ est instanciée comme étant un ensemble vide de groupe. Une variable booléenne $caracteriseTemp$ est initialisée à $VRAI$. Nous entrons alors dans une seconde boucle (boucle 2), qui va ouvrir la boucle 3 afin de tester si tous les groupes de $Groupes$ peuvent caractériser les groupes de $Groupes \backslash DejaVus$, auquel cas, nous avons une caractérisation de taille $k$ avec la combinaison $combinaison$. La mise à jour de $DejaVus$ dans la boucle 2 a pour effet de ne pas tester plusieurs fois des groupes ayant déjà été testés. Dès lors que la variable $caracteriseTemp$ est instanciée à $FAUX$, un mécanisme de sortie de boucle permet de tester directement la prochaine combinaison de $C_n^k$. Si en sortie de boucle 2, la variable $caracteriseTemp$ est toujours instanciée à $VRAI$, cela signifie que la combinaison courante caractérise l'instance, l'algorithme s'arrête alors en retournant la valeur booléenne $VRAI$. Si nous arrivons en sortie de boucle 1, cela signifie que toutes les combinaisons de $C_n^k$ ont été testées et que aucune d'entre elles n'a pu caractériser l'instance. Dans ce cas, l'algorithme s'arrête en retournant la valeur booléenne $FAUX$. Dans cet algorithme, nous faisons appel à la fonction \emph{caractérise\_groupes} pour savoir si deux groupes différents peuvent être caractérisés avec une combinaison $combinaison$. Nous présentons cette fonction dans la figure \ref{algoCaractGroupes}.


Cette fonction permet de savoir si deux groupes $g1$ et $g2$ sont caractérisables avec une combinaison $combinaison$ donnée. Nous notons $e1$(resp.$e2$) la variable qui va contenir une après l'autre les entités du groupe $g1$(resp.$g2$). Les entités de chacun des groupes ($e1$ et $e2$) sont comparées deux à deux sur la présence/absence des gènes définit par la combinaison $combinaison$. Si deux entités sont identiques sur ces gènes alors on ne peut pas caractériser les deux groupes avec cette combinaison(et a fortiori, on ne peut pas non plus caractériser l'instance avec cette combinaison). Dès lors, l'algorithme s'arrête immédiatement en retournant la valeur boolénne $FAUX$. Si toutes les comparaisons possibles entre $e1$ et $e2$ ont été faites, cela signifie que les deux groupes $g1$ et $g2$ sont caractérisables avec la combinaison $combinaison$. L'algorithme s'arrête alors en retournant la valeur booléenne $VRAI$.



\subsubsection{Heuristique de tri sur les gènes }
\paragraph{Tri des gènes par taux de similarité $\mathcal{T}$}
Les outils mis en place dans la sous-section \ref{subsectionInstanceDifficile} nous permettent de mettre en place une heuristique basée sur le trie des gènes: nous ordonnons les gènes par ordre croissant sur leur taux de similarité $\mathcal{T}$. Dès lors, nous parcourons en priorité les gènes présentants un faible taux de similarité. Ce trie par $\mathcal{T}$ permet d'explorer en priorité les gènes susceptible de caractériser une instance. Le coût de calcul de ce trie est insignifiant et est effectué de façon unique avant le lancement de l'algorithme de résolution.

\subparagraph{Résultats}

\begin{figure}
\centering
	\begin{minipage}[c]{0.49\linewidth}
	\centering
	\input{./figure/rch10_sh_vs_tau_nbComp.tex}
	\end{minipage}
	\begin{minipage}[c]{0.49\linewidth}
	\centering
	\begin{tikzpicture}[scale=0.8]
\begin{axis}[
legend entries={rch10,rch10 $\mathcal{T}$},
%legend style={at={(0.5,1.03)},anchor=south},legend columns=3
xlabel={Caractérisation de taille k},
ylabel={Temps d'éxécution en seconde},
xmin={10},
xmax={40}]
\addplot +[mark=none] table[x=k,y=temps]{./resultats/sh_rch10.dat};
\addplot +[mark=none] table[x=k,y=temps]{./resultats/tau_rch10.dat};
\end{axis}
\end{tikzpicture}
	\end{minipage}
\caption{Heuristique $\mathcal{T}$ sur rch10}
\label{tauRch10}
\end{figure}

\begin{figure}
\centering
	\begin{minipage}[c]{0.49\linewidth}
	\centering
	\input{./figure/s3836_sh_vs_tau_nbComp.tex}
	\end{minipage}
	\begin{minipage}[c]{0.49\linewidth}
	\centering
	\begin{tikzpicture}[scale=0.8]
\begin{axis}[
legend entries={s3836-0,s3836-0 $\mathcal{T}$},
%legend style={at={(0.5,1.03)},anchor=south},legend columns=3
xlabel={Caractérisation de taille k},
ylabel={Temps d'éxécution en seconde},
xmin={15},
xmax={40}]

\addplot +[mark=none] table[x=k,y=temps]{./resultats/sh_s3836.dat};
\addplot +[mark=none] table[x=k,y=temps]{./resultats/tau_s3836.dat};
\end{axis}
\end{tikzpicture}
	\end{minipage}
\caption{Heuristique $\mathcal{T}$ sur s3836-0}
\label{taus3836}
\end{figure}

Sur l'instance réelle rch10, nous constatons que le nombre de comparaisons d'entités  et le temps d'éxécution \textbf{sont considérablement diminuées} par l'heuristique(figure \ref{tauRch10}). Sur l'instance aléatoire, nous obervons que cette heuristique a une faible influence négative(figure \ref{taus3836}). Ce phénomène s'explique par un coefficient $\Delta\cal{T}$ très bas (0.019). Il résulte de cette observation que \textbf{toute instance ayant un coefficient $\Delta\cal{T} \simeq $ 0 ne pourra pas bénéficier, lors de sa résolution, des avantages d'une heuristique de tri sur les gènes}.
 
La possibilité que l'influence soit négative nous permet de déduire que \textbf{le tri des gènes par $\mathcal{T}$ n'est pas un tri optimal}. 


\subsubsection{Heuristique de tri sur les groupes}
\paragraph{Tri des groupes par taux de similarité $\Gamma$}
Lors de la précédente sous-section, nous avons ordonnés statiquement les gènes d'après leurs $\mathcal{T}$. Ici, nous souhaitons ordonner les entités d'une instance. Nous définissons alors le concept de taux de similarité entre deux groupes.

\begin{definition}
Le \textbf{taux de similarite $\Gamma(g,g')$ entre deux groupes $g$ et $g'$ } correspond à la moyenne des sommes des moyennes des valeurs du masque de $g$ et $g'$.\\
Formellement, soit $g$ et $g'$ deux groupes d'une instance $\mathcal{I}$, $M_g$(resp. $M_{g'}$) le masque du groupe $g$(resp. $g'$), $M_g[i]$(resp. $M_{g'}[i]$) la valeur du masque du groupe $g$(resp. $g'$) en position $i$,
$$ \Gamma(g,g')= \frac{\sum_{i=1}^{|\mathcal{X}|}(\frac{M_g[i]+M_{g'}[i]}{2})}{|\mathcal{X}|}$$
\end{definition}

%\begin{definition}
%Le \textbf{taux de similarité globale $\Gamma(g)$ d'un groupe $g$} correspond à la moyenne de ses taux de similarité locaux.\\
%Formellement, soit $g$ le groupe d'une instance $\mathcal{I}$,
%$$ \Gamma(g)=\frac{\sum_{g'=1}^{|\mathcal{G}|}\tau(g,g') /g \neq g'}{|\mathcal{G}|-1} $$
%\end{definition}

Une instance est caractérisable si tous les groupes qui la composent sont caractérisables entre eux. Nous proposons de regarder en priorité les groupes ayant le plus de chance de ne pas caractériser l'instance car dès lors que deux groupes ne sont pas caractérisables entre eux, nous pouvons arrêter la recherche sur la combinaison courante et tester la suivante, nous faisons ainsi l'économie de tester les groupes facilement caractérisables entre eux(qui sont de toute manière testés lorsque la combinaison courante est une caractérisation valide). Nous construisons alors une liste de paire de groupe $[g,g']$ de tel manière que la liste soit triée par ordre décroissant de $\Gamma(g,g')$ et qu'elle contienne tous les groupes qu'il faut caractériser entre eux pour caractériser une instance. En effet, plus deux groupes sont similaires entre eux, plus leur chance de ne pas caractériser l'instance est forte.

Nous proposons cette heuristique de façon statique (tout les gènes sont pris en compte, la liste est créée de façon unique avant le lancement de l'algorithme) et de façon dynamique(à chaque combinaison, seuls les gènes concernés par la combinaison sont pris en compte pour le calcul de $\Gamma(g,g')$, nous construisons donc autant de liste qu'il y a de combinaisons testées)



\subparagraph{Résultats}


\begin{figure}
\centering
	\begin{minipage}[c]{0.49\linewidth}
	\centering
	\begin{tikzpicture}[scale=0.8]
\begin{axis}[
legend entries={rch10, rch10 $\Gamma$ statique},
%legend style={at={(0.5,1.03)},anchor=south},legend columns=3
xlabel={Caractérisation de taille k},
ylabel={Nombre de comparaisons d'entités},
xmin={24},
xmax={40},
%title={Résolution sans heuristique de rch10 et s3836-0 sur les comparaisons}
]
\addplot +[mark=none] table[x=k,y=nbComp]{./resultats/sh_rch10.dat};
\addplot +[mark=none] table[x=k,y=nbComp]{./resultats/gamma_rch10.dat};
\end{axis}
\end{tikzpicture}
	\end{minipage}
	\begin{minipage}[c]{0.49\linewidth}
	\centering
	\input{./figure/rch10_sh_vs_gamma_temps.tex}
	\end{minipage}
\caption{Heuristique $\Gamma$ statique sur rch10}
\label{GammaStatiqueRch10}
\end{figure}

\begin{figure}
\centering
	\begin{minipage}[c]{0.49\linewidth}
	\centering
	\input{./figure/rch10_sh_vs_gamma_dynamique_nbComp.tex}
	\end{minipage}
	\begin{minipage}[c]{0.49\linewidth}
	\centering
	\input{./figure/rch10_sh_vs_gamma_dynamique_temps.tex}
	\end{minipage}
\caption{Heuristique $\Gamma$ dynamique sur rch10}
\label{GammaDynamiqueRch10}
\end{figure}

\begin{figure}
\centering
%	\begin{minipage}[c]{0.49\linewidth}
%	\centering
	\input{./figure/rch10_gamma_vs_gamma_dynamique_nbComp_zoom.tex}
%	\end{minipage}
\caption{Zoom sur comparaisons heuristique $\Gamma$ statique et heuristique $\Gamma$ dynamique}
\label{compGammaZoom}
\end{figure}

\begin{figure}
\centering
	\begin{minipage}[c]{0.49\linewidth}
	\centering
	\begin{tikzpicture}[scale=0.8]
\begin{axis}[
legend entries={rch10 $\Gamma$, rch10 $\Gamma$ dynamique},
%legend style={at={(0.5,1.03)},anchor=south},legend columns=3
xlabel={Caractérisation de taille k},
ylabel={Nombre de comparaisons d'entités},
xmin={24},
xmax={50},
]
\addplot +[mark=none] table[x=k,y=nbComp]{./resultats/gamma_rch10.dat};
\addplot +[mark=none] table[x=k,y=nbComp]{./resultats/gamma_dynamique_rch10.dat};
\end{axis}
\end{tikzpicture}
	\end{minipage}
	\begin{minipage}[c]{0.49\linewidth}
	\centering
	\begin{tikzpicture}[scale=0.8]
\begin{axis}[
legend entries={rch10, rch10 $\Gamma$, rch10 $\Gamma$ dynamique},
%legend style={at={(0.5,1.03)},anchor=south},legend columns=3
xlabel={Caractérisation de taille k},
ylabel={Temps d'éxécution en seconde},
xmin={24},
xmax={40}]
%\addplot +[mark=none] table[x=k,y=temps]{./resultats/sh_rch10.dat};
\addplot +[mark=none] table[x=k,y=temps]{./resultats/gamma_rch10.dat};
\addplot +[mark=none] table[x=k,y=temps]{./resultats/gamma_dynamique_rch10.dat};

\end{axis}
\end{tikzpicture}
	\end{minipage}
\caption{Comparaisons heuristique $\Gamma$ statique et heuristique $\Gamma$ dynamique sur rch10}
\label{compGamma}
\end{figure}

\begin{figure}
\centering
	\begin{minipage}[c]{0.49\linewidth}
	\centering
	\begin{tikzpicture}[scale=0.8]
\begin{axis}[
legend entries={rch10, rch10 $\Gamma$, rch10 $\Gamma$ dynamique},
%legend style={at={(0.5,1.03)},anchor=south},legend columns=3
xlabel={Caractérisation de taille k},
ylabel={Nombre de comparaisons d'entités},
xmin={15},
xmax={27},
%title={Résolution sans heuristique de rch10 et s3836-0 sur les comparaisons}
]
\addplot +[mark=none] table[x=k,y=nbComp]{./resultats/sh_s3836.dat};
\addplot +[mark=none] table[x=k,y=nbComp]{./resultats/gamma_s3836.dat};
\addplot +[mark=none] table[x=k,y=nbComp]{./resultats/gamma_dynamique_s3836.dat};
\end{axis}
\end{tikzpicture}
	\end{minipage}
	\begin{minipage}[c]{0.49\linewidth}
	\centering
	\begin{tikzpicture}[scale=0.8]
\begin{axis}[
legend entries={rch10, rch10 $\Gamma$ statique, rch10 $\Gamma$ dynamique},
%legend style={at={(0.5,1.03)},anchor=south},legend columns=3
xlabel={Caractérisation de taille k},
ylabel={Temps d'éxécution en seconde},
xmin={15},
xmax={27}]
\addplot +[mark=none] table[x=k,y=temps]{./resultats/sh_s3836.dat};
\addplot +[mark=none] table[x=k,y=temps]{./resultats/gamma_s3836.dat};
\addplot +[mark=none] table[x=k,y=temps]{./resultats/gamma_dynamique_s3836.dat};
\end{axis}
\end{tikzpicture}
	\end{minipage}
\caption{Comparaisons heuristique $\Gamma$ statique et heuristique $\Gamma$ dynamique sur s3836-0}
\label{compGammaS3836}
\end{figure}

Sur la figure \ref{GammaStatiqueRch10}, nous observons que l'heuristique $\Gamma$ statique permet de réduire considérablement le nombre de comparaisons d'entité. Le temps de résolution est également amélioré. Lorsque nous traitons cette instance avec l'heuristique $\Gamma$ dynamique(figure \ref{GammaDynamiqueRch10}), nous apercevons que le nombre de comparaisons d'entité est également plus faible, mais nous payons le caractère dynamique de l'heuristique puisque le temps de résolution est beaucoup plus élevé que la résolution standard. La borne inférieure est par voie de conséquence très élevée ($k=32$ contre $k=25$ en résolution standard). Nous nous intéressons dès lors à comparer les deux heuristiques sur l'instance. nous constatons que $\Gamma$ dynamique est meilleure que $\Gamma$ statique sur le nombre de comparaisons d'entité jusque $k=35$ (figure \ref{compGammaZoom}), mais que passé cette borne, $\Gamma$ statique est bien plus performante(figure \ref{compGamma}) aussi bien en terme de nombre de comparaison d'entité que en terme de temps de résolution. Les observations sur l'instance aléatoire n'étant pas beaucoup plus fructueuse pour l'heuristique $\Gamma$ dynamique (figure \ref{compGammaS3836}), nous concluons que cette dernière n'est pas à retenir pour les résolutions à venir, au contraire de l'heuristique $\Gamma$ statique qui, pour un coût de calcul insignifiant, nous réduit le temps de résolution de manière conséquente.

La figure \ref{compGammaZoom} nous prouve également que aucune des deux heuristiques n'effectue un tri optimal.

\subsubsection{Autres heuristiques }
\paragraph{Heuristique des plus mauvais d'abord (pmda)}
Lorsque nous parcourons une instance pour une caractérisation de taille $k$, nous affectons un poids sur la paire d'entités comparées deux à deux sur les indices de la combinaison courante.



\begin{definition}[Poids $P$ d'une paire d'entités]
Formellement, soit $e_a$, $e_z$ deux entités n'appartenant pas au même groupe, $\mathcal{C}$ une combinaison de $\mathcal{C}_{|\mathcal{X}|}^k$, $P$ le poids de la paire $\{e_a,e_z\}$.
$$ P = |\{i / e_a(i)=e_z(i), \forall i \in \mathcal{C}\}| $$
Si $P>=k-1$ alors la \textbf{paire d'entités $\{e_a,e_z\}$ est critique} car lors du parcours de la prochaine combinaison, cette paire a la plus forte probabilité d'être identique sur les nouveaux indices.
\end{definition}

L'idée est donc de traiter au plus vite ce type d'entités afin d'examiner au plus vite d'autres combinaisons.

\begin{exemple}
Soit les 2 entités suivantes:
\begin{center}
\begin{tabular}{|c|c|c|c|c|c|c|c|}
\hline 
Groupe & \backslashbox{Entités}{Gènes} & g1 & g2 & g3 & g4 & g5 & g6 \\ 
\hline 
1 & e1 & 1 & 0 & 1 & 0 & 0 & 0 \\ 
\hline 
20 & e400 & 1 & 0 & 0 & 0 & 0 & 1 \\ 
\hline 
\end{tabular}
\end{center}
Supposons que nous sommes dans le cas d'une caractérisation de taille 3, nous parcourons les combinaisons de $\mathcal{C}_6^3 $. Supposons que le groupe 1 soit de taille 1, si nous sommes à la comparaison entre e1 et e400, cela signifie que nous avons déjà effectué 399 comparaisons d'entités. 
Observons la résolution de cet exemple:
\begin{itemize}
\item Regardons alors la combinaison courante $\mathcal{C}=\{123\}$ : nous apercevons que seul g3 permet la caractérisation. Comme plusieurs des combinaisons suivantes ne différeront que d'un élément, cet paire d'entités \{e1,e400\} a la plus forte probabilité d'être similiraire lors de la prochaine combinaison, nous gardons donc en mémoire cet ensemble qui a un poids égale à $k-1$.
\item Supposons que $\mathcal{C}=\{123\}$ n'ai pas caractérisé notre instance, nous parcourons alors $\mathcal{C}=\{124\}$ : nous commençons par parcourir les ensembles critiques obtenus lors du parcours précédent, soit la comparaison entre e1 et e400. Le poids est alors égal à $k$, ce qui signifie que nous pouvons arrêter notre recherche sur cette combinaison (car celle ci ne pourra en aucun cas caractériser l'instance). Cependant nous gardons en mémoire cet ensemble critique. Notons que nous avons fait là l'économie de 399 comparaisons d'entités.
\item Nous parcourons alors $\mathcal{C}=\{125\}$ : même constat , de nouveau une économie de 399 comparaisons d'entités. 
\item Nous parcourons alors $\mathcal{C}=\{126\}$ : aucun effet, mais la paire \{e1,e400\} est toujours considérée comme critique.
\item Nous parcourons alors $\mathcal{C}=\{134\}$ : aucun effet, mais la paire  \{e1,e400\} n'est plus considérée comme critique car son poids $<k-1$.
\end{itemize}
\label{exemplePmda}
\end{exemple}

Nous proposons une variante à cette heuristique pour laquelle dans le dernier point de l'exemple \ref{exemplePmda}, nous continuons de considérer la paire d'entité comme critique(i.e.: une paire d'entité qui a été définit comme étant critique ne peut plus changer de statut). Nous appelons cette variante pmdaNoMaj\footnote{Plus mauvais d'abord sans mise à jour}.

\subparagraph{Résultats}
\begin{figure}
\centering
	\begin{minipage}[c]{0.49\linewidth}
	\centering
	\begin{tikzpicture}[scale=0.8]
\begin{axis}

[legend entries={rch10, rch10 pmda, rch10 pmda maj},
xlabel={Caractérisation de taille k},
ylabel={Nombre de comparaisons d'entités},
xmin={25},
xmax={40}]

\addplot +[mark=none] table[x=k,y=nbComp]{./resultats/sh_rch10.dat};
\addplot +[mark=none] table[x=k,y=nbComp]{./resultats/pmda_noMaj_rch10.dat};
\addplot +[mark=none] table[x=k,y=nbComp]{./resultats/pmda_maj_rch10.dat};
\end{axis}
\end{tikzpicture}
	\end{minipage}
	\begin{minipage}[c]{0.49\linewidth}
	\centering
	\begin{tikzpicture}[scale=0.8]
\begin{axis}[
legend entries={rch10, rch10 pmda, rch10 pmdaNoMaj},
xlabel={Caractérisation de taille k},
ylabel={Temps d'éxécution en seconde},
xmin={25},
xmax={40}]

\addplot +[mark=none] table[x=k,y=temps]{./resultats/sh_rch10.dat};
\addplot +[mark=none] table[x=k,y=temps]{./resultats/pmda_maj_rch10.dat};
\addplot +[mark=none] table[x=k,y=temps]{./resultats/pmda_noMaj_rch10.dat};

\end{axis}
\end{tikzpicture}

	\end{minipage}
\caption{Heuristiques pmda sur rch10}
\label{pmdaRch10}
\end{figure}

\begin{figure}
\centering
	\begin{minipage}[c]{0.49\linewidth}
	\centering
	\begin{tikzpicture}[scale=0.8]
\begin{axis}[
legend entries={s3836-0,s3836-0 pmda, s3836-0 pmda maj},
%legend style={at={(0.5,1.03)},anchor=south},legend columns=3
xlabel={Caractérisation de taille k},
ylabel={Nombre de comparaisons d'entités},
xmin={15},
xmax={27},
%title={Résolution sans heuristique de rch10 et s3836-0 sur les comparaisons}
]
\addplot +[mark=none] table[x=k,y=nbComp]{./resultats/sh_s3836.dat};
\addplot +[mark=none] table[x=k,y=nbComp]{./resultats/pmda_noMaj_s3836.dat};
\addplot +[mark=none] table[x=k,y=nbComp]{./resultats/pmda_maj_s3836.dat};

\end{axis}
\end{tikzpicture}
	\end{minipage}
	\begin{minipage}[c]{0.49\linewidth}
	\centering
	\begin{tikzpicture}[scale=0.8]
\begin{axis}[
legend entries={s3836-0,s3836-0 pmda, s3836-0 pmdaNoMaj},
%legend style={at={(0.5,1.03)},anchor=south},legend columns=3
xlabel={Caractérisation de taille k},
ylabel={Temps d'éxécution en seconde},
xmin={15},
xmax={27}]
\addplot +[mark=none] table[x=k,y=temps]{./resultats/sh_s3836.dat};
\addplot +[mark=none] table[x=k,y=temps]{./resultats/pmda_maj_s3836.dat};
\addplot +[mark=none] table[x=k,y=temps]{./resultats/pmda_noMaj_s3836.dat};

\end{axis}
\end{tikzpicture}
	\end{minipage}
\caption{Heuristiques pmda sur s3836-0}
\label{pmdaS3836}
\end{figure}

Nous constatons que les deux heuristiques sont efficaces sur les deux instances(figure \ref{pmdaRch10} et \ref{pmdaS3836}). L'heuristique pmdaNoMaj est légèrement plus efficace que pmda sur l'instance rch10(cela se confirme sur d'autres instances).

\paragraph{Heuristique des valeurs taboues}
\label{heuristiqueTabou}

Nous avons vu dans l'exemple \ref{exempleDifficulté} page \pageref{exempleDifficulté} qu'il est possible d'utiliser les propriétés des masques pour améliorer le temps de résolution d'une instance MIN-PCM. L'heuristique des valeurs taboues permet justement de tirer partie de ces propriétés. Cette heuristique ne pourra fonctionner que sur des instances dont le coefficient $\sigma \simeq 0$.

\begin{exemple}{Heuristique des valeurs taboues\\}
{
	Considérons les masques des deux groupes suivants:\\
	\begin{center}
		\begin{tabular}{|c|c|c|c|c|c|}
		\hline 
		\backslashbox{Groupe}{Gènes} & g0 & g1 & g2 & g3 & g4 \\ 
		\hline 
		Gr1 & 1 & 0.8 & 1 & 0 & 1 \\ 
		\hline 
		Gr2 & 1 & 1 & 0.7 & 0 & 0 \\  
		\hline 
		\end{tabular}
	\end{center}
	
Lorsque nous essayons de caractériser le groupe Gr1 et Gr2 avec une combinaison de gènes, nous sommes obligés de comparer chaque entité avec les gènes présent dans la combinaison dont les valeurs dans les masques de Gr1 et Gr2 ne correspondent pas à des valeurs entières(0 ou 1). Mais dans le cas où ces valeurs correspondent à des valeurs entières, nous pouvons considérer les deux masques comme des entités propres puisque toutes les entités des groupes présenteront la même valeur sur les gènes en question. Nous pouvons ainsi faire l'économie du parcours des combinaisons des différentes entités du groupe Gr1 et Gr2, ces valeurs n'ayant pas besoin d'être observées, nous considérons qu'elles sont taboues. Dans notre exemple, nous devrons parcourir de façon exhaustive les colonnes des gènes g1 et g2 lorsque celles-ci feront partie d'une combinaison à tester. Mais ce ne sera pas le cas pour les colonnes des autres gènes (g0, g3 et g4) pour lesquelles nous avons une comparaisons directe.
}
\end{exemple}



\subparagraph{Résultats}

\begin{figure}
\centering
	\begin{minipage}[c]{0.49\linewidth}
	\centering
	\input{./figure/rch10_sh_vs_tabou_nbComp.tex}
	\end{minipage}
	\begin{minipage}[c]{0.49\linewidth}
	\centering
	\input{./figure/rch10_sh_vs_tabou_temps.tex}
	\end{minipage}
\caption{Heuristique des valeurs taboues sur rch10}
\label{tabourch10}
\end{figure}

\begin{figure}
\centering
	\begin{minipage}[c]{0.49\linewidth}
	\centering
	\input{./figure/s3836_sh_vs_tabou_nbComp.tex}
	\end{minipage}
	\begin{minipage}[c]{0.49\linewidth}
	\centering
	\begin{tikzpicture}[scale=0.8]
\begin{axis}[
legend entries={rch10, rch10 tabou},
%legend style={at={(0.5,1.03)},anchor=south},legend columns=3
xlabel={Caractérisation de taille k},
ylabel={Nombre de comparaisons d'entités},
xmin={15},
xmax={27},
%title={Résolution sans heuristique de rch10 et s3836-0 sur les comparaisons}
]
\addplot +[mark=none] table[x=k,y=temps]{./resultats/sh_s3836.dat};
\addplot +[mark=none] table[x=k,y=temps]{./resultats/tabou_s3836.dat};
\end{axis}
\end{tikzpicture}
	\end{minipage}
\caption{Heuristique des valeurs taboues sur s3836-0}
\label{tabou3836}
\end{figure}

Nous observons sur la figure \ref{tabourch10} que l'heuristique permet de réduire le nombre de comparaisons d'entités ainsi que le temps de résolution sur l'instance réelle dont le coefficient $\sigma = 0.093$. Ce n'est pas le cas pour l'instance aléatoire (figure \ref{tabou3836}) dont le coefficient $\sigma = 1$.

\subsubsection{Comparaisons entre heuristiques}
\label{sectCompar}

Nous comparons ici les meilleures heuristiques retenues dans cette sous section afin d'identifier lesquelles sont les plus influentes lors de la résolution d'une instance MIN-PCM. Nous ajoutons également une heuristique qui combine les heuristiques retenues(all).

\begin{figure}
\centering
	\begin{minipage}[c]{0.49\linewidth}
	\centering
	\input{./figure/rch10_compare_nbComp.tex}
	\end{minipage}
	\begin{minipage}[c]{0.49\linewidth}
	\centering
	\begin{tikzpicture}[scale=0.8]
\begin{axis}[
legend entries={rch10 trie par $\mathcal{T}$,rch10 par $\Gamma$,rch10 pmdaNoMaj,rch10 tabou,rch10 all},
legend style={at={(0.5,1.03)},anchor=south},legend columns=1
xlabel={Caractérisation de taille k},
ylabel={Temps d'éxécution en seconde},
xmin={11},
xmax={35},
%ymax={1000000000}
%title={Résolution sans heuristique de rch10 et s3836-0 sur les comparaisons}
]
\addplot +[mark=none] table[x=k,y=temps]{./resultats/tau_rch10.dat};
\addplot +[mark=none] table[x=k,y=temps]{./resultats/gamma_rch10.dat};
\addplot +[mark=none, color=green] table[x=k,y=temps]{./resultats/pmda_noMaj_rch10.dat};
\addplot +[mark=none] table[x=k,y=temps]{./resultats/tabou_rch10.dat};
\addplot +[mark=none, color=violet] table[x=k,y=temps]{./resultats/all_rch10.dat};
\end{axis}
\end{tikzpicture}
	\end{minipage}
\caption{Comparaisons des heuristiques sur rch10}
\label{compareRch10}
\end{figure}

Nous observons sur la figure \ref{compareRch10} que l'heuristique de tri par taux de similarité $\cal{T}$ (en bleu) est la plus influente des heuristiques. Vient ensuite l'heuristique de tri des groupes par $\Gamma$ (en rouge), celle-ci a une influence quasi similaire avec l'heuristique des plus mauvais d'abord sans mise à jour (pmdaNoMaj, en vert). L'heuristique des valeurs taboues(en noir) est la moins influente. Nous constatons que la combinaison de ces heuristiques (en violet) permet d'obtenir une excellente heuristique.

\begin{figure}
\centering
	\begin{minipage}[c]{0.49\linewidth}
	\centering
	\begin{tikzpicture}[scale=0.8]
\begin{axis}[
legend entries={s3836-0 trie par $\mathcal{T}$,s3836-0 par $\Gamma$,s3836-0 pmdaNoMaj,s3836-0 tabou, s3836-0 all},
legend style={at={(0.5,1.03)},anchor=south},legend columns=1
xlabel={Caractérisation de taille k},
ylabel={Nombre de comparaisons d'entités},
xmin={14},
xmax={26},
%ymax={1000000000}
%title={Résolution sans heuristique de rch10 et s3836-0 sur les comparaisons}
]
\addplot +[mark=none] table[x=k,y=nbComp]{./resultats/tau_s3836.dat};
\addplot +[mark=none] table[x=k,y=nbComp]{./resultats/gamma_s3836.dat};
\addplot +[mark=none, color=green] table[x=k,y=nbComp]{./resultats/pmda_noMaj_s3836.dat};
\addplot +[mark=none] table[x=k,y=nbComp]{./resultats/tabou_s3836.dat};
\addplot +[mark=none, color=violet] table[x=k,y=nbComp]{./resultats/all_s3836.dat};

\end{axis}
\end{tikzpicture}
	\end{minipage}
	\begin{minipage}[c]{0.49\linewidth}
	\centering
	\input{./figure/s3836_compare_temps.tex}
	\end{minipage}
\caption{Comparaisons des heuristiques sur s3836-0}
\label{compares3836}
\end{figure}

Sur la figure \ref{compares3836}, l'heuristique de tri par taux de similarité $\cal{T}$ (en bleu) est la moins influente, cela est dû au coefficient $\Delta\mathcal{T} \simeq 0$. L'heuristique des plus mauvais d'abord sans mise à jour (pmdaNoMaj, en vert) surpasse toutes les heuristiques, y compris l'heuristique combinant toutes les autres (en violet), nous pensons qu'il s'agit là du coût de calcul de l'heuristique des valeurs taboues (en noir) et qui n'a pas beaucoup d'influence du fait que le coefficient de difficulté $\sigma = 1$. De plus, l'heuristique de trie par $\cal{T}$ peut avoir une influence négative lors de la résolution d'instance aléatoire. L'heuristique de tri des groupes par $\Gamma$ (en rouge) apporte une amélioration significative.

Nous concluons que les heuristiques choisies pour la résolution d'une instance MIN-PCM doivent l'être en fonction des différents coefficients ($\Delta\mathcal{T}$, $\sigma$, $\rho$) de l'instance.

\subsubsection{Résultats}
\label{sectionCompare}
Nous présentons ici les résultats obtenus avec notre solveur \emph{Exact-Proj-Car2} (EPC2) qui fonctionne avec toutes les meilleures heuristiques présentées dans cette sous section. Nous pouvons ainsi comparer nos résultats avec ceux obtenus par \cite{Chhel2013}.
La colonne PL présente les résultats obtenus par reformulation en programmation linéaire sur le solveur IBM \textit{cplex}\footnote{\url{http://www.ibm.com/software/integration/optimization/cplex-optimizer}}. Les instances marquées par "-" n'ont pas pu être chargées en mémoire car elles sollicitaient plus de 32 Go de RAM. La colonne EPC présente les résultats du solveur \emph{Exact-Proj-Car} fournis dans \cite{Chhel2013}. Les expérimentations sont faites sur une machine composée d'un processeur Intel Core\up{tm} i7-2620M CPU à 2.70GHz (deux cœurs) avec 4 Go Ram tournant sous Linux 64-bits. L'option de compilation -Ofast est activée pour obtenir notre exécutable. Le temps autorisé pour chaque exécution est de 10 minutes. Les résultats en gras indiquent que les solutions sont optimales.

La colonne \textit{temps} indique le temps cumulé depuis le début de la recherche à $k=|\cal{X}|$.

\begin{center}
\begin{tabular}{|c|c|c|c|c|c|c|c|c|c|c|c|}
\hline 
\multirow{2}*{Instances} & \multirow{2}*{Entités} & \multirow{2}*{Gènes}& \multirow{2}*{$\Delta\cal{T}$}&  \multirow{2}*{$\rho$} & \multirow{2}*{$\sigma$} & \multirow{2}*{PL} & \multirow{2}*{EPC} & \multicolumn{2}{c|}{EPC2} \\
\cline{9-10} 
 & & & & & & & & k & temps \\
\hline 
s301-0 & 500 & 400 & 0.025 & 0.034 & 0.999 & - & 13 & 13 & 2.072\\ 
\hline 
s326-0 & 500 & 500 & 0.026 & 0.033 & 1 & - & 13 & 13 & 1.707 \\ 
\hline 
s413-30 & 500 & 600 & 0.027 & 0.035 & 1 & - & 13 & 13 & 1.050\\ 
\hline 
s555-20 & 800 & 800 & 0.029 & 0.039 & 0.999 & - & 13 & 13 & 14.924\\ 
\hline 
s625-20 & 500 & 1000 & 0.027 & 0.035 & 1 & - & 13 & 13 & 1.636 \\ 
\hline 
s754-10 & 600 & 200 & 0.027 & 0.034 & 1 & - & 13 & \textcolor{red}{14} & 0.366\\ 
\hline 
s882-20 & 600 & 400 & 0.024 & 0.032 & 1 & - & 13 & 13 & 543.176 \\ 
\hline 
s2501-70 & 800 & 800 & 0.024 & 0.033 & 1 & - & 15 & \textcolor{blue}{14} & \textcolor{cyan}{282.935} \\ 
\hline 
s31294-50 & 200 & 1000 & 0.049 & 0.065 & 0.993 & 10 & 10 & 10 & 381.108 \\ 
\hline 
s3836-0 & 1000 & 1000 & 0.019 & 0.024 & 1 & - & 16 & \textcolor{blue}{15} & \textcolor{cyan}{13.072} \\ 
\hline 
rch8 & 56 & 27 & 0.339 & 0.569 & 0.067 & \textbf{9} & \textbf{9} & \textbf{9} & 0.038 \\ 
\hline 
raphv & 108 & 68 & 0.301 & 0.588 & 0.419 & \textbf{6} & \textbf{6} & \textbf{6} & 2.055 \\ 
\hline 
raphy & 112 & 70 & 0.293 & 0.609 & 0.668 & \textbf{6} & \textbf{6} & \textbf{6} & 6.846 \\ 
\hline 
rarep & 112 & 72 & 0.295 & 0.651 & 0.502 & \textbf{12} & 39 & \textcolor{blue}{16} & 206.037 \\ 
\hline 
rch10 & 112 & 86 & 0.237 & 0.626 & 0.094 & \textbf{10} & 25 & \textcolor{blue}{11} & 383.006\\ 
\hline 
\end{tabular} 
\end{center}

Nous avons placé en bleu les caractérisations de taille inférieure à EPC. Les temps de couleur cyan permettent de souligner la rapidité d'éxécution de EPC2 par rapport à EPC:  EPC2 caractérise l'instance s3836-0 en 13 secondes pour un k=15 alors que EPC ne réduit pas en dessous de k=16 en 10 minutes. Nous observons le même type de phénomène sur l'instance s2501-70. Notons toutefois que l'instance s754-10 est moins bien caractérisée par EPC2, nous pensons que dans ce cas précis, l'influence du tri par $\cal{T}$ est négative, cela est du à $\Delta\cal{T}$ proche de 0. Notre solveur réduit de façon significative les caractérisations des instances réelles rarep et rch10. 


\subsection{Recherche incomplète}

Nous nous intéressons ici à obtenir une méthode de recherche incomplète qui se sert des propriétés révélées dans la sous section \ref{subsectionInstanceDifficile}. Nous mettons un mécanisme de roulette proportionnelle (non adaptative) sur les gènes en fonction de leur taux de similarité $\mathcal{T}$: lors d'un tirage, chaque gène a une probabilité inversement proportionelle à son taux de similarité d'être sélectionné.

\begin{definition}{}
\label{defProbaGene}
La probabilité $P(i)$ du gène i d'être sélectionné lors d'un tirage se calcule de la façon suivante:\\
Soit $i \in [1,\ldots,|\mathcal{X}|]$ , les gènes $i$ sont ordonnées par ordre croissant de similarité $\mathcal{T}$,
$$ P(i)=\frac{1-\mathcal{T}(i)}{\sum_{i=1}^{|\mathcal{X}|} (1-\mathcal{T}(i))} $$
$$ \sum_{i=1}^{|\mathcal{X}|} P(i) = 1$$
\end{definition}

\begin{algorithm}
	\textbf{Réel $proba[]$ initialisation\_proba (Réel $taux[]$)}\\
	\tcp{$taux[i]$ est le taux de similarité $\mathcal{T}$ du gène $i$, $taux$ est ordonné par ordre croissant de $\mathcal{T}$ }				
	Réel $sommeTauxInverse \leftarrow 0$\\
	Réel $proba[]$\\
	
	
	\PourTous {$i \in [1,\ldots,|\mathcal{X}|$ \tcp{Boucle 1}} 
	{
		$sommeTauxInverse \leftarrow sommeTauxInverse + taux[i]$
	}	
	
	\PourTous {$i \in [1,\ldots,|\mathcal{X}|$ \tcp{Boucle 2}} 
	{
		$proba[i] \leftarrow \frac{1-taux[i]}{sommeTauxInverse}$
	}
	
	\PourTous {$i \in [2,\ldots,|\mathcal{X}|$ \tcp{Boucle 3}} 
	{
		$proba[i] \leftarrow proba[i]+proba[i-1]$
	}
	\caption{Algorithme d'initialisation des probabilités de sélection des gènes d'une instance}
	\label{algoInitialiseProba}
\end{algorithm}

\begin{algorithm}
	\textbf{roulette ($taux$, $Groupes$, $k$, $n$)}\\
	\tcp
	{
		$taux[i]$ est le taux de similarité $\mathcal{T}$ du gène $i$, $taux$ est ordonné par ordre croissant de $\mathcal{T}$\\
		$Groupes$ est un ensemble contenant tous les groupes de l'instance\\
		$k$ correspond au nombre de gène souhaité pour la caractérisation\\
		$n$ correspond aux nombres de gènes de l'instance
	}
	Réel $proba[] \leftarrow$ initialisation\_proba($taux$)\\	
	\Tq{$VRAI$ \tcp{Boucle 1}}
	{
		Entier $combinaison\{\}$ \tcp{un ensemble d'entier qui correspondra aux indices des gènes dont on souhaite la caractérisation}		
		\Tq{$|combinaison| < k $ \tcp{Boucle 2}}
		{
			Réel $alea \leftarrow $ nombre aléatoire entre 0 et 1 \\
			\PourTous{$i \in [1,\ldots,n]$ \tcp{Boucle 3}}
			{
				\Si{$proba[i] \geq alea$}
				{
					$combinaison \leftarrow combinaison \cup i$\\
					Sortir de la boucle 3
				}
			}
		}
		\Si{caractérise\_instance($I$,$combinaison$)}
		{
			afficher("La combinaison ",$combinaison$," permet de caractériser l'instance.")\\
			$k \leftarrow k-1$\\
			$combinaison \leftarrow \{\}$		
		}
	}
	\caption{Algorithme de recherche approchée par roulette proportionelle}
	\label{algoRoulette}
\end{algorithm}
		
En pratique, nous utilisons l'algorithme de la figure \ref{algoInitialiseProba}: nous instancions un tableau  de réel $proba$ qui contient les probabilités de chaque gène, ainsi, $proba[i]$ contient la probabilité du gène $i$ tel qui a été calculé dans la définition \ref{defProbaGene}(Boucle 1 et 2). La somme des éléments du tableau $proba = 1$. Nous transformons ce tableau afin de répartir ces probabilités sur une échelle de 0 à 1 dans la boucle 3. Nous pourrons ainsi faire un tirage par roulette proportionnelle tel qui est définit par l'algorithme de la figure \ref{algoRoulette}.

L'algorithme roulette décrit notre recherche approchée, nous initialisons le tableau $proba$ des probabilités de sélection des gènes. Nous créons une combinaison $combinaison$ en fonction de ces probabilités dans la boucle 2 . Si la combinaison caractérise l'instance, nous décrémentons la taille $k$ de la caractérisation souhaitée et nous réinitialisons l'ensemble $combinaison$ à l'ensemble vide. Nous réitérons ce processus dans une boucle infinie, nous demandons au programme de s'arrêter à l'issu d'un temps déterminé par l'utilisateur. Notons que la méthode caractérise\_instance nous permet de savoir si la combinaison $combinaison$ caractérise l'instance et qu'elle bénéficie des heuristiques définit dans la sous-section \ref{rechercheExacte}.

\begin{remarque}[Améliorations de la méthode]
Cette méthode de recherche peut être améliorée: lorsque nous caractérisons une instance avec $k$ gènes, nous pourrions tester les combinaisons possibles de $C_k^{k-1}$ dans la combinaison courante jusqu'à caractériser de nouveau à $k-1$ si cela est possible, et dans le cas contraire uniquement, relancer un tirage. Par manque de temps, nous ne testerons pas cette variante, l'objectif de notre démarche étant de démontrer la pertinence d'utilisation du coefficient $\mathcal{T}$ pour une recherche approchée.
\end{remarque}

\subsubsection{Résultats}
Nous présentons ici les résultats de notre méthode. Nous faisons la comparaison avec l'heuristique du programme LSPC (Local Search Proj Car) proposée dans \cite{Chhel2013}. La colonne itération correspond aux nombres d'itérations de la boucle infinie (boucle 1) présentée dans l'algorithme de la figure \ref{algoRoulette}.
La colonne PL présente les résultats obtenus par reformulation en programmation linéaire sur le solveur IBM \textit{cplex}\footnote{\url{http://www.ibm.com/software/integration/optimization/cplex-optimizer}}. Les instances marquées par "-" n'ont pas pu être chargées en mémoire car elles sollicitaient plus de 32 Go de RAM. Les expérimentations sont faites sur une machine composée d'un processeur Intel Core\up{tm} i7-2620M CPU à 2.70GHz (deux cœurs) avec 4 Go Ram tournant sous Linux 64-bits. L'option de compilation -Ofast est activée pour obtenir notre exécutable. Le temps autorisé pour chaque exécution est de 10 minutes. Les résultats en gras indiquent que les solutions sont optimales.

\begin{center}
\begin{tabular}{|c|c|c|c|c|c|c|c|c|c|c|}
\hline 
\multirow{2}*{Instances} & \multirow{2}*{Entités} & \multirow{2}*{Gènes}&\multirow{2}*{$\Delta\cal{T}$} & \multirow{2}*{$\rho$} & \multirow{2}*{$\sigma$} & \multirow{2}*{PL} & \multirow{2}*{LSPC} & \multicolumn{3}{c|}{Roulette proportionelle} \\
\cline{9-11} 
 & & & & & & & & k & temps & itérations \\
\hline 
s301-0 & 500  & 400 & 0.025 & 0.034 & 0.999 & - & 14 & \textcolor{blue}{13} & 421.005 & 1612515 \\ 
\hline 
s326-0 & 500 & 500 & 0.026 & 0.033 & 1 & - & 14 & \textcolor{blue}{13} & 93.247 & 390166\\ 
\hline 
s413-30 & 500 & 600 & 0.027 & 0.035 & 1 & - & 13 & 13 & 428.634 & 1388939 \\ 
\hline 
s555-20 & 800 & 800 & 0.029 & 0.039 & 0.999 & - & 13 & 13 & 395.024 & 1286275 \\ 
\hline 
s625-20 & 500 & 1000 & 0.027 & 0.035 & 1 & - & 13 & 13 & 415.878 & 1359740 \\ 
\hline 
s754-10 & 600 & 200 & 0.027 & 0.034 & 1 & - & 14 & 14 & 25.876 & 84672\\ 
\hline 
s882-20 & 600 & 400 & 0.024 & 0.032 & 1 & - & 14 & 14 & 3.023 & 9113\\ 
\hline 
s2501-70 & 800 & 800 & 0.024 & 0.033 & 1 & - & 15 & 15 & 4.24 & 4350 \\ 
\hline 
s31294-50 & 200 & 1000 & 0.049 & 0.065 & 0.993 & 10 & 11 & 11 & 0.963 & 5273\\ 
\hline 
s3836-0 & 1000 & 1000 & 0.019 & 0.024 & 1 & - & 16 & 16 & 5.266 & 1006 \\ 
\hline
rch8 & 56 & 27 & 0.339 & 0.569 & 0.067 & \textbf{9} & 9 & 9 & 0.031 & 22440 \\ 
\hline 
raphv & 108 & 68 & 0.302 & 0.588 & 0.419 & \textbf{6} & 9 & \textcolor{blue}{6} & 0.657 & 470528\\ 
\hline 
raphy & 112 & 70 & 0.294 & 0.609 & 0.667 & \textbf{6} & 8 & \textcolor{blue}{6} & 0.873 & 524344\\ 
\hline 
rarep & 112 & 72 & 0.295 & 0.651 & 0.502 & \textbf{12} & 14 & 14 & 36.627 & 17793513
\\ 
\hline 
rch10 & 112 & 86 & 0.238 & 0.626 & 0.094 & \textbf{10} & 15 & \textcolor{blue}{12} & 65.615 & 35676187\\ 
\hline 
\end{tabular} 
\end{center}

Nous remarquons que notre méthode, bien que grandement améliorable, est sensiblement plus efficace que LSPC sur les instances aléatoires et est beaucoup plus efficace sur les instances réelles: nous trouvons la borne optimale sur les instances rch8, raphv et raphy , et nous l'approchons pour les instances rarep et rch10. Ces résultats démontrent la pertinence de l'utilisation des taux de similarité $\mathcal{T}$ pour les instances réelles.

%\newpage
%\section{Conclusions} 

\subsection{Difficultés rencontrées}

\subsection{Conclusions générales}

\subsection{Évolutions possibles et perspective de recherche}
\subsubsection{A court terme}
\subsubsection{A long terme}


\listoffigures

\listoftables

\subsection*{\begin{center} Résumé \end{center}}
\addcontentsline{toc}{chapter}{Résumé / Abstract}

\par Ce rapport porte sur la ....
\\\\
\textbf{Mots-clés:} azerty

\subsection*{\huge \begin{center} Abstract \end{center}}

\par This report concerns the ....
\\\\
\textbf{Keywords:} azerty 



\bibliography{src/bib}
\bibliographystyle{alpha}

\end{document}
