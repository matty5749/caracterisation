\section{Contributions}

\subsection{Introduction} 
Cette section présente les démarches de recherche qui ont été effectué durant le stage.\\
Dans un premier temps, nous faisons une proposition qui permet d'identifier si une instance est difficile ou non.\\
Ensuite, nous abordons la résolution du problème MIN-PCM avec deux approches différentes :
\begin{itemize}
\item Une recherche exacte qui à la possibilité de prouver la borne minimum du MIN-PCM sur des instances de taille raisonnable.
\item Une recherche approché qui à la possibilité de trouver des solutions de bonne qualité mais non nécessairement optimal en un temps polynômial sur des instance de grande taille.
\end{itemize} 
\textit{TODO: Enfin, nous générons des instances pseudo-aléatoire de différents degrés de difficultés que nous soumettons à nos algorithmes.}

\subsection{Définition d'une instance difficile}
Prenons deux instances: une réelle (rch10) et une aléatoire (s3836-0), voici leurs caractéristiques:
\begin{center}
\begin{tabular}{|c|c|c|c|c|c|}
\hline 
Instances & Entités & Groupes & Gènes & Résolution PL & Résolution EPC\footnote{[Exact-Proj-Car CHHEL et al]} \\ 
\hline 
s3836-0 & 1000 & 15 & 1000 & - & 16 \\ 
\hline
rch10 & 173 & 27 & 98 & \textbf{10}\footnote{En gras = solution optimal} & 14 \\ 
\hline
\end{tabular} 
\end{center}
\vspace{7mm}

A priori, on peut supposer que l'instance aléatoire est plus difficile à résoudre: elle est bien plus volumineuse que l'instance réelle à tel point qu'elle nécessite plus de 32 Go de RAM pour une résolution en programmation linéaire.

Observons leurs résolutions avec notre algorithme sans heuristique:
\begin{figure}[H]
\centering
	\begin{minipage}[c]{0.49\linewidth}
	\centering
	\begin{tikzpicture}[scale=0.8]
\begin{axis}[
legend entries={rch10,s3836-0},
%legend style={at={(0.5,1.03)},anchor=south},legend columns=3
xlabel={Caractérisation de taille k},
ylabel={Nombre de comparaisons d'entités},
xmin={14},
xmax={40}
]
\addplot +[mark=none] table[x=k,y=nbComp]{./resultats/sh_rch10.dat};
\addplot +[mark=none] table[x=k,y=nbComp]{./resultats/sh_s3836.dat};
\end{axis}
\end{tikzpicture}


	\end{minipage}
	\begin{minipage}[c]{0.49\linewidth}
	\centering
	\begin{tikzpicture}[scale=0.8]
\begin{axis}[
legend entries={rch10,s3836-0},
%legend style={at={(0.5,1.03)},anchor=south},legend columns=3
xlabel={Caractérisation de taille k},
ylabel={Temps d'éxécution en seconde},
xmin={14},
xmax={40}]
\addplot +[mark=none] table[x=k,y=temps]{./resultats/sh_rch10.dat};
\addplot +[mark=none] table[x=k,y=temps]{./resultats/sh_s3836.dat};
\end{axis}
\end{tikzpicture}


	\end{minipage}
\caption{Résolution sans heuristique de rch10 et s3836-0}
\end{figure} 

Nous apercevons que l'instance aléatoire est facilement résolu jusqu'à une caractérisation de taille 15. Ce n'est pas le cas de l'instance réelle qui ne peut plus caractériser en un temps raisonnable à partir d'une caractérisation de taille 25. Ce type d'observation étant \textbf{systématique} quelque soit les caractéristiques des instances réelles ou aléatoire comparés, nous pouvons alors affirmer que la taille d'une instance ne suffit pas à elle seule pour définir sa difficulté. Dès lors, nous nous posons les deux questions suivantes:\\

\begin{itemize}
\item \textbf{Qu'est ce qui peut bien être à l'origine de cette différence de résolution entre une instance aléatoire et une instance réelle?}
\item \textbf{Existe il une méthode permettant de définir si une instance est difficile à résoudre ou non ?}\\
\end{itemize}
Afin de répondre à ces questions, nous définissons les notions suivantes:

\begin{definition}
Le \textbf{masque $M$ d'un groupe $g$} correspond à la moyenne des présences/absences des gènes pour chaque entité du groupe.\\
Formellement, soit $M_g$ le masque d'un groupe $g$, $g \in \mathcal{G}$, $M_g[i]$ la valeur du masque $g$ en position $i$, $i \in [1,|\mathcal{X}|]$,
% $|\mathcal{X}|$ étant le nombre de gènes de l'instance après la suppression des redondances,
$$\forall i \in  [1, |\mathcal{X}|], M_g[i]= \frac{\sum_{i=1}^{|\mathcal{G}|}e_i}{|\mathcal{G}|} $$
\end{definition}

\begin{definition}
Le \textbf{ratio $r$ d'un masque $M$} correspond au pourcentage de valeur entière (0/1) présentent dans le masque.\\
Formellement, soit $M_g$ le masque d'un groupe $g$, $r_g(I)$ le ratio du groupe $g$ dans l'image $I$, $g \in \mathcal{G}$,
\begin{center}
$$ r_g(I)=\frac{|{i / M_g[i] \in \{0,1\}}|}{|\mathcal{X}|},\forall i \in [1,|\mathcal{X}|]$$
\end{center}
\end{definition}

\subsubsection*{Exemple :}
\begin{center}
\begin{tabular}{|c|c|c|c|c|c|c|c|c|c|c|}
\hline 
\backslashbox{Entités}{Gènes} & g0 & g1 & g2 & g3 & g4 & g5 & g6 & g7 & g8 & g9 \\ 
\hline 
e1 & 1 & 1 & 0 & 1 & 1 & 1 & 0 & 0 & 0 & 1 \\ 
\hline 
e2 & 1 & 1 & 0 & 1 & 1 & 1 & 0 & 1 & 0 & 1 \\ 
\hline 
e3 & 1 & 1 & 0 & 0 & 0 & 1 & 0 & 0 & 0 & 0 \\ 
\hline 
e4 & 1 & 1 & 0 & 1 & 0 & 1 & 0 & 0 & 0 & 0 \\ 
\hline 
e5 & 1 & 1 & 0 & 1 & 1 & 1 & 0 & 1 & 0 & 0 \\ 
\hline 
\hline
Masque & 1 & 1 & 0 & 0.8 & 0.6 & 1 & 0 & 0.4 & 0 & 0.4 \\
\hline
\end{tabular}
\end{center}
Le ratio $r$ de ce groupe est : \\
$r=6/10$\\
soit  $r=0.6$

\begin{definition}
L'\textbf{image $I$ d'une instance} est une matrice en deux dimensions de taille $|\mathcal{G}|*|\mathcal{X}|$ où chaque ligne correspond au masque de chacun des groupes de l'instance.\\
Formellement, soit $I_g$ la ligne g de la matrice $I$ correspondant à l'image de l'instance $\mathcal{I}$, $M_g$ le masque du groupe $g$ de l'instance $\mathcal{I}$,
$$\forall g \in [1,|\mathcal{G}|], I_g=M_g$$
\end{definition}


\begin{definition}
Le \textbf{taux de similarité globale} $\mathcal{T}_j$ d'un gène $j$ correspond à la moyenne des valeurs de la colonne $j$ sur l'image $I$ d'une instance $\mathcal{I}$.\\
Formellement, soit $I$ l'image d'une instance $\mathcal{I}$, $i$ la $i$\up{ème} ligne de $I$, $j$ la $j$\up{ème} colonne de $I$, $I_{ij}$ est la valeur dans $I$ en ligne $i$ et en colonne $j$, $\mathcal{T}_j(I)$ le taux de similarité globale du gène $j$ dans l'image $I$,
$$ \text{Soit } X=\frac{\sum_{i=1}^{|\mathcal{G}|} I_{ij}}{\mathcal{G}} $$ 
$$\text{Si } X<0.5 \text{ alors } \mathcal{T}_j(I)=(0.5-X)*2 $$
$$\text{sinon }\mathcal{T}_j(I)=(0.5-(1-X))*2$$ 
\end{definition}
Ainsi formuler, $\mathcal{T}_j(I) \in [0,1]$, et, plus le taux de similarité globale d'un gène est élevé, plus sa présence(resp. abscence) dans l'instance est redondante.

\begin{definition}
Le \textbf{coefficient de difficulté $\rho$ d'une instance}, correspond à la moyenne des taux de similarité globaux d'une instance.\\
Formellement, soit $I$ l'image d'une instance $\mathcal{I}$, $j$ la $j$\up{ème} colonne de $I$, $\mathcal{T}_j(I)$ le taux de similarité globale du gène $j$ dans l'image $I$,
$$ \rho=\frac{\sum_{j=1}^{|\mathcal{X}|}\tau_j(I)}{|\mathcal{X}|} $$
\end{definition}

\begin{definition}
Le \textbf{coefficient de difficulté $\sigma$ d'une instance}, correspond au complémentaire de la moyenne des ratios des masques d'une instance. Nous utilisons le complémentaire de façon à ce que l'échelle du coefficient $\sigma$ soit calqué sur celui de $\rho$.
Formellement, soit $I$ l'image d'une instance $\mathcal{I}$, $i$ la $i$\up{ème} ligne de $I$, $r_i(I)$ le ratio du groupe $i$ dans l'image $I$,
$$ \sigma=1-\frac{\sum_{i=1}^{|\mathcal{G}|}r_i(I)} {|\mathcal{G}|} $$
\end{definition}

\begin{definition}
Le coefficient $\Delta\cal{T}$ d'une instance correspond à l'écart type des taux de similarité $\cal{T}$ des gènes.
Formellement, soit $I$ l'image d'une instance $\mathcal{I}$, $i$ la $i$\up{ème} ligne de $I$, $r_i(I)$ le ratio du groupe $i$ dans l'image $I$,
$$\Delta\mathcal{T}=\sqrt{\frac{\sum_{j=1}^{|\mathcal{X}|} (moyenne-\mathcal{T}_j)^2}{|\mathcal{X}|}}$$
\end{definition}

Reprenons nos deux instances rch10 et s3836-0 et calculons leurs coefficients de difficultés:

\begin{center}
\begin{tabular}{|c|c|c|}
\hline 
Instances & $\rho$ & $\sigma$ \\ 
\hline 
s3836-0 & 0 & 0.024322 \\ 
\hline
rch10 & 0.626381 & 0.906115 \\ 
\hline
\end{tabular} 
\end{center}

Nous observons que le coefficient de difficulté $\rho$ semble plus significatif que $\sigma$ pour déterminer la difficulté d'une instance, mais nous ne sommes pas en mesure d'indiquer dans quel proportions. Cependant les travaux de [CHHEL et al, 2013] nous indiquent que seul une heuristique sur le choix des variables est en mesure de pouvoir améliorer un algorithme de recherche exacte. Cela nous conforte dans l'idée que $\rho$ a plus d'influence que $\sigma$ sur la difficulté d'une instance. 

\begin{proposition}
Une instance dont le coefficient de difficulté $\rho$ est proche de 1 est une \textbf{instance difficile} à résoudre.
\end{proposition}

\begin{proposition}
Une instance dont le coefficient de difficulté $\rho$ est proche de 1 et dont le coefficient de difficulté $\sigma$ est proche de 1 est une \textbf{instance très difficile} à résoudre.
\end{proposition}

\begin{proposition}
La taille d'une instance (caractérisé par son nombre de gènes et d'entités mais pas son nombre de groupe) est une information sur la difficulté de sa résolution. On peut passer outre cette difficulté uniquement si l'instance à un coeficient $sigma$ proche de 1, dans le cas contraire, cette difficulté ne peut pas être réduite de façon significative par un algorithme.
\end{proposition}

TODO:Proposition$\Delta\cal{T}$

%Les observations sur notre jeu de 15 instances nous permettent de conclure que $\sigma$ en particulier, et $\rho$ dans une moindre mesure, nous permettent de définir ce qu'est une instance difficile. 

\subsubsection{Observations}

Nous présentons ici les instances avec leurs coefficients de difficultés respectifs:
\begin{center}
\begin{tabular}{|c|c|c|c|c|c|c|c|c|c|}
\hline 
Instances & Entités(SR) & Groupes & Gènes(SR)& $\Delta\cal{T}$ & $\rho$ & $\sigma$ & PL & EPC & LSPC \\ 
\hline 
s301-0 & 500 & 30 & 400 & 0.0247583 & 0.034 & 0.999917 & - & 13 & 14 \\ 
\hline 
s326-0 & 500 & 10 & 500 & 0.0264423 & 0.033 & 1 & - & 13 & 14 \\ 
\hline 
s413-30 & 500 & 20 & 600 & 0.027368 & 0.035 & 1 & - & 13 & 13 \\ 
\hline 
s555-20 & 800 & 20 & 800 & 0.028946 & 0.039 & 0.999833 & - & 13 & 13 \\ 
\hline 
s625-20 & 500 & 5 & 1000 & 0.0270316 & 0.035 & 1 & - & 13 & 13 \\ 
\hline 
s754-10 & 600 & 10 & 200 & 0.0266147 & 0.034 & 1 & - & 13 & 14 \\ 
\hline 
s882-20 & 600 & 10 & 400 & 0.0240376 & 0.032 & 1 & - & 13 & 14 \\ 
\hline 
s2501-70 & 800 & 10 & 800 & 0.0243325 & 0.033 & 1 & - & 15 & 15 \\ 
\hline 
s31294-50 & 200 & 15 & 1000 & 0.0493203 & 0.065 & 0.9928 & 10 & 10 & 11 \\ 
\hline 
s3836-0 & 1000 & 15 & 1000 & 0.0187905 & 0.024 & 1 & - & 16 & 16 \\ 
\hline 
raphv & 109 (108) & 8 & 155 (68) & 0.301636 & 0.588 & 0.419118 & \textbf{6} & \textbf{6} & 9 \\ 
\hline 
raphy & 113 (112) & 4 & 155 (70) & 0.293601 & 0.609 & 0.667857 & \textbf{6} & \textbf{6} & 8 \\ 
\hline 
rarep & 112 & 7 & 155 (72) & 0.295408 & 0.651 & 0.501984 & \textbf{12} & 39 & 14 \\ 
\hline 
rch8 & 132 (56) & 21 & 37 (27) & 0.339488 & 0.569 & 0.0670193 & \textbf{9} & \textbf{9} & 9 \\ 
\hline 
rch10 & 173 (112) & 27 & 98 (86) & 0.237992 & 0.626 & 0.0938845 & \textbf{10} & 25 & 15 \\ 
\hline 
\end{tabular} 
\end{center}

Toute les instances aléatoires ont un coefficient $\rho$ proche de 0. Mais elles ont pour difficulté leurs coefficients $\sigma$ qui sont proche de 0 également, cela signifie que nous ne pourrons pas réduire de façon significative leurs temps de résolution. L'instance raphv et raphy ont un ordre difficulté similaire, rch8 a un coefficient $\sigma$ proche de 1 ainsi qu'un coefficient $\rho$ satisfaisant, de plus, elle est de faible taille, c'est l'instance qui semble la plus facile à résoudre. L'instance rch10 à un très bon coefficient $\sigma$, elle doit être plus facile à résoudre que l'instance rarep. Ces deux dernières semble être les instances les plus difficiles à résoudre.Cette série d'observations et de raisonnement est corroboré par les résultats obtenus par [CHHEL et al,2013].

\subsection{Recherche exacte}
\subsubsection{Introduction}
Nous présentons les heuristiques ayant été mise en place pour la résolution d'instance MIN-PCM. Chaque nouvelle heuristique est ajouté à(aux) précédentes. Les comparaisons se font sur l'instance rch10, entre l'ancienne heuristique la plus puissante connu et la nouvelle mise en place. 

Une comparaison entre la meilleure combinaison d'heuristique trouvé et l'heuristique "CCD" proposé par [CHHEL et al,2013] dans \textit{Exact-Proj-Car} est présenté à la fin de la cette section.

\subsubsection{Heuristique de trie sur les gènes }
\paragraph{Trie des gènes par taux de similarité $\mathcal{T}$}
Les outils mis en place pour définir ce qu'est une instance difficile vont maintenant nous permettre d'obtenir de nouvelles heuristique. Dans cette section, nous ordonnons les gènes par ordre croissant sur leur taux de similarité globale $\mathcal{T}$. Dès lors, nous parcourons en priorité les gènes présentant un faible taux de similarité. Ce trie par $\mathcal{T}$ permet de générer un arbre de recherche qui a pour particularité de présenter en priorité les branches ayant le plus de chances de fournir une éventuelle solution. Un tel arbre de recherche nous permet de caractériser bien plus vite une instance. Le coût de calcul de ce trie est insignifiant et est effectué de façon unique avant le lancement de l'algorithme de résolution.

\subparagraph{Résultats}

%\begin{figure}[H]
%\centering
%	\begin{minipage}[c]{0.49\linewidth}
%	\centering
%	\begin{tikzpicture}[scale=0.8]
\begin{axis}[
legend entries={rch10,rch10 $\mathcal{T}$},
%legend style={at={(0.5,1.03)},anchor=south},legend columns=3
xlabel={Caractérisation de taille k},
ylabel={Nombre de comparaisons d'entités},
xmin={10},
xmax={40},
]
\addplot +[mark=none] table[x=k,y=nbComp]{./resultats/sh_rch10.dat};
\addplot +[mark=none] table[x=k,y=nbComp]{./resultats/tau_rch10.dat};
\end{axis}
\end{tikzpicture} 

%	\end{minipage}
%	\begin{minipage}[c]{0.49\linewidth}
%	\centering
%	\begin{tikzpicture}[scale=0.8]
\begin{axis}[
legend entries={rch10,rch10 $\mathcal{T}$},
%legend style={at={(0.5,1.03)},anchor=south},legend columns=3
xlabel={Caractérisation de taille k},
ylabel={Temps d'éxécution en seconde},
xmin={10},
xmax={40}]
\addplot +[mark=none] table[x=k,y=temps]{./resultats/sh_rch10.dat};
\addplot +[mark=none] table[x=k,y=temps]{./resultats/tau_rch10.dat};
\end{axis}
\end{tikzpicture}
%	\end{minipage}
%\caption{Heuristique $\mathcal{T}$ sur rch10}
%\end{figure}
%
%\begin{figure}[H]
%\centering
%	\begin{minipage}[c]{0.49\linewidth}
%	\centering
%	\begin{tikzpicture}[scale=0.8]
\begin{axis}[
legend entries={s3836-0,s3836-0 $\mathcal{T}$},
%legend style={at={(0.5,1.03)},anchor=south},legend columns=3
xlabel={Caractérisation de taille k},
ylabel={Nombre de comparaisons d'entités},
xmin={15},
xmax={40},
%title={Résolution sans heuristique de rch10 et s3836-0 sur les comparaisons}
]
\addplot +[mark=none] table[x=k,y=nbComp]{./resultats/sh_s3836.dat};
\addplot +[mark=none] table[x=k,y=nbComp]{./resultats/tau_s3836.dat};
\end{axis}
\end{tikzpicture}
%	\end{minipage}
%	\begin{minipage}[c]{0.49\linewidth}
%	\centering
%	\begin{tikzpicture}[scale=0.8]
\begin{axis}[
legend entries={s3836-0,s3836-0 $\mathcal{T}$},
%legend style={at={(0.5,1.03)},anchor=south},legend columns=3
xlabel={Caractérisation de taille k},
ylabel={Temps d'éxécution en seconde},
xmin={15},
xmax={40}]

\addplot +[mark=none] table[x=k,y=temps]{./resultats/sh_s3836.dat};
\addplot +[mark=none] table[x=k,y=temps]{./resultats/tau_s3836.dat};
\end{axis}
\end{tikzpicture}
%	\end{minipage}
%\caption{Heuristique $\mathcal{T}$ sur s3836-0}
%\end{figure}

Sur l'instance réelle rch10, on constante que le nombre de comparaisons d'entités  et le temps d'éxécution est considérablement diminué par l'heuristique. Cependant, on s'aperçoit que cette heuristique a peu d'influence (voir même une influence négative dans le cas présent) sur la résolution d'instance aléatoire car celles-ci ne disposent pas de variations significative entre les $\mathcal{T}$ des gènes. La possibilité que l'influence soit négative nous permet de conclure que le trie des gènes par $\mathcal{T}$ n'est pas un trie optimal. 


\subsubsection{Heuristique de trie sur les groupes}
\paragraph{Trie des groupes par taux de similarité $\Gamma$}
Lors de la précédente section, nous avons ordonné les gènes d'après leurs $\mathcal{T}$. Ici, nous souhaitons ordonner les entités dans le but de pouvoir effectuer des retour arrière(backtrack) lorsque nous parcourons notre arbre de recherche. Nous définissons alors les concepts de taux de similarité locale et de taux se similarité globale d'un groupe.

\begin{definition}
Le \textbf{taux de similarite locale $\tau(g,g')$ d'un groupe $g$} par rapport à un groupe $g'$ correspond à la moyenne des sommes des moyennes des valeurs du masque de $g$ et $g'$.\\
Formellement, soit $g$ et $g'$ deux groupes d'une instance $\mathcal{I}$, $M_g$ le masque du groupe $g$, $M_{g'}$ le masque du groupe $g'$, $M_g[i]$ la valeur du masque du groupe $g$ en position $i$,
$$ \tau(g,g')= \frac{\sum_{i=1}^{|\mathcal{X}|}(\frac{M_g[i]+M_{g'}[i]}{2})}{|\mathcal{X}|}$$
\end{definition}

\begin{definition}
Le \textbf{taux de similarité globale $\Gamma(g)$ d'un groupe $g$} correspond à la moyenne de ses taux de similarité locaux.\\
Formellement, soit $g$ le groupe d'une instance $\mathcal{I}$,
$$ \Gamma(g)=\frac{\sum_{g'=1}^{|\mathcal{G}|}\tau(g,g') /g \neq g'}{|\mathcal{G}|-1} $$
\end{definition}

Pour caractériser une instance, nous devons comparer chaque entité n'appartenant pas au même groupe entre elles. Notre idée est de placer les entités ayant le plus de chance de ne pas caractériser l'instance en premier car cela nous permettrait d'effectuer une coupure le plus haut possible dans notre arbre de recherche. Pour cela nous trions les groupes par ordre décroissant sur leurs taux de similarité globaux $\Gamma$.



\subparagraph{Résultats}

%\begin{figure}[H]
%\centering
%	\begin{minipage}[c]{0.49\linewidth}
%	\centering
%	\begin{tikzpicture}[scale=0.8]
\begin{axis}[
legend entries={rch10, rch10 $\Gamma$ statique},
%legend style={at={(0.5,1.03)},anchor=south},legend columns=3
xlabel={Caractérisation de taille k},
ylabel={Nombre de comparaisons d'entités},
xmin={24},
xmax={40},
%title={Résolution sans heuristique de rch10 et s3836-0 sur les comparaisons}
]
\addplot +[mark=none] table[x=k,y=nbComp]{./resultats/sh_rch10.dat};
\addplot +[mark=none] table[x=k,y=nbComp]{./resultats/gamma_rch10.dat};
\end{axis}
\end{tikzpicture}
%	\end{minipage}
%	\begin{minipage}[c]{0.49\linewidth}
%	\centering
%	\begin{tikzpicture}[scale=0.8]
\begin{axis}[
legend entries={rch10, rch10 $\Gamma$},
%legend style={at={(0.5,1.03)},anchor=south},legend columns=3
xlabel={Caractérisation de taille k},
ylabel={Temps d'éxécution en seconde},
xmin={24},
xmax={40}]

\addplot +[mark=none] table[x=k,y=temps]{./resultats/sh_rch10.dat};
\addplot +[mark=none] table[x=k,y=temps]{./resultats/gamma_rch10.dat};
\end{axis}
\end{tikzpicture}
%	\end{minipage}
%\caption{Heuristique $\Gamma$ sur rch10}
%\end{figure}
%
%\begin{figure}[H]
%\centering
%	\begin{minipage}[c]{0.49\linewidth}
%	\centering
%	\begin{tikzpicture}[scale=0.8]
\begin{axis}[
legend entries={rch10, rch10 $\Gamma$ statique, rch10 $\Gamma$ dynamique},
%legend style={at={(0.5,1.03)},anchor=south},legend columns=3
xlabel={Caractérisation de taille k},
ylabel={Nombre de comparaisons d'entités},
xmin={15},
xmax={27},
%title={Résolution sans heuristique de rch10 et s3836-0 sur les comparaisons}
]
\addplot +[mark=none] table[x=k,y=nbComp]{./resultats/sh_s3836.dat};
\addplot +[mark=none] table[x=k,y=nbComp]{./resultats/gamma_s3836.dat};
\addplot +[mark=none] table[x=k,y=nbComp]{./resultats/gamma_dynamique_s3836.dat};
\end{axis}
\end{tikzpicture}
%	\end{minipage}
%	\begin{minipage}[c]{0.49\linewidth}
%	\centering
%	\begin{tikzpicture}[scale=0.8]
\begin{axis}[
legend entries={s3836-0,s3836-0 $\Gamma$},
%legend style={at={(0.5,1.03)},anchor=south},legend columns=3
xlabel={Caractérisation de taille k},
ylabel={Temps d'éxécution en seconde},
xmin={15},
xmax={27}]
\addplot +[mark=none] table[x=k,y=temps]{./resultats/sh_s3836.dat};
\addplot +[mark=none] table[x=k,y=temps]{./resultats/gamma_s3836.dat};
\end{axis}
\end{tikzpicture}
%	\end{minipage}
%\caption{Heuristique $\Gamma$ sur s3836-0}
%\end{figure}
%
%\begin{figure}[H]
%\centering
%	\begin{minipage}[c]{0.49\linewidth}
%	\centering
%	\begin{tikzpicture}[scale=0.8]
\begin{axis}[
legend entries={rch10 $\Gamma$, rch10 $\Gamma$ dynamique},
%legend style={at={(0.5,1.03)},anchor=south},legend columns=3
xlabel={Caractérisation de taille k},
ylabel={Nombre de comparaisons d'entités},
xmin={33},
xmax={50},
]
\addplot +[mark=none] table[x=k,y=nbComp]{./resultats/gamma_rch10.dat};
\addplot +[mark=none] table[x=k,y=nbComp]{./resultats/gamma_dynamique_rch10.dat};
\end{axis}
\end{tikzpicture}
%	\end{minipage}
%	\begin{minipage}[c]{0.49\linewidth}
%	\centering
%	\begin{tikzpicture}[scale=0.8]
\begin{axis}[
legend entries={rch10, rch10 $\Gamma$, rch10 $\Gamma$ dynamique},
%legend style={at={(0.5,1.03)},anchor=south},legend columns=3
xlabel={Caractérisation de taille k},
ylabel={Temps d'éxécution en seconde},
xmin={24},
xmax={40}]
%\addplot +[mark=none] table[x=k,y=temps]{./resultats/sh_rch10.dat};
\addplot +[mark=none] table[x=k,y=temps]{./resultats/gamma_rch10.dat};
\addplot +[mark=none] table[x=k,y=temps]{./resultats/gamma_dynamique_rch10.dat};

\end{axis}
\end{tikzpicture}
%	\end{minipage}
%\caption{Comparaisons heuristique $\Gamma$ et heuristique $\Gamma$ dynamique}
%\end{figure}

Nous constatons que notre heuristique permet de baisser significativement le nombre de comparaisons .

\subsubsection{Heuristique de retour en arrière (backtracking)}
\paragraph{Heuristique des plus mauvais d'abord (pmda)}
Lorsque nous parcourons une instance pour une caractérisation de taille $k$. On peut affecter un poids sur la paire d'entités comparés deux à deux sur les indices de la combinaison courante.

Formellement, soit $e_a$, $e_z$ deux entités n'appartenant pas au même groupe, $\mathcal{C}$ une combinaison de $\mathcal{C}_{|\mathcal{X}|}^k$, $P$ le poids de la paire $\{e_a,e_z\}$.
$$ P = |\{i / e_a(i)=e_z(i), \forall i \in \mathcal{C}\}| $$

\begin{definition}
Si $P>=k-1$ alors la \textbf{paire d'entités $\{e_a,e_z\}$ est critique} car lors du parcours de la prochaine combinaisons, cette paire à la plus forte probabilité de permettre une coupure dans l'arbre de recherche.
\end{definition}

\subparagraph{Exemple :}
Soit les 2 entités suivante:
\begin{center}
\begin{tabular}{|c|c|c|c|c|c|c|c|}
\hline 
Groupe & \backslashbox{Entités}{Gènes} & g1 & g2 & g3 & g4 & g5 & g6 \\ 
\hline 
1 & e1 & 1 & 0 & 1 & 0 & 0 & 0 \\ 
\hline 
20 & e400 & 1 & 0 & 0 & 0 & 0 & 1 \\ 
\hline 
\end{tabular}
\end{center}
Supposons que nous sommes dans le cas d'une caractérisation de taille 3, nous parcourons les combinaisons de $\mathcal{C}_6^3 $. Supposons que le groupe 1 soit de taille 1, si nous sommes à la comparaison entre e1 et e400, cela signifie que nous avons déjà effectué 399 comparaisons d'entités. 

Regardons alors la combinaison courante 123 : nous apercevons que seul g3 permet la caractérisation. Comme plusieurs des combinaisons suivantes ne différeront que d'un élément, cet paire d'entités \{e1,e400\} à la plus forte probabilité d'être similiraire lors de la prochaine combinaison, nous gardons donc en mémoire cet ensemble qui a un poids égale à $k-1$.

Supposons que 123 n'ai pas caractérisé notre instance, nous parcourons alors 124 : nous commençons par parcourir les ensembles critiques obtenus lors du parcours précédent, soit la comparaison entre e1 et e400. Le poids est alors égale à $k$, ce qui signifie que nous pouvons arrêter notre recherche sur cette combinaison (car celle ci ne pourra en aucun cas caractériser l'instance). Cependant nous gardons en mémoire cet ensemble critique. Notons que nous avons fait là l'économie de 399 comparaisons d'entités.

Nous parcourons alors 125 : même constat , de nouveau une économie de 399 comparaisons d'entités.

Nous parcourons alors 126 : aucun effet, mais la paire \{e1,e400\} est toujours considérer comme critique.

Nous parcourons alors 134 : aucun effet, mais la paire  \{e1,e400\} n'est plus considéré comme critique car son poids $<k-1$.

\subparagraph{Résultats}
%\begin{figure}[H]
%\centering
%	\begin{minipage}[c]{0.49\linewidth}
%	\centering
%	\begin{tikzpicture}[scale=0.8]
\begin{axis}

[legend entries={rch10, rch10 pmda, rch10 pmdaNoMaj},
xlabel={Caractérisation de taille k},
ylabel={Nombre de comparaisons d'entités},
xmin={25},
xmax={40}]

\addplot +[mark=none] table[x=k,y=nbComp]{./resultats/sh_rch10.dat};
\addplot +[mark=none] table[x=k,y=nbComp]{./resultats/pmda_maj_rch10.dat};
\addplot +[mark=none] table[x=k,y=nbComp]{./resultats/pmda_noMaj_rch10.dat};
\end{axis}
\end{tikzpicture}
%	\end{minipage}
%	\begin{minipage}[c]{0.49\linewidth}
%	\centering
%	\begin{tikzpicture}[scale=0.8]
\begin{axis}[
legend entries={rch10, rch10 pmda, rch10 pmdaNoMaj},
xlabel={Caractérisation de taille k},
ylabel={Temps d'éxécution en seconde},
xmin={25},
xmax={40}]

\addplot +[mark=none] table[x=k,y=temps]{./resultats/sh_rch10.dat};
\addplot +[mark=none] table[x=k,y=temps]{./resultats/pmda_maj_rch10.dat};
\addplot +[mark=none] table[x=k,y=temps]{./resultats/pmda_noMaj_rch10.dat};

\end{axis}
\end{tikzpicture}

%	\end{minipage}
%\caption{Heuristiques pmda sur rch10}
%\end{figure}
%
%\begin{figure}[H]
%\centering
%	\begin{minipage}[c]{0.49\linewidth}
%	\centering
%	\begin{tikzpicture}[scale=0.8]
\begin{axis}[
legend entries={s3836-0,s3836-0 pmda, s3836-0 pmda maj},
%legend style={at={(0.5,1.03)},anchor=south},legend columns=3
xlabel={Caractérisation de taille k},
ylabel={Nombre de comparaisons d'entités},
xmin={15},
xmax={27},
%title={Résolution sans heuristique de rch10 et s3836-0 sur les comparaisons}
]
\addplot +[mark=none] table[x=k,y=nbComp]{./resultats/sh_s3836.dat};
\addplot +[mark=none] table[x=k,y=nbComp]{./resultats/pmda_noMaj_s3836.dat};
\addplot +[mark=none] table[x=k,y=nbComp]{./resultats/pmda_maj_s3836.dat};

\end{axis}
\end{tikzpicture}
%	\end{minipage}
%	\begin{minipage}[c]{0.49\linewidth}
%	\centering
%	\begin{tikzpicture}[scale=0.8]
\begin{axis}[
legend entries={s3836-0,s3836-0 pmda, s3836-0 pmdaNoMaj},
%legend style={at={(0.5,1.03)},anchor=south},legend columns=3
xlabel={Caractérisation de taille k},
ylabel={Temps d'éxécution en seconde},
xmin={15},
xmax={27}]
\addplot +[mark=none] table[x=k,y=temps]{./resultats/sh_s3836.dat};
\addplot +[mark=none] table[x=k,y=temps]{./resultats/pmda_maj_s3836.dat};
\addplot +[mark=none] table[x=k,y=temps]{./resultats/pmda_noMaj_s3836.dat};

\end{axis}
\end{tikzpicture}
%	\end{minipage}
%\caption{Heuristiques pmda sur s3836-0}
%\end{figure}

On constante que le nombre de comparaisons d'entités ainsi que le temps d'éxécution sont considérablement diminué par l'heuristique.

Ce type de résultat ayant été observé systématiquement sur un jeu de 15 instances(aléatoire et réelle), on peut conclure que cette heuristique est efficace.

\paragraph{Heuristique des valeurs tabous}
\subparagraph{Résultats}

%\begin{figure}[H]
%\centering
%	\begin{minipage}[c]{0.49\linewidth}
%	\centering
%	\begin{tikzpicture}[scale=0.8]
\begin{axis}[
legend entries={rch10, rch10 tabou},
%legend style={at={(0.5,1.03)},anchor=south},legend columns=3
xlabel={Caractérisation de taille k},
ylabel={Nombre de comparaisons d'entités},
xmin={25},
xmax={40},
%title={Résolution sans heuristique de rch10 et s3836-0 sur les comparaisons}
]
\addplot +[mark=none] table[x=k,y=nbComp]{./resultats/sh_rch10.dat};
\addplot +[mark=none] table[x=k,y=nbComp]{./resultats/tabou_rch10.dat};
\end{axis}
\end{tikzpicture}
%	\end{minipage}
%	\begin{minipage}[c]{0.49\linewidth}
%	\centering
%	\begin{tikzpicture}[scale=0.8]
\begin{axis}[
legend entries={rch10, rch10 tabou},
%legend style={at={(0.5,1.03)},anchor=south},legend columns=3
xlabel={Caractérisation de taille k},
ylabel={Nombre de comparaisons d'entités},
xmin={25},
xmax={40},
%title={Résolution sans heuristique de rch10 et s3836-0 sur les comparaisons}
]
\addplot +[mark=none] table[x=k,y=temps]{./resultats/sh_rch10.dat};
\addplot +[mark=none] table[x=k,y=temps]{./resultats/tabou_rch10.dat};
\end{axis}
\end{tikzpicture}
%	\end{minipage}
%\caption{Heuristique tabou sur rch10}
%\end{figure}
%
%\begin{figure}[H]
%\centering
%	\begin{minipage}[c]{0.49\linewidth}
%	\centering
%	\begin{tikzpicture}[scale=0.8]
\begin{axis}[
legend entries={rch10, rch10 tabou},
%legend style={at={(0.5,1.03)},anchor=south},legend columns=3
xlabel={Caractérisation de taille k},
ylabel={Nombre de comparaisons d'entités},
xmin={15},
xmax={27},
%title={Résolution sans heuristique de rch10 et s3836-0 sur les comparaisons}
]
\addplot +[mark=none] table[x=k,y=nbComp]{./resultats/sh_s3836.dat};
\addplot +[mark=none] table[x=k,y=nbComp]{./resultats/tabou_s3836.dat};
\end{axis}
\end{tikzpicture}
%	\end{minipage}
%	\begin{minipage}[c]{0.49\linewidth}
%	\centering
%	\begin{tikzpicture}[scale=0.8]
\begin{axis}[
legend entries={rch10, rch10 tabou},
%legend style={at={(0.5,1.03)},anchor=south},legend columns=3
xlabel={Caractérisation de taille k},
ylabel={Nombre de comparaisons d'entités},
xmin={15},
xmax={27},
%title={Résolution sans heuristique de rch10 et s3836-0 sur les comparaisons}
]
\addplot +[mark=none] table[x=k,y=temps]{./resultats/sh_s3836.dat};
\addplot +[mark=none] table[x=k,y=temps]{./resultats/tabou_s3836.dat};
\end{axis}
\end{tikzpicture}
%	\end{minipage}
%\caption{Heuristique tabou sur s3836-0}
%\end{figure}



\subsection{Comparaisons entre heuristique}

\begin{figure}[H]
\begin{center}
\begin{tikzpicture}
\begin{axis}[
legend entries={rch10 trie par $\mathcal{T}$,rch10 par $\Gamma$,rch10 pmda noMaj,rch10 pmda maj,rch10 tabou},
%legend style={at={(0.5,1.03)},anchor=south},legend columns=3
xlabel={Caractérisation de taille k},
ylabel={Nombre de comparaisons d'entités},
xmin={12},
xmax={35},
%ymax={1000000000}
%title={Résolution sans heuristique de rch10 et s3836-0 sur les comparaisons}
]
\addplot +[mark=none] table[x=k,y=nbComp]{./resultats/tau_rch10.dat};
\addplot +[mark=none] table[x=k,y=nbComp]{./resultats/gamma_rch10.dat};
\addplot +[mark=none] table[x=k,y=nbComp]{./resultats/pmda_noMaj_rch10.dat};
\addplot +[mark=none] table[x=k,y=nbComp]{./resultats/pmda_maj_rch10.dat};
\addplot +[mark=none] table[x=k,y=nbComp]{./resultats/tabou_rch10.dat};
%\addplot +[mark=none] table[x=k,y=nbComp]{./resultats/ratio_rch10.dat};
\end{axis}
\end{tikzpicture}
\end{center}
\caption{Comparaisons des heuristiques sur rch10: nombre de comparaisons}
\end{figure}


\begin{figure}[H]
\begin{center}
\begin{tikzpicture}
\begin{axis}[
legend entries={rch10 trie par $\mathcal{T}$,rch10 par $\Gamma$,rch10 pmda noMaj,rch10 pmda maj,rch10 tabou},
%legend style={at={(0.5,1.03)},anchor=south},legend columns=3
xlabel={Caractérisation de taille k},
ylabel={Nombre de comparaisons d'entités},
xmin={12},
xmax={35},
%ymax={100}
%title={Résolution sans heuristique de rch10 et s3836-0 sur les comparaisons}
]
\addplot +[mark=none] table[x=k,y=temps]{./resultats/tau_rch10.dat};
\addplot +[mark=none] table[x=k,y=temps]{./resultats/gamma_rch10.dat};
\addplot +[mark=none] table[x=k,y=temps]{./resultats/pmda_noMaj_rch10.dat};
\addplot +[mark=none] table[x=k,y=temps]{./resultats/pmda_maj_rch10.dat};
\addplot +[mark=none] table[x=k,y=temps]{./resultats/tabou_rch10.dat};
%\addplot +[mark=none] table[x=k,y=temps]{./resultats/ratio_rch10.dat};
\end{axis}
\end{tikzpicture}
\end{center}
\caption{Comparaisons des heuristiques sur rch10: temps d'exécution}
\end{figure}

\subsubsection{Résultats}

Nous présentons ici les résultats obtenus avec notre solveur \emph{Exact-Proj-Car2} (EPC2) qui fonctionne avec toute les meilleures heuristiques présentées dans cette section. Nous pouvons ainsi comparer nos résultats avec ceux obtenus par [CHHEL et al.,2013].

La colonne \textit{temps} indique le temps cumulé depuis le début de la recherche à $k=|\cal{X}|$.

\begin{center}
\begin{tabular}{|c|c|c|c|c|c|c|c|c|c|c|c|}
\hline 
\multirow{2}*{Instances} & \multirow{2}*{Entités(SR)} & \multirow{2}*{Gènes(SR)}& \multirow{2}*{$\Delta\cal{T}$}&  \multirow{2}*{$\rho$} & \multirow{2}*{$\sigma$} & \multirow{2}*{PL} & \multirow{2}*{EPC} & \multicolumn{2}{c|}{EPC2} \\
\cline{9-10} 
 & & & & & & & & k & temps \\
\hline 
s301-0 & 500 & 400 & 0.0247583 & 0.034 & 0.999917 & - & 13 & 13 & 2.07279\\ 
\hline 
s326-0 & 500 & 500 & 0.0264423 & 0.033 & 1 & - & 13 & 13 & 1.70796 \\ 
\hline 
s413-30 & 500 & 600 & 0.027368 & 0.035 & 1 & - & 13 & 13 & 1.05077\\ 
\hline 
s555-20 & 800 & 800 & 0.028946 & 0.039 & 0.999833 & - & 13 & 13 & 14.9249\\ 
\hline 
s625-20 & 500 & 1000 & 0.0270316 & 0.035 & 1 & - & 13 & 13 & 1.63653 \\ 
\hline 
s754-10 & 600 & 200 & 0.0266147 & 0.034 & 1 & - & 13 & \textcolor{red}{14} & 0.366073\\ 
\hline 
s882-20 & 600 & 400 & 0.0240376 & 0.032 & 1 & - & 13 & 13 & 543.176 \\ 
\hline 
s2501-70 & 800 & 800 & 0.0243325 & 0.033 & 1 & - & 15 & \textcolor{blue}{14} & \textcolor{cyan}{282.935} \\ 
\hline 
s31294-50 & 200 & 1000 & 0.0493203 & 0.065 & 0.9928 & 10 & 10 & 10 & 381.108 \\ 
\hline 
s3836-0 & 1000 & 1000 & 0.0187905 & 0.024 & 1 & - & 16 & \textcolor{blue}{15} & \textcolor{cyan}{13.072} \\ 
\hline 
rch8 & 132 (56) & 37 (27) & 0.339488 & 0.569 & 0.0670193 & \textbf{9} & \textbf{9} & \textbf{9} & 0.038407 \\ 
\hline 
raphv & 109 (108) & 155 (68) & 0.301636 & 0.588 & 0.419118 & \textbf{6} & \textbf{6} & \textbf{6} & 2.0551 \\ 
\hline 
raphy & 113 (112) & 155 (70) & 0.293601 & 0.609 & 0.667857 & \textbf{6} & \textbf{6} & \textbf{6} & 6.84672 \\ 
\hline 
rarep & 112 & 155 (72) & 0.295408 & 0.651 & 0.501984 & \textbf{12} & 39 & \textcolor{blue}{16} & 206.037 \\ 
\hline 
rch10 & 173 (112) & 98 (86) & 0.237992 & 0.626 & 0.0938845 & \textbf{10} & 25 & \textcolor{blue}{11} & 383.006\\ 
\hline 
\end{tabular} 
\end{center}

Nous avons placer en bleu les caractérisations de taille inférieur à EPC. Les temps de couleur cyan permettent de souligner la rapidité d'éxécution de EPC2 par rapport à EPC:  EPC2 caractérise l'instance s3836-0 en 13 secondes pour un k=15 alors que EPC ne réduit pas en dessous de k=16 en 10 minutes. Nous observons le même type de phénomène sur l'instance s2501-70. Notons toutefois que l'instance s754-10 est moins bien caractériser par EPC2, nous pensons que dans ce cas précis, l'influence du trie par $\cal{T}$ est négative, cela est du à $\Delta\cal{T}$ proche de 0. Notre solveur réduit de façon significative les caractérisations des instances réelles rarep et rch10. 


\subsection{Recherche incomplète}
\subsubsection{Résultats}
\begin{center}
\begin{tabular}{|c|c|c|c|c|c|c|c|c|c|c|}
\hline 
\multirow{2}*{Instances} & \multirow{2}*{Entités(SR)} & \multirow{2}*{Gènes(SR)}&\multirow{2}*{$\Delta\cal{T}$} & \multirow{2}*{$\rho$} & \multirow{2}*{$\sigma$} & \multirow{2}*{PL} & \multirow{2}*{LSPC} & \multicolumn{3}{c|}{Roulette proportionelle} \\
\cline{9-11} 
 & & & & & & & & k & temps & itérations \\
\hline 
s301-0 & 500  & 400 & 0.0247583 & 0.034 & 0.999917 & - & 14 & \textcolor{blue}{13} & 421.005 & 1612515 \\ 
\hline 
s326-0 & 500 & 500 & 0.0264423 & 0.033 & 1 & - & 14 & \textcolor{blue}{13} & 93.2472 & 390166\\ 
\hline 
s413-30 & 500 & 600 & 0.027368 & 0.035 & 1 & - & 13 & 13 & 428.634 & 1388939 \\ 
\hline 
s555-20 & 800 & 800 & 0.028946 & 0.039 & 0.999833 & - & 13 & 13 & 395.024 & 1286275 \\ 
\hline 
s625-20 & 500 & 1000 & 0.0270316 & 0.035 & 1 & - & 13 & 13 & 415.878 & 1359740 \\ 
\hline 
s754-10 & 600 & 200 & 0.0266147 & 0.034 & 1 & - & 14 & 14 & 25.8758 & 84672\\ 
\hline 
s882-20 & 600 & 400 & 0.0240376 & 0.032 & 1 & - & 14 & 14 & 3.02248 & 9113\\ 
\hline 
s2501-70 & 800 & 800 & 0.0243325 & 0.033 & 1 & - & 15 & 15 & 4.24 & 4350 \\ 
\hline 
s31294-50 & 200 & 1000 & 0.0493203 & 0.065 & 0.9928 & 10 & 11 & 11 & 0.962858 & 5273\\ 
\hline 
s3836-0 & 1000 & 1000 & 0.0187905 & 0.024 & 1 & - & 16 & 16 & 5.26595 & 1006 \\ 
\hline
rch8 & 132 (56) & 37 (27) & 0.339488 & 0.569 & 0.0670193 & \textbf{9} & 9 & 9 & 0.031263 & 22440 \\ 
\hline 
raphv & 109 (108) & 155 (68) & 0.301636 & 0.588 & 0.419118 & \textbf{6} & 9 & \textcolor{blue}{6} & 0.656676 & 470528\\ 
\hline 
raphy & 113 (112) & 155 (70)& 0.293601 & 0.609 & 0.667857 & \textbf{6} & 8 & \textcolor{blue}{6} & 0.87275 & 524344\\ 
\hline 
rarep & 112 & 155 (72) & 0.295408 & 0.651 & 0.501984 & \textbf{12} & 14 & 14 & 36.6272 & 17793513
\\ 
\hline 
rch10 & 173 (112) & 98 (86) & 0.237992 & 0.626 & 0.0938845 & \textbf{10} & 15 & \textcolor{blue}{12} & 65.6151 & 35676187\\ 
\hline 
\end{tabular} 
\end{center}


\subsection{Générateur d'instance difficile}
\subsubsection{Résultats} 
