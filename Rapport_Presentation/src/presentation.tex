\PassOptionsToPackage{table}{xcolor}
\documentclass{beamer}
\usepackage[utf8]{inputenc}
\usepackage[frenchb]{babel}
\usetheme{Warsaw}
\title{Minimisation du problème de caractérisation multiple}
\author{
J-M. Chantrein\\
\and
sous la direction de\\
F.Lardeux
\and
F.Saubion
}

\institute{LERIA}
\date{04 Juillet 2014 \\ \vspace{0.5cm}\large{Master 2 Intelligence décisionnelle}}
%\today
\setbeamercovered{invisible}

\setbeamertemplate{navigation symbols}{}
\addtobeamertemplate{footline}{\hfill\insertframenumber/\inserttotalframenumber\hspace{2em}\null}

%\usepackage[frenchb]{babel}
%\usepackage[utf8]{inputenc}

\usepackage{array,multirow,rotating,hhline,subfig}
\usepackage{amsmath,amssymb,mathrsfs}
\usepackage{float}
\usepackage{graphicx}
\usepackage{tikz}
\usetikzlibrary{fit,decorations.pathmorphing,arrows,shapes}
\usepackage{multirow}
\usepackage{array}
\usepackage{float}
\usepackage{pgfplots}
\usepackage{slashbox}
\usepackage[french, figure, onelanguage]{algorithm2e}


\tikzset{
  invisible/.style={opacity=0},
  visible on/.style={alt={#1{}{invisible}}},
  alt/.code args={<#1>#2#3}{%
    \alt<#1>{\pgfkeysalso{#2}}{\pgfkeysalso{#3}} % \pgfkeysalso doesn't change the path
  },
}

\tikzset{
  visible/.style={alt={#1{}{opacity=0.05}}},
  alt/.code args={<#1>#2#3}{%
    \alt<#1>{\pgfkeysalso{#2}}{\pgfkeysalso{#3}} % \pgfkeysalso doesn't change the path
  },
}

%\setbeameroption{show notes}

\begin{document}

\begin{frame}
\titlepage 
\end{frame}

%\AtBeginSection[]
%{
%  \begin{frame}<beamer>
%    \frametitle{Plan}
%    \tableofcontents[currentsection]
%  \end{frame}
%}


%\begin{frame}{Plan}
%	\tableofcontents[hideothersubsections]
%\end{frame}

\section*{Introduction}
\subsection*{Prérequis}
\begin{frame}{}

\begin{block}{}
	\begin{itemize}
		\item Certaines bactéries sont responsables de pathologies sur une large gammme de culture économiquement importante.
		\pause
		\item Ces pathologies induisent des pertes de rendement et diminuent la valeur marchande des semences. 
	\end{itemize}
\end{block}

\pause
\begin{exampleblock}{Pathovar phaseoli de Xanthomonas Axonopodis (Xap)}
	\begin{itemize}
		\item Il regroupe toutes les souches pathogènes sur le haricot.
		\pause
		\item Il n'est pas endémique en Europe.
		\pause
		\item Mais pour limiter son introduction, il est inscrit sur la liste des agents pathogènes de quarantaine.
		\pause
		\item La mise en quarantaine des containers de haricots induit de forts coûts de stockage.
	\end{itemize}
\end{exampleblock}
\end{frame}

\begin{frame}{}
\begin{block}{Identification des souches bactériennes}
	\begin{itemize}
		\item Obtenir un répertoire de gènes de virulence auprès de biologistes.
		\pause
		\item Chercher la plus petite combinaison de gène de virulence spécifique aux souches bactériennes.
		\pause
		\item Utiliser cette combinaison pour la mise au point d'un test PCR\footnote{Polymerase Chain Reaction} Multiplex (Micropuce ADN).
	\end{itemize}
\end{block}


\end{frame}

\begin{frame}{PCR Multiplex}
\begin{center}
	\begin{tikzpicture}
		\tikzstyle{spot}=[circle,draw,thick]
		\draw (0,0) rectangle (2,2);
		\node[visible=<2->][spot,fill=yellow!90] at (0.5,0.5) {};
		\node[visible=<2->][spot,fill=blue!70] at (1,0.5) {};
		\node[visible=<2->][spot,fill=red!50] at (1.5,0.5) {};
		
		\node[visible=<2->][spot,fill=yellow!70] at (0.5,1) {};
		\node[visible=<2->][spot,fill=blue!50] at (1,1) {};
		\node[visible=<2->][spot,fill=red!80] at (1.5,1) {};
		
		\node[spot,fill=yellow!50] at (0.5,1.5) {};
		\node[spot,fill=blue!90] at (1,1.5) {};
		\node[spot,fill=red!60] at (1.5,1.5) {};
	\end{tikzpicture}
\end{center}

\begin{block}{Remarque}<3>
	\begin{itemize}
		\item Nous cherchons la plus petite combinaison de gène de virulence afin de minimiser la taille de la puce et donc son coût.
	\end{itemize}
\end{block}
\end{frame}

\begin{frame}{Problème de caractérisation multiple (PCM)}
	\begin{block}{Données du PCM}
		\begin{itemize}
			\item Nous disposons d'un ensemble d'entités (souches bactériennes) regroupées en groupes (pathovars).
			\pause
			\item Chaque entité est définie par la présence/absence de caractères(gènes).
			\pause
			\item Une entité peut être vue comme étant une interprétation booléenne sur les gènes.
			\pause
			\item Les gènes sont donc considérés comme les variables du problème.
			\pause
			\item Chaque groupe fournit une table de vérité partielle d'une fonction booléenne dont l'interprétation est vraie pour le groupe en question et fausse pour tous les autres.  
		\end{itemize}
	\end{block}
\end{frame}

\begin{frame}{Représentation et résolution d'une instance PCM}
%	\begin{itemize}
%		\item Nous pouvons représenter une instance PCM sous la forme d'une matrice à 2 dimensions. 
%	\end{itemize}
	\vspace{0.5cm}
	 
	\begin{overprint}%{11.2cm}{20cm}
		\only<1>
		{
			\begin{minipage}[l]{0.46\linewidth}
				\begin{center}
					\begin{tabular}{|c||c|c|c|c|}
						\hline
						\multirow{2}{*}{Souches}&\multirow{2}{*}{Groupes}&\multicolumn{3}{c|}{Gènes
						}\\
						&&$a$&$b$&$c$\\
						\hline
						\hline
						$e_1$&\multirow{2}{*}{$g_1$}& 0 & 0 & 0\\
						\cline{1-1} \cline{3-5}
						$e_2$&& 0 & 0 & 1\\
						\hline
						\hline
						$e_3$&$g_2$& 1 &  1 & 1\\
						\hline
						\hline
						$e_4$&\multirow{2}{*}{$g_3$}& 1 & 1 & 0\\
						\cline{1-1} \cline{3-5}
						$e_5$&& 0 & 1 & 0\\
						\hline
					\end{tabular}
				\end{center}
			\end{minipage}
			\hspace{0.4cm}
			\begin{minipage}[r]{0.46\linewidth}
			\end{minipage}
		}
		\only<2-4>
		{
			\begin{minipage}[l]{0.46\linewidth}
				\begin{center}
					\begin{tabular}{|c||c|c|c|c|}
						\hline
						\multirow{2}{*}{Souches}&\multirow{2}{*}{Groupes}&\multicolumn{3}{c|}{Gènes
						}\\
						&&$a$&$b$&$c$\\
						\hline
						\hline
						$e_1$&\multirow{2}{*}{$g_1$}& 0 & 0 & 0\\
						\cline{1-1} \cline{3-5}
						$e_2$&& 0 & 0 & 1\\
						\hline
						\hline
						$e_3$&$g_2$& 1 &  1 & 1\\
						\hline
						\hline
						$e_4$&\multirow{2}{*}{$g_3$}& 1 & 1 & 0\\
						\cline{1-1} \cline{3-5}
						$e_5$&& 0 & 1 & 0\\
						\hline
					\end{tabular}
				\end{center}
			\end{minipage}
			\hspace{0.4cm}
			\begin{minipage}[r]{0.46\linewidth}
				Une caractérisation de taille $k=n$ est trivial:
				\begin{itemize}
					\item $g_1: (\lnot a \land \lnot b \land \lnot c) \lor (\lnot a \land \lnot b \land c)$
					\item<3-4> $g_2: a \land b \land c$
					\item<4> $g_3: (a \land b \land \lnot c) \lor (\lnot a \land b \land \lnot c)$ 
				\end{itemize}			
			\end{minipage}
		}
		\only<5-7>
		{
			\begin{minipage}[l]{0.46\linewidth}
				\begin{center}
					\begin{tabular}{|c||c|c|c|c|}
						\hline
						\multirow{2}{*}{Souches}&\multirow{2}{*}{Groupes}&\multicolumn{3}{c|}{Gènes
						}\\
						&&$a$&$b$&$c$\\
						\hline
						\hline
						$e_1$&\multirow{2}{*}{$g_1$}& 0 & 0 & 0\\
						\cline{1-1} \cline{3-5}
						$e_2$&& 0 & 0 & 1\\
						\hline
						\hline
						$e_3$&$g_2$& 1 &  1 & 1\\
						\hline
						\hline
						$e_4$&\multirow{2}{*}{$g_3$}& 1 & 1 & 0\\
						\cline{1-1} \cline{3-5}
						$e_5$&& 0 & 1 & 0\\
						\hline
					\end{tabular}
				\end{center}
			\end{minipage}
			\hspace{0.6cm}
			\begin{minipage}[r]{0.46\linewidth}
				\begin{block}{Résolution pour k=2}
					Il nous faut parcourir au maximum $C_3^2$ combinaisons:
					\begin{itemize}
						\item $\{a,b\}$
						\item<6-7> $\{b,c\}$
						\item<7> $\{a,c\}$ 
					\end{itemize}
				\end{block}
			\end{minipage}
		}
		\only<8>
		{
			\begin{minipage}[l]{0.46\linewidth}
				\begin{center}
					\begin{tabular}{|c||c|c|c|c|}
						\hline
						\multirow{2}{*}{Souches}&\multirow{2}{*}{Groupes}&\multicolumn{3}{c|}{Gènes
						}\\
						&&\cellcolor{blue!75}$a$&\cellcolor{blue!75}$b$&$c$\\
						\hline
						\hline
						$e_1$&\multirow{2}{*}{$g_1$}& \cellcolor{cyan}0 & 0 & 0\\
						\cline{1-1} \cline{3-5}
						$e_2$&& 0 & 0 & 1\\
						\hline
						\hline
						$e_3$&$g_2$& \cellcolor{cyan}1 & 1 & 1\\
						\hline
						\hline
						$e_4$&\multirow{2}{*}{$g_3$}& 1 & 1 & 0\\
						\cline{1-1} \cline{3-5}
						$e_5$&& 0 & 1 & 0\\
						\hline
					\end{tabular}
				\end{center}
			\end{minipage}
			\hspace{0.6cm}
			\begin{minipage}[r]{0.46\linewidth}
				\begin{block}{Résolution pour k=2}
					Combinaisons $\{a,b\}$:
					\begin{itemize}
						\item $e1(a) \not = e3(a) $ \\ $\Rightarrow$ On passe à l'entité suivante.
					\end{itemize}
				\end{block}
			\end{minipage}
		}
		\only<9>
		{
			\begin{minipage}[l]{0.46\linewidth}
				\begin{center}
					\begin{tabular}{|c||c|c|c|c|}
						\hline
						\multirow{2}{*}{Souches}&\multirow{2}{*}{Groupes}&\multicolumn{3}{c|}{Gènes
						}\\
						&&\cellcolor{blue!75}$a$&\cellcolor{blue!75}$b$&$c$\\
						\hline
						\hline
						$e_1$&\multirow{2}{*}{$g_1$}& \cellcolor{cyan}0 & 0 & 0\\
						\cline{1-1} \cline{3-5}
						$e_2$&& 0 & 0 & 1\\
						\hline
						\hline
						$e_3$&$g_2$& 1 & 1 & 1\\
						\hline
						\hline
						$e_4$&\multirow{2}{*}{$g_3$}& \cellcolor{cyan}1 & 1 & 0\\
						\cline{1-1} \cline{3-5}
						$e_5$&& 0 & 1 & 0\\
						\hline
					\end{tabular}
				\end{center}
			\end{minipage}
			\hspace{0.6cm}
			\begin{minipage}[r]{0.46\linewidth}
				\begin{block}{Résolution pour k=2}
					Combinaison $\{a,b\}$:
					\begin{itemize}
						\item $e1(a) \not = e4(a) $ \\ $\Rightarrow$ On passe à l'entité suivante.
					\end{itemize}
				\end{block}
			\end{minipage}
		}
		\only<10>
		{
			\begin{minipage}[l]{0.46\linewidth}
				\begin{center}
					\begin{tabular}{|c||c|c|c|c|}
						\hline
						\multirow{2}{*}{Souches}&\multirow{2}{*}{Groupes}&\multicolumn{3}{c|}{Gènes
						}\\
						&&\cellcolor{blue!75}$a$&\cellcolor{blue!75}$b$&$c$\\
						\hline
						\hline
						$e_1$&\multirow{2}{*}{$g_1$}& \cellcolor{cyan}0 & 0 & 0\\
						\cline{1-1} \cline{3-5}
						$e_2$&& 0 & 0 & 1\\
						\hline
						\hline
						$e_3$&$g_2$& 1 & 1 & 1\\
						\hline
						\hline
						$e_4$&\multirow{2}{*}{$g_3$}& 1 & 1 & 0\\
						\cline{1-1} \cline{3-5}
						$e_5$&& \cellcolor{cyan}0 & 1 & 0\\
						\hline
					\end{tabular}
				\end{center}
			\end{minipage}
			\hspace{0.6cm}
			\begin{minipage}[r]{0.46\linewidth}
				\begin{block}{Résolution pour k=2}
					Combinaison $\{a,b\}$:
					\begin{itemize}
						\item $e1(a) = e5(a) $ \\ $\Rightarrow$ Identique sur le gène $a$, on doit observer le gène $b$.
					\end{itemize}
				\end{block}
			\end{minipage}
		}
		\only<11>
		{
			\begin{minipage}[l]{0.46\linewidth}
				\begin{center}
					\begin{tabular}{|c||c|c|c|c|}
						\hline
						\multirow{2}{*}{Souches}&\multirow{2}{*}{Groupes}&\multicolumn{3}{c|}{Gènes
						}\\
						&&\cellcolor{blue!75}$a$&\cellcolor{blue!75}$b$&$c$\\
						\hline
						\hline
						$e_1$&\multirow{2}{*}{$g_1$}& 0 & \cellcolor{cyan}0 & 0\\
						\cline{1-1} \cline{3-5}
						$e_2$&& 0 & 0 & 1\\
						\hline
						\hline
						$e_3$&$g_2$& 1 & 1 & 1\\
						\hline
						\hline
						$e_4$&\multirow{2}{*}{$g_3$}& 1 & 1 & 0\\
						\cline{1-1} \cline{3-5}
						$e_5$&& 0 & \cellcolor{cyan}1 & 0\\
						\hline
					\end{tabular}
				\end{center}
			\end{minipage}
			\hspace{0.6cm}
			\begin{minipage}[r]{0.46\linewidth}
				\begin{block}{Résolution pour k=2}
					Combinaison $\{a,b\}$:
					\begin{itemize}
						\item $e1(b) \not = e5(b) $ \\ $\Rightarrow$ Nous devons maintenant observer $e2$.
					\end{itemize}
				\end{block}
			\end{minipage}
		}
		\only<12>
		{
			\begin{minipage}[l]{0.46\linewidth}
				\begin{center}
					\begin{tabular}{|c||c|c|c|c|}
						\hline
						\multirow{2}{*}{Souches}&\multirow{2}{*}{Groupes}&\multicolumn{3}{c|}{Gènes
						}\\
						&&\cellcolor{blue!75}$a$&\cellcolor{blue!75}$b$&$c$\\
						\hline
						\hline
						$e_1$&\multirow{2}{*}{$g_1$}& 0 & 0 & 0\\
						\cline{1-1} \cline{3-5}
						$e_2$&& \cellcolor{cyan}0 & 0 & 1\\
						\hline
						\hline
						$e_3$&$g_2$& \cellcolor{cyan}1 & 1 & 1\\
						\hline
						\hline
						$e_4$&\multirow{2}{*}{$g_3$}& 1 & 1 & 0\\
						\cline{1-1} \cline{3-5}
						$e_5$&& 0 & 1 & 0\\
						\hline
					\end{tabular}
				\end{center}
			\end{minipage}
			\hspace{0.6cm}
			\begin{minipage}[r]{0.46\linewidth}
				\begin{block}{Résolution pour k=2}
					Combinaison $\{a,b\}$:
					\begin{itemize}
						\item $e2(a) \not = e3(a) $ \\ $\Rightarrow$ On passe à l'entité suivante.
					\end{itemize}
				\end{block}
			\end{minipage}
		}
		\only<13>
		{
			\begin{minipage}[l]{0.46\linewidth}
				\begin{center}
					\begin{tabular}{|c||c|c|c|c|}
						\hline
						\multirow{2}{*}{Souches}&\multirow{2}{*}{Groupes}&\multicolumn{3}{c|}{Gènes
						}\\
						&&\cellcolor{blue!75}$a$&\cellcolor{blue!75}$b$&$c$\\
						\hline
						\hline
						$e_1$&\multirow{2}{*}{$g_1$}& 0 & 0 & 0\\
						\cline{1-1} \cline{3-5}
						$e_2$&& \cellcolor{cyan}0 & 0 & 1\\
						\hline
						\hline
						$e_3$&$g_2$& 1 & 1 & 1\\
						\hline
						\hline
						$e_4$&\multirow{2}{*}{$g_3$}& \cellcolor{cyan}1 & 1 & 0\\
						\cline{1-1} \cline{3-5}
						$e_5$&& 0 & 1 & 0\\
						\hline
					\end{tabular}
				\end{center}
			\end{minipage}
			\hspace{0.6cm}
			\begin{minipage}[r]{0.46\linewidth}
				\begin{block}{Résolution pour k=2}
					Combinaison $\{a,b\}$:
					\begin{itemize}
						\item $e2(a) \not = e4(a) $ \\ $\Rightarrow$ On passe à l'entité suivante.
					\end{itemize}
				\end{block}
			\end{minipage}
		}
		\only<14>
		{
			\begin{minipage}[l]{0.46\linewidth}
				\begin{center}
					\begin{tabular}{|c||c|c|c|c|}
						\hline
						\multirow{2}{*}{Souches}&\multirow{2}{*}{Groupes}&\multicolumn{3}{c|}{Gènes
						}\\
						&&\cellcolor{blue!75}$a$&\cellcolor{blue!75}$b$&$c$\\
						\hline
						\hline
						$e_1$&\multirow{2}{*}{$g_1$}& 0 & 0 & 0\\
						\cline{1-1} \cline{3-5}
						$e_2$&& \cellcolor{cyan}0 & 0 & 1\\
						\hline
						\hline
						$e_3$&$g_2$& 1 & 1 & 1\\
						\hline
						\hline
						$e_4$&\multirow{2}{*}{$g_3$}& 1 & 1 & 0\\
						\cline{1-1} \cline{3-5}
						$e_5$&& \cellcolor{cyan}0 & 1 & 0\\
						\hline
					\end{tabular}
				\end{center}
			\end{minipage}
			\hspace{0.6cm}
			\begin{minipage}[r]{0.46\linewidth}
				\begin{block}{Résolution pour k=2}
					Combinaison $\{a,b\}$:
					\begin{itemize}
						\item $e2(a) = e5(a) $ \\ $\Rightarrow$ Identique sur le gène $a$, on doit observer le gène $b$.
					\end{itemize}
				\end{block}
			\end{minipage}
		}
		\only<15-16>
		{
			\begin{minipage}[l]{0.46\linewidth}
				\begin{center}
					\begin{tabular}{|c||c|c|c|c|}
						\hline
						\multirow{2}{*}{Souches}&\multirow{2}{*}{Groupes}&\multicolumn{3}{c|}{Gènes
						}\\
						&&\cellcolor{blue!75}$a$&\cellcolor{blue!75}$b$&$c$\\
						\hline
						\hline
						$e_1$&\multirow{2}{*}{$g_1$}& 0 & 0 & 0\\
						\cline{1-1} \cline{3-5}
						$e_2$&& 0 & \cellcolor{cyan}0 & 1\\
						\hline
						\hline
						$e_3$&$g_2$& 1 & 1 & 1\\
						\hline
						\hline
						$e_4$&\multirow{2}{*}{$g_3$}& 1 & 1 & 0\\
						\cline{1-1} \cline{3-5}
						$e_5$&& 0 & \cellcolor{cyan}1 & 0\\
						\hline
					\end{tabular}
				\end{center}
			\end{minipage}
			\hspace{0.6cm}
			\begin{minipage}[r]{0.46\linewidth}
				\begin{block}{Résolution pour k=2}
					Combinaison $\{a,b\}$:
					\begin{itemize}
						\item $e2(b) \not = e5(b) $ \\ $\Rightarrow$ La combinaison $\{a,b\}$ permet de caractériser $g1$ des autres groupes.
						\item<16> Nous devons maintenant observer $g2$ vis à vis des autres groupes.
					\end{itemize}
				\end{block}
			\end{minipage}
		}
		\only<17>
		{
			\begin{minipage}[l]{0.46\linewidth}
				\begin{center}
					\begin{tabular}{|c||c|c|c|c|}
						\hline
						\multirow{2}{*}{Souches}&\multirow{2}{*}{Groupes}&\multicolumn{3}{c|}{Gènes
						}\\
						&&\cellcolor{blue!75}$a$&\cellcolor{blue!75}$b$&$c$\\
						\hline
						\hline
						$e_1$&\multirow{2}{*}{$g_1$}& 0 & 0 & 0\\
						\cline{1-1} \cline{3-5}
						$e_2$&& 0 & 0 & 1\\
						\hline
						\hline
						$e_3$&$g_2$& \cellcolor{cyan}1 & 1 & 1\\
						\hline
						\hline
						$e_4$&\multirow{2}{*}{$g_3$}& \cellcolor{cyan}1 & 1 & 0\\
						\cline{1-1} \cline{3-5}
						$e_5$&& 0 & 1 & 0\\
						\hline
					\end{tabular}
				\end{center}
			\end{minipage}
			\hspace{0.6cm}
			\begin{minipage}[r]{0.46\linewidth}
				\begin{block}{Résolution pour k=2}
					Combinaison $\{a,b\}$:
					\begin{itemize}
						\item $e3(a) = e4(a) $ \\ $\Rightarrow$ Identique sur le gène $a$, on doit observer le gène $b$.
					\end{itemize}
				\end{block}
			\end{minipage}
		}
		\only<18-19>
		{
			\begin{minipage}[l]{0.46\linewidth}
				\begin{center}
					\begin{tabular}{|c||c|c|c|c|}
						\hline
						\multirow{2}{*}{Souches}&\multirow{2}{*}{Groupes}&\multicolumn{3}{c|}{Gènes
						}\\
						&&\cellcolor{blue!75}$a$&\cellcolor{blue!75}$b$&$c$\\
						\hline
						\hline
						$e_1$&\multirow{2}{*}{$g_1$}& 0 & 0 & 0\\
						\cline{1-1} \cline{3-5}
						$e_2$&& 0 & 0 & 1\\
						\hline
						\hline
						$e_3$&$g_2$& 1 & \cellcolor{cyan}1 & 1\\
						\hline
						\hline
						$e_4$&\multirow{2}{*}{$g_3$}& 1 & \cellcolor{cyan}1 & 0\\
						\cline{1-1} \cline{3-5}
						$e_5$&& 0 & 1 & 0\\
						\hline
					\end{tabular}
				\end{center}
			\end{minipage}
			\hspace{0.6cm}
			\begin{minipage}[r]{0.46\linewidth}
				\begin{block}{Résolution pour k=2}
					Combinaison $\{a,b\}$:
					\begin{itemize}
						\item $e3(b) = e4(b) $ \\ $\Rightarrow$ Identique sur le gène $b$
						\item<19> Échec de la caractérisation avec la combinaison $\{a,b\}$.
					\end{itemize}
				\end{block}
			\end{minipage}
		}
		\only<20>
		{
			\begin{minipage}[l]{0.46\linewidth}
				\begin{center}
					\begin{tabular}{|c||c|c|c|c|}
						\hline
						\multirow{2}{*}{Souches}&\multirow{2}{*}{Groupes}&\multicolumn{3}{c|}{Gènes
						}\\
						&&$a$&\cellcolor{blue!75}$b$&\cellcolor{blue!75}$c$\\
						\hline
						\hline
						$e_1$&\multirow{2}{*}{$g_1$}& 0 & \cellcolor{cyan}0 & 0\\
						\cline{1-1} \cline{3-5}
						$e_2$&& 0 & 0 & 1\\
						\hline
						\hline
						$e_3$&$g_2$& 1 & \cellcolor{cyan}1 & 1\\
						\hline
						\hline
						$e_4$&\multirow{2}{*}{$g_3$}& 1 & 1 & 0\\
						\cline{1-1} \cline{3-5}
						$e_5$&& 0 & 1 & 0\\
						\hline
					\end{tabular}
				\end{center}
			\end{minipage}
			\hspace{0.6cm}
			\begin{minipage}[r]{0.46\linewidth}
				\begin{block}{Résolution pour k=2}
					Combinaison $\{b,c\}$:
					\begin{itemize}
						\item $e1(b) \not = e3(b) $ \\ $\Rightarrow$ On passe à l'entité suivante.
					\end{itemize}
				\end{block}
			\end{minipage}
		}
		\only<21>
		{
			\begin{minipage}[l]{0.46\linewidth}
				\begin{center}
					\begin{tabular}{|c||c|c|c|c|}
						\hline
						\multirow{2}{*}{Souches}&\multirow{2}{*}{Groupes}&\multicolumn{3}{c|}{Gènes
						}\\
						&&$a$&\cellcolor{blue!75}$b$&\cellcolor{blue!75}$c$\\
						\hline
						\hline
						$e_1$&\multirow{2}{*}{$g_1$}& 0 & \cellcolor{cyan}0 & 0\\
						\cline{1-1} \cline{3-5}
						$e_2$&& 0 & 0 & 1\\
						\hline
						\hline
						$e_3$&$g_2$& 1 & 1 & 1\\
						\hline
						\hline
						$e_4$&\multirow{2}{*}{$g_3$}& 1 & \cellcolor{cyan}1 & 0\\
						\cline{1-1} \cline{3-5}
						$e_5$&& 0 & 1 & 0\\
						\hline
					\end{tabular}
				\end{center}
			\end{minipage}
			\hspace{0.6cm}
			\begin{minipage}[r]{0.46\linewidth}
				\begin{block}{Résolution pour k=2}
					Combinaison $\{b,c\}$:
					\begin{itemize}
						\item $e1(b) \not = e4(b) $ \\ $\Rightarrow$ On passe à l'entité suivante.
					\end{itemize}
				\end{block}
			\end{minipage}
		}
		\only<22>
		{
			\begin{minipage}[l]{0.46\linewidth}
				\begin{center}
					\begin{tabular}{|c||c|c|c|c|}
						\hline
						\multirow{2}{*}{Souches}&\multirow{2}{*}{Groupes}&\multicolumn{3}{c|}{Gènes
						}\\
						&&$a$&\cellcolor{blue!75}$b$&\cellcolor{blue!75}$c$\\
						\hline
						\hline
						$e_1$&\multirow{2}{*}{$g_1$}& 0 & \cellcolor{cyan}0 & 0\\
						\cline{1-1} \cline{3-5}
						$e_2$&& 0 & 0 & 1\\
						\hline
						\hline
						$e_3$&$g_2$& 1 & 1 & 1\\
						\hline
						\hline
						$e_4$&\multirow{2}{*}{$g_3$}& 1 & 1 & 0\\
						\cline{1-1} \cline{3-5}
						$e_5$&& 0 & \cellcolor{cyan}1 & 0\\
						\hline
					\end{tabular}
				\end{center}
			\end{minipage}
			\hspace{0.6cm}
			\begin{minipage}[r]{0.46\linewidth}
				\begin{block}{Résolution pour k=2}
					Combinaison $\{b,c\}$:
					\begin{itemize}
						\item $e1(b) \not = e5(b) $ \\ $\Rightarrow$ Nous devons maintenant observer $e2$.
					\end{itemize}
				\end{block}
			\end{minipage}
		}
		\only<23>
		{
			\begin{minipage}[l]{0.46\linewidth}
				\begin{center}
					\begin{tabular}{|c||c|c|c|c|}
						\hline
						\multirow{2}{*}{Souches}&\multirow{2}{*}{Groupes}&\multicolumn{3}{c|}{Gènes
						}\\
						&&$a$&\cellcolor{blue!75}$b$&\cellcolor{blue!75}$c$\\
						\hline
						\hline
						$e_1$&\multirow{2}{*}{$g_1$}& 0 & 0 & 0\\
						\cline{1-1} \cline{3-5}
						$e_2$&& 0 & \cellcolor{cyan}0 & 1\\
						\hline
						\hline
						$e_3$&$g_2$& 1 & \cellcolor{cyan}1 & 1\\
						\hline
						\hline
						$e_4$&\multirow{2}{*}{$g_3$}& 1 & 1 & 0\\
						\cline{1-1} \cline{3-5}
						$e_5$&& 0 & 1 & 0\\
						\hline
					\end{tabular}
				\end{center}
			\end{minipage}
			\hspace{0.6cm}
			\begin{minipage}[r]{0.46\linewidth}
				\begin{block}{Résolution pour k=2}
					Combinaison $\{b,c\}$:
					\begin{itemize}
						\item $e2(b) \not = e3(b) $ \\ $\Rightarrow$ On passe à l'entité suivante.
					\end{itemize}
				\end{block}
			\end{minipage}
		}
		\only<24>
		{
			\begin{minipage}[l]{0.46\linewidth}
				\begin{center}
					\begin{tabular}{|c||c|c|c|c|}
						\hline
						\multirow{2}{*}{Souches}&\multirow{2}{*}{Groupes}&\multicolumn{3}{c|}{Gènes
						}\\
						&&$a$&\cellcolor{blue!75}$b$&\cellcolor{blue!75}$c$\\
						\hline
						\hline
						$e_1$&\multirow{2}{*}{$g_1$}& 0 & 0 & 0\\
						\cline{1-1} \cline{3-5}
						$e_2$&& 0 & \cellcolor{cyan}0 & 1\\
						\hline
						\hline
						$e_3$&$g_2$& 1 & 1 & 1\\
						\hline
						\hline
						$e_4$&\multirow{2}{*}{$g_3$}& 1 & \cellcolor{cyan}1 & 0\\
						\cline{1-1} \cline{3-5}
						$e_5$&& 0 & 1 & 0\\
						\hline
					\end{tabular}
				\end{center}
			\end{minipage}
			\hspace{0.6cm}
			\begin{minipage}[r]{0.46\linewidth}
				\begin{block}{Résolution pour k=2}
					Combinaison $\{b,c\}$:
					\begin{itemize}
						\item $e2(b) \not = e4(b) $ \\ $\Rightarrow$ On passe à l'entité suivante.
					\end{itemize}
				\end{block}
			\end{minipage}
		}
		\only<25-26>
		{
			\begin{minipage}[l]{0.46\linewidth}
				\begin{center}
					\begin{tabular}{|c||c|c|c|c|}
						\hline
						\multirow{2}{*}{Souches}&\multirow{2}{*}{Groupes}&\multicolumn{3}{c|}{Gènes
						}\\
						&&$a$&\cellcolor{blue!75}$b$&\cellcolor{blue!75}$c$\\
						\hline
						\hline
						$e_1$&\multirow{2}{*}{$g_1$}& 0 & 0 & 0\\
						\cline{1-1} \cline{3-5}
						$e_2$&& 0 & \cellcolor{cyan}0 & 1\\
						\hline
						\hline
						$e_3$&$g_2$& 1 & 1 & 1\\
						\hline
						\hline
						$e_4$&\multirow{2}{*}{$g_3$}& 1 & 1 & 0\\
						\cline{1-1} \cline{3-5}
						$e_5$&& 0 & \cellcolor{cyan}1 & 0\\
						\hline
					\end{tabular}
				\end{center}
			\end{minipage}
			\hspace{0.6cm}
			\begin{minipage}[r]{0.46\linewidth}
				\begin{block}{Résolution pour k=2}
					Combinaison $\{b,c\}$:
					\begin{itemize}
						\item $e2(b) \not = e5(b) $ \\ $\Rightarrow$ La combinaison $\{b,c\}$ permet de caractériser $g1$ des autres groupes.
						\item<26> Nous devons maintenant observer $g2$ vis à vis des autres groupes.
					\end{itemize}
				\end{block}
			\end{minipage}
		}
		\only<27>
		{
			\begin{minipage}[l]{0.46\linewidth}
				\begin{center}
					\begin{tabular}{|c||c|c|c|c|}
						\hline
						\multirow{2}{*}{Souches}&\multirow{2}{*}{Groupes}&\multicolumn{3}{c|}{Gènes
						}\\
						&&$a$&\cellcolor{blue!75}$b$&\cellcolor{blue!75}$c$\\
						\hline
						\hline
						$e_1$&\multirow{2}{*}{$g_1$}& 0 & 0 & 0\\
						\cline{1-1} \cline{3-5}
						$e_2$&& 0 & 0 & 1\\
						\hline
						\hline
						$e_3$&$g_2$& 1 & \cellcolor{cyan}1 & 1\\
						\hline
						\hline
						$e_4$&\multirow{2}{*}{$g_3$}& 1 & \cellcolor{cyan}1 & 0\\
						\cline{1-1} \cline{3-5}
						$e_5$&& 0 & 1 & 0\\
						\hline
					\end{tabular}
				\end{center}
			\end{minipage}
			\hspace{0.6cm}
			\begin{minipage}[r]{0.46\linewidth}
				\begin{block}{Résolution pour k=2}
					Combinaison $\{b,c\}$:
					\begin{itemize}
						\item $e3(b) = e4(b) $ \\ $\Rightarrow$ Identique sur le gène $b$, on doit observer le gène $c$.
					\end{itemize}
				\end{block}
			\end{minipage}
		}
		\only<28>
		{
			\begin{minipage}[l]{0.46\linewidth}
				\begin{center}
					\begin{tabular}{|c||c|c|c|c|}
						\hline
						\multirow{2}{*}{Souches}&\multirow{2}{*}{Groupes}&\multicolumn{3}{c|}{Gènes
						}\\
						&&$a$&\cellcolor{blue!75}$b$&\cellcolor{blue!75}$c$\\
						\hline
						\hline
						$e_1$&\multirow{2}{*}{$g_1$}& 0 & 0 & 0\\
						\cline{1-1} \cline{3-5}
						$e_2$&& 0 & 0 & 1\\
						\hline
						\hline
						$e_3$&$g_2$& 1 & 1 & \cellcolor{cyan}1\\
						\hline
						\hline
						$e_4$&\multirow{2}{*}{$g_3$}& 1 & 1 & \cellcolor{cyan}0\\
						\cline{1-1} \cline{3-5}
						$e_5$&& 0 & 1 & 0\\
						\hline
					\end{tabular}
				\end{center}
			\end{minipage}
			\hspace{0.6cm}
			\begin{minipage}[r]{0.46\linewidth}
				\begin{block}{Résolution pour k=2}
					Combinaison $\{b,c\}$:
					\begin{itemize}
						\item $e3(c) \not = e4(c) $ \\ $\Rightarrow$ On passe à l'entité suivante.
					\end{itemize}
				\end{block}
			\end{minipage}
		}
		\only<29>
		{
			\begin{minipage}[l]{0.46\linewidth}
				\begin{center}
					\begin{tabular}{|c||c|c|c|c|}
						\hline
						\multirow{2}{*}{Souches}&\multirow{2}{*}{Groupes}&\multicolumn{3}{c|}{Gènes
						}\\
						&&$a$&\cellcolor{blue!75}$b$&\cellcolor{blue!75}$c$\\
						\hline
						\hline
						$e_1$&\multirow{2}{*}{$g_1$}& 0 & 0 & 0\\
						\cline{1-1} \cline{3-5}
						$e_2$&& 0 & 0 & 1\\
						\hline
						\hline
						$e_3$&$g_2$& 1 & \cellcolor{cyan}1 & 1\\
						\hline
						\hline
						$e_4$&\multirow{2}{*}{$g_3$}& 1 & 1 & 0\\
						\cline{1-1} \cline{3-5}
						$e_5$&& 0 & \cellcolor{cyan}1 & 0\\
						\hline
					\end{tabular}
				\end{center}
			\end{minipage}
			\hspace{0.6cm}
			\begin{minipage}[r]{0.46\linewidth}
				\begin{block}{Résolution pour k=2}
					Combinaison $\{b,c\}$:
					\begin{itemize}
						\item $e3(b) = e5(b) $ \\ $\Rightarrow$ Identique sur le gène $b$, on doit observer le gène $c$.
					\end{itemize}
				\end{block}
			\end{minipage}
		}
		\only<30-31>
		{
			\begin{minipage}[l]{0.46\linewidth}
				\begin{center}
					\begin{tabular}{|c||c|c|c|c|}
						\hline
						\multirow{2}{*}{Souches}&\multirow{2}{*}{Groupes}&\multicolumn{3}{c|}{Gènes
						}\\
						&&$a$&\cellcolor{blue!75}$b$&\cellcolor{blue!75}$c$\\
						\hline
						\hline
						$e_1$&\multirow{2}{*}{$g_1$}& 0 & 0 & 0\\
						\cline{1-1} \cline{3-5}
						$e_2$&& 0 & 0 & 1\\
						\hline
						\hline
						$e_3$&$g_2$& 1 & 1 & \cellcolor{cyan}1\\
						\hline
						\hline
						$e_4$&\multirow{2}{*}{$g_3$}& 1 & 1 & 0\\
						\cline{1-1} \cline{3-5}
						$e_5$&& 0 & 1 & \cellcolor{cyan}0\\
						\hline
					\end{tabular}
				\end{center}
			\end{minipage}
			\hspace{0.6cm}
			\begin{minipage}[r]{0.46\linewidth}
				\begin{block}{Résolution pour k=2}
					Combinaison $\{b,c\}$:
					\begin{itemize}
						\item $e3(c) \not = e5(c) $ \\ $\Rightarrow$ La combinaison $\{b,c\}$ permet de caractériser $g2$ des autres groupes.
						\item<31> $\Rightarrow$ La combinaison $\{b,c\}$ permet de caractériser cette instance.
					\end{itemize}
				\end{block}
			\end{minipage}
		}
		\only<32-34>
		{
			\begin{minipage}[l]{0.46\linewidth}
				\begin{center}
					\begin{tabular}{|c||c|c|c|c|}
						\hline
						\multirow{2}{*}{Souches}&\multirow{2}{*}{Groupes}&\multicolumn{3}{c|}{Gènes
						}\\
						&&$a$&$b$&$c$\\
						\hline
						\hline
						$e_1$&\multirow{2}{*}{$g_1$}& 0 & 0 & 0\\
						\cline{1-1} \cline{3-5}
						$e_2$&& 0 & 0 & 1\\
						\hline
						\hline
						$e_3$&$g_2$& 1 &  1 & 1\\
						\hline
						\hline
						$e_4$&\multirow{2}{*}{$g_3$}& 1 & 1 & 0\\
						\cline{1-1} \cline{3-5}
						$e_5$&& 0 & 1 & 0\\
						\hline
					\end{tabular}
				\end{center}
			\end{minipage}
			\hspace{0.4cm}
			\begin{minipage}[r]{0.46\linewidth}
				Formules de caractérisation de l'instance:
				\begin{itemize}
					\item $g_1:(\lnot b \land \lnot c) \lor (\lnot b \land c)$
					\item<33-34> $g_2: b \land c$
					\item<34> $g_3: b \land \lnot c$ 
				\end{itemize}			
			\end{minipage}
		}
	\end{overprint}
\end{frame}

\begin{frame}{}
	\begin{block}{Minimisation du problème de caractérisation multiple(MIN-PCM)}
		\begin{itemize}
			\item Consiste à chercher la caractérisation de taille $k$ contenant le moins de variables.
			\pause
			\item Permet de minimiser la taille du PCR-Multiplex et donc son coût.
			\pause
			\item Lors de la recherche, nous essayons de caractériser à partir de $k=n-1$ jusqu'à ce qu' il n 'y ai plus de caractérisation possible.
		\end{itemize}
	\end{block}
\end{frame}

\begin{frame}
	\begin{alertblock}{Complexité}
			\begin{itemize}
				\item Le PCM appartient à la classe de complexité NP-complet.
				\pause
				\item MIN-PCM appartient à la classe de complexité NP-difficile.
				\pause
				\item Plus spécifiquemment, le PCM appartient à la classe de complexité W[2]-complet.
				\pause
				\item MIN-PCM appartient à la classe de complexité W[2]-difficile.
				\pause
				\item La seule possibilité d'améliorer significativement la résolution complète est d'utiliser des heuristiques sur les choix de variables
				\pause
			\end{itemize}
	\end{alertblock}
	\begin{block}{Problématique}
		\begin{itemize}
				\item A partir des conclusions des précédents travaux, nous cherchons à savoir quels sont les gènes à examiner en priorités lors de la résolution d'une instance MIN-PCM.
			\end{itemize}
	\end{block}
	
\end{frame}

\begin{frame}{Plan}
	\tableofcontents
\end{frame}

\section*{Contributions}

\subsection{Définition d'une instance difficile}
\begin{frame}{}%{Comparaisons de résolutions d'instance réelle et aléatoire}
	\begin{center}
		\begin{tabular}{|c|c|c|c|c|c|c|}
		\hline 
		Instances & Entités & Gènes & Borne min connue \\ 
		\hline 
		s3836-0 & 1000 & 1000 & 16\\ 
		\hline
		rch10 & 173 & 98 & \textbf{10} \\ 
		\hline
		\end{tabular} 
	\end{center}

	\begin{overprint}
		\begin{itemize}
			\only<2-3> {\item Borne minimum connue de  s3836-0 obtenue par une méthode de type recherche arborescente.}
			\only<3> {\item Borne minimum connue de  rch10 obtenue par une transformation du MIN-PCM en MIN-ONE, résolu par le solveur \emph{cplex} d'IBM.}
		\end{itemize}
		\only<4>
		{
			\begin{figure}
				\begin{tikzpicture}[scale=0.8]
\begin{axis}[
legend entries={rch10,s3836-0},
%legend style={at={(0.5,1.03)},anchor=south},legend columns=3
xlabel={Caractérisation de taille k},
ylabel={Nombre de comparaisons d'entités},
xmin={14},
xmax={40}
]
\addplot +[mark=none] table[x=k,y=nbComp]{./resultats/sh_rch10.dat};
\addplot +[mark=none] table[x=k,y=nbComp]{./resultats/sh_s3836.dat};
\end{axis}
\end{tikzpicture}


			\end{figure}
		}
		\only<5>
		{
			\begin{figure}
				\begin{tikzpicture}[scale=0.8]
\begin{axis}[
legend entries={rch10,s3836-0},
%legend style={at={(0.5,1.03)},anchor=south},legend columns=3
xlabel={Caractérisation de taille k},
ylabel={Temps d'éxécution en seconde},
xmin={14},
xmax={40}]
\addplot +[mark=none] table[x=k,y=temps]{./resultats/sh_rch10.dat};
\addplot +[mark=none] table[x=k,y=temps]{./resultats/sh_s3836.dat};
\end{axis}
\end{tikzpicture}


			\end{figure}
		}
	\end{overprint} 
\end{frame}

\begin{frame}
	\begin{exampleblock}{Masque et ratio d'un groupe}
	Soit le groupe suivant:
		\begin{center}
			\begin{tabular}{|c|c|c|c|c|}
			\hline 
			\backslashbox{Entités}{Gènes} & g0 & g1 & g2 & g3 \\ 
			\hline 
			e1 & 1 & 0 & 1 & 1 \\
			\hline 
			e2 & 1 & 0 & 1 & 0 \\ 
			\hline 
			e3 & 1 & 0 & 0 & 0 \\ 
			\hline 
			e4 & 1 & 0 & 1 & 0 \\ 
			\hline 
			e5 & 1 & 0 & 1 & 1 \\ 
			\hline 
			\hline
			Masque & 1 & 0 & 0.8 & 0.4 \\
			\hline
			\end{tabular}
		\end{center}
	\pause
	Le ratio $r$ de ce groupe est $r=2/4$ soit $r=0.5$
	\end{exampleblock}
	
	\pause
		\begin{itemize}
			\item Les instances réelles(resp. aléatoire) sont constituées de groupes ayant un fort(resp. faible) ratio.
		\end{itemize}
\end{frame}

\begin{frame}
	\begin{exampleblock}{Image d'une instance et taux de similarité $\cal{T}$ des gènes}
	Soit l'image suivante:
		\begin{center}
			\begin{tabular}{|c|c|c|c|c|}
			\hline 
			\backslashbox{Groupes}{Gènes} & g0 & g1 & g2 & g3 \\ 
			\hline 
			Masque de $g1$ & 0.9 & 0.1 & 0.5 & 0.6 \\ 
			\hline 
			Masque de $g2$ & 0.9 & 0.1 & 0.5 & 0.6 \\
			\hline 
			Masque de $g3$ & 0.9 & 0.1 & 0.5 & 0.6 \\
			\hline 
			\hline
			$\cal{T}$ & 0.8 & 0.8 & 0 & 0.2 \\ 
			\hline
			\end{tabular}
		\end{center}
	\end{exampleblock}
	
	\pause
	\begin{overprint}
		\only<2-3>
		{
			\begin{itemize}
				\item<2-3> Plus la moyenne des taux $\cal{T}$ est élevée, plus l'instance sera difficile à résoudre.		
				\item<3> Les instances réelles présentent une moyenne de taux $\cal{T}$ beaucoup plus élevée que les instances aléatoires(i.e: les groupes sont fortement similaires).
			\end{itemize}
		}
		\only<4-5>
		{
			\begin{block}{Heuristique de trie des gènes par $\cal{T}$}
				\begin{itemize}
					\item<4-5> Les gènes présentant des taux de similarité faible ont plus de chance de caractériser une instance.
					\item<5> L' heuristique de trie par $\cal{T}$ consiste à examiner en priorité les gènes ayant de faible taux de similarité.
				\end{itemize}
			\end{block}
		}
	\end{overprint}
\end{frame}

\begin{frame}
				\begin{center}
					\begin{tabular}{|c||c|c|c|c|}
						\hline
						\multirow{2}{*}{Souches}&\multirow{2}{*}{Groupes}&\multicolumn{3}{c|}{Gènes
						}\\
						&&$a$&$b$&$c$\\
						\hline
						\hline
						$e_1$&\multirow{2}{*}{$g_1$}& 0 & 0 & 0\\
						\cline{1-1} \cline{3-5}
						$e_2$&& 0 & 0 & 1\\
						\hline
						\hline
						$e_3$&$g_2$& 1 &  1 & 1\\
						\hline
						\hline
						$e_4$&\multirow{2}{*}{$g_3$}& 1 & 1 & 0\\
						\cline{1-1} \cline{3-5}
						$e_5$&& 0 & 1 & 0\\
						\hline
					\end{tabular}
				\end{center}
				\begin{block}{Heuristique de trie des groupes par $\Gamma$}
					\begin{itemize}
					\item Une caractérisation exige un parcours exhaustif de paires de groupes ($[g_1,g_2]$, $[g_1,g_3]$, $[g_2,g_3]$).
					\pause
					\item Idée: Parcourir en priorités les paires de groupes ayant des forts taux de similarité $\Gamma$ entre groupe ($[g_2,g_3]$, $[g_3,g_1]$, $[g_1,g_2]$).
					\pause
					\item Elimination rapide des combinaisons infructueuse.
					\end{itemize}
				\end{block}
\end{frame}

\subsection{Heuristiques pour une recherche exacte}
\begin{frame}{Heuristiques mises en \oe uvre sur une instance réelle}
	 
	\begin{overprint}
	\only<1>
	{
		\centering
		\begin{tikzpicture}[scale=0.8]
			\begin{axis}[
			legend entries={rch10 standard, rch10 masque 1, rch10 masque 2, rch10 trie par $\Gamma$, rch10 trie par $\mathcal{T}$,rch10 all},
			legend style={at={(1.3,1)},anchor=north},legend columns=1,
			xlabel={Caractérisation de taille k},
			ylabel={Nombre de comparaisons d'entités},
			xmin={11},
			xmax={35}]
			\addplot +[mark=none] table[x=k,y=nbComp]{./resultats/sh_rch10.dat};
%			\addplot +[mark=none] table[x=k,y=nbComp]{./resultats/tau_rch10.dat};
%			\addplot +[mark=none] table[x=k,y=nbComp]{./resultats/gamma_rch10.dat};
%			\addplot +[mark=none, color=green] table[x=k,y=nbComp]{./resultats/pmda_noMaj_rch10.dat};
%			\addplot +[mark=none] table[x=k,y=nbComp]{./resultats/tabou_rch10.dat};
%			\addplot +[mark=none, color=violet] table[x=k,y=nbComp]{./resultats/all_rch10.dat};
			\end{axis}
			\end{tikzpicture}
			\begin{itemize}
			  \item standard : parcours par ordre lexicographique des caractères. 
			\end{itemize}
	}
	\only<2>
	{
		\centering
		\begin{tikzpicture}[scale=0.8]
			\begin{axis}[
			legend entries={rch10 standard, rch10 masque 1, rch10 masque 2, rch10 trie par $\Gamma$, rch10 trie par $\mathcal{T}$,rch10 all},
			legend style={at={(1.3,1)},anchor=north},legend columns=1,
			xlabel={Caractérisation de taille k},
			ylabel={Nombre de comparaisons d'entités},
			xmin={11},
			xmax={35},
			%ymax={1000000000}
			%title={Résolution sans heuristique de rch10 et s3836-0 sur les comparaisons}
			]
			\addplot +[mark=none] table[x=k,y=nbComp]{./resultats/sh_rch10.dat};
			\addplot +[mark=none, color=green] table[x=k,y=nbComp]{./resultats/pmda_noMaj_rch10.dat};
			\addplot +[mark=none] table[x=k,y=nbComp]{./resultats/tabou_rch10.dat};
%			\addplot +[mark=none] table[x=k,y=nbComp]{./resultats/gamma_rch10.dat};
%			\addplot +[mark=none] table[x=k,y=nbComp]{./resultats/tau_rch10.dat};
%			\addplot +[mark=none, color=violet] table[x=k,y=nbComp]{./resultats/all_rch10.dat};
			\end{axis}
			\end{tikzpicture}
	}
	\only<3>
	{
		\centering
		\begin{tikzpicture}[scale=0.8]
			\begin{axis}[
			legend entries={rch10 standard, rch10 masque 1, rch10 masque 2, rch10 trie par $\Gamma$, rch10 trie par $\mathcal{T}$,rch10 all},
			legend style={at={(1.3,1)},anchor=north},legend columns=1,
			xlabel={Caractérisation de taille k},
			ylabel={Nombre de comparaisons d'entités},
			xmin={11},
			xmax={35},
			%ymax={1000000000}
			%title={Résolution sans heuristique de rch10 et s3836-0 sur les comparaisons}
			]
			\addplot +[mark=none] table[x=k,y=nbComp]{./resultats/sh_rch10.dat};
			\addplot +[mark=none, color=green] table[x=k,y=nbComp]{./resultats/pmda_noMaj_rch10.dat};
			\addplot +[mark=none] table[x=k,y=nbComp]{./resultats/tabou_rch10.dat};
			\addplot +[mark=none] table[x=k,y=nbComp]{./resultats/gamma_rch10.dat};
%			\addplot +[mark=none] table[x=k,y=nbComp]{./resultats/tau_rch10.dat};
%			\addplot +[mark=none, color=violet] table[x=k,y=nbComp]{./resultats/all_rch10.dat};
			\end{axis}
			\end{tikzpicture}
	}
	\only<4>
	{
		\centering
		\begin{tikzpicture}[scale=0.8]
			\begin{axis}[
			legend entries={rch10 standard, rch10 masque 1, rch10 masque 2, rch10 trie par $\Gamma$, rch10 trie par $\mathcal{T}$,rch10 all},
			legend style={at={(1.3,1)},anchor=north},legend columns=1,
			xlabel={Caractérisation de taille k},
			ylabel={Nombre de comparaisons d'entités},
			xmin={11},
			xmax={35},
			%ymax={1000000000}
			%title={Résolution sans heuristique de rch10 et s3836-0 sur les comparaisons}
			]
			\addplot +[mark=none] table[x=k,y=nbComp]{./resultats/sh_rch10.dat};
			\addplot +[mark=none, color=green] table[x=k,y=nbComp]{./resultats/pmda_noMaj_rch10.dat};
			\addplot +[mark=none] table[x=k,y=nbComp]{./resultats/tabou_rch10.dat};
			\addplot +[mark=none] table[x=k,y=nbComp]{./resultats/gamma_rch10.dat};
			\addplot +[mark=none, color=red] table[x=k,y=nbComp]{./resultats/tau_rch10.dat};
%			\addplot +[mark=none, color=violet] table[x=k,y=nbComp]{./resultats/all_rch10.dat};
			\end{axis}
			\end{tikzpicture}
	}
	\only<5>
	{
		\centering
		\begin{tikzpicture}[scale=0.8]
			\begin{axis}[
			legend entries={rch10 standard, rch10 masque 1, rch10 masque 2, rch10 trie par $\Gamma$, rch10 trie par $\mathcal{T}$,rch10 all},
			legend style={at={(1.3,1)},anchor=north},legend columns=1,
			xlabel={Caractérisation de taille k},
			ylabel={Nombre de comparaisons d'entités},
			xmin={11},
			xmax={35},
			%ymax={1000000000}
			%title={Résolution sans heuristique de rch10 et s3836-0 sur les comparaisons}
			]
			\addplot +[mark=none] table[x=k,y=nbComp]{./resultats/sh_rch10.dat};
			\addplot +[mark=none, color=green] table[x=k,y=nbComp]{./resultats/pmda_noMaj_rch10.dat};
			\addplot +[mark=none] table[x=k,y=nbComp]{./resultats/tabou_rch10.dat};
			\addplot +[mark=none] table[x=k,y=nbComp]{./resultats/gamma_rch10.dat};
			\addplot +[mark=none, color=red] table[x=k,y=nbComp]{./resultats/tau_rch10.dat};
			\addplot +[mark=none, color=violet] table[x=k,y=nbComp]{./resultats/all_rch10.dat};
			\end{axis}
			\end{tikzpicture}
	}
	\only<6>
	{
		\begin{itemize}
			\item Observons maintenant les temps d'éxécutions.
		\end{itemize}
	}
% 	\only<7>
% 	{
% 		\centering
% 		\begin{tikzpicture}[scale=0.8]
% 			\begin{axis}[
% 			legend entries={rch10 standard, rch10 masque 1, rch10 masque 2, rch10 trie par $\Gamma$, rch10 trie par $\mathcal{T}$,rch10 all},
% 			legend style={at={(1.3,1)},anchor=north},legend columns=1
% 			xlabel={Caractérisation de taille k},
% 			ylabel={Nombre de comparaisons d'entités},
% 			xmin={11},
% 			xmax={35},
% 			%ymax={1000000000}
% 			%title={Résolution sans heuristique de rch10 et s3836-0 sur les comparaisons}
% 			]
% 			\addplot +[mark=none] table[x=k,y=temps]{./resultats/sh_rch10.dat};
% %			\addplot +[mark=none] table[x=k,y=temps]{./resultats/tau_rch10.dat};
% %			\addplot +[mark=none] table[x=k,y=temps]{./resultats/gamma_rch10.dat};
% %			\addplot +[mark=none, color=green] table[x=k,y=temps]{./resultats/pmda_noMaj_rch10.dat};
% %			\addplot +[mark=none] table[x=k,y=temps]{./resultats/tabou_rch10.dat};
% %			\addplot +[mark=none, color=violet] table[x=k,y=temps]{./resultats/all_rch10.dat};
% 			\end{axis}
% 			\end{tikzpicture}
% 	}
% 	\only<8>
% 	{
% 		\centering
% 		\begin{tikzpicture}[scale=0.8]
% 			\begin{axis}[
% 			legend entries={rch10 standard, rch10 masque 1, rch10 masque 2, rch10 trie par $\Gamma$, rch10 trie par $\mathcal{T}$,rch10 all},
% 			legend style={at={(1.3,1)},anchor=north},legend columns=1
% 			xlabel={Caractérisation de taille k},
% 			ylabel={Nombre de comparaisons d'entités},
% 			xmin={11},
% 			xmax={35},
% 			%ymax={1000000000}
% 			%title={Résolution sans heuristique de rch10 et s3836-0 sur les comparaisons}
% 			]
% 			\addplot +[mark=none] table[x=k,y=temps]{./resultats/sh_rch10.dat};
% 			\addplot +[mark=none, color=green] table[x=k,y=temps]{./resultats/pmda_noMaj_rch10.dat};
% 			\addplot +[mark=none] table[x=k,y=temps]{./resultats/tabou_rch10.dat};
% %			\addplot +[mark=none] table[x=k,y=temps]{./resultats/gamma_rch10.dat};
% %			\addplot +[mark=none] table[x=k,y=temps]{./resultats/tau_rch10.dat};
% %			\addplot +[mark=none, color=violet] table[x=k,y=temps]{./resultats/all_rch10.dat};
% 			\end{axis}
% 			\end{tikzpicture}
% 	}
% 	\only<9>
% 	{
% 		\centering
% 		\begin{tikzpicture}[scale=0.8]
% 			\begin{axis}[
% 			legend entries={rch10 standard, rch10 masque 1, rch10 masque 2, rch10 trie par $\Gamma$, rch10 trie par $\mathcal{T}$,rch10 all},
% 			legend style={at={(1.3,1)},anchor=north},legend columns=1
% 			xlabel={Caractérisation de taille k},
% 			ylabel={Nombre de comparaisons d'entités},
% 			xmin={11},
% 			xmax={35},
% 			%ymax={1000000000}
% 			%title={Résolution sans heuristique de rch10 et s3836-0 sur les comparaisons}
% 			]
% 			\addplot +[mark=none] table[x=k,y=temps]{./resultats/sh_rch10.dat};
% 			\addplot +[mark=none, color=green] table[x=k,y=temps]{./resultats/pmda_noMaj_rch10.dat};
% 			\addplot +[mark=none] table[x=k,y=temps]{./resultats/tabou_rch10.dat};
% 			\addplot +[mark=none] table[x=k,y=temps]{./resultats/gamma_rch10.dat};
% %			\addplot +[mark=none] table[x=k,y=temps]{./resultats/tau_rch10.dat};
% %			\addplot +[mark=none, color=violet] table[x=k,y=temps]{./resultats/all_rch10.dat};
% 			\end{axis}
% 			\end{tikzpicture}
% 	}
% 	\only<10>
% 	{
% 		\centering
% 		\begin{tikzpicture}[scale=0.8]
% 			\begin{axis}[
% 			legend entries={rch10 standard, rch10 masque 1, rch10 masque 2, rch10 trie par $\Gamma$, rch10 trie par $\mathcal{T}$,rch10 all},
% 			legend style={at={(1.3,1)},anchor=north},legend columns=1
% 			xlabel={Caractérisation de taille k},
% 			ylabel={Nombre de comparaisons d'entités},
% 			xmin={11},
% 			xmax={35},
% 			%ymax={1000000000}
% 			%title={Résolution sans heuristique de rch10 et s3836-0 sur les comparaisons}
% 			]
% 			\addplot +[mark=none] table[x=k,y=temps]{./resultats/sh_rch10.dat};
% 			\addplot +[mark=none, color=green] table[x=k,y=temps]{./resultats/pmda_noMaj_rch10.dat};
% 			\addplot +[mark=none] table[x=k,y=temps]{./resultats/tabou_rch10.dat};
% 			\addplot +[mark=none] table[x=k,y=temps]{./resultats/gamma_rch10.dat};
% 			\addplot +[mark=none, color=red] table[x=k,y=temps]{./resultats/tau_rch10.dat};
% %			\addplot +[mark=none, color=violet] table[x=k,y=temps]{./resultats/all_rch10.dat};
% 			\end{axis}
% 			\end{tikzpicture}
% 	}
	\only<7>
	{
		\centering
		\begin{tikzpicture}[scale=0.8]
			\begin{axis}[
			legend entries={rch10 standard, rch10 masque 1, rch10 masque 2, rch10 trie par $\Gamma$, rch10 trie par $\mathcal{T}$,rch10 all},
			legend style={at={(1.3,1)},anchor=north},legend columns=1,
			xlabel={Caractérisation de taille k},
			ylabel={Temps d'éxécution en seconde},
			xmin={11},
			xmax={35},
			%ymax={1000000000}
			%title={Résolution sans heuristique de rch10 et s3836-0 sur les comparaisons}
			]
			\addplot +[mark=none] table[x=k,y=temps]{./resultats/sh_rch10.dat};
			\addplot +[mark=none, color=green] table[x=k,y=temps]{./resultats/pmda_noMaj_rch10.dat};
			\addplot +[mark=none] table[x=k,y=temps]{./resultats/tabou_rch10.dat};
			\addplot +[mark=none] table[x=k,y=temps]{./resultats/gamma_rch10.dat};
			\addplot +[mark=none, color=red] table[x=k,y=temps]{./resultats/tau_rch10.dat};
			\addplot +[mark=none, color=violet] table[x=k,y=temps]{./resultats/all_rch10.dat};
			\end{axis}
			\end{tikzpicture}
	}
	
	\end{overprint}
\end{frame}

%\begin{frame}{Heuristiques mises en \oe uvre sur une instance aléatoire}
%		\begin{overprint}
		\only<1>
		{
			\begin{tikzpicture}[scale=0.8]
			\begin{axis}[
			legend entries={s3836-0 standard, s3836-0 trie par $\mathcal{T}$,s3836-0 par $\Gamma$,s3836-0 pmdaNoMaj,s3836-0 tabou, s3836-0 all},
			legend style={at={(1.4,1)},anchor=north}, legend columns=1
			xlabel={Caractérisation de taille k},
			ylabel={Nombre de comparaisons d'entités},
			xmin={14},
			xmax={26},
			%ymax={1000000000}
			%title={Résolution sans heuristique de rch10 et s3836-0 sur les comparaisons}
			]
			\addplot +[mark=none] table[x=k,y=nbComp]{./resultats/sh_s3836.dat};
			\addplot +[mark=none] table[x=k,y=nbComp]{./resultats/tau_s3836.dat};
			\addplot +[mark=none] table[x=k,y=nbComp]{./resultats/gamma_s3836.dat};
			\addplot +[mark=none, color=green] table[x=k,y=nbComp]{./resultats/pmda_noMaj_s3836.dat};
			\addplot +[mark=none, color=cyan] table[x=k,y=nbComp]{./resultats/tabou_s3836.dat};
			\addplot +[mark=none, color=violet] table[x=k,y=nbComp]{./resultats/all_s3836.dat};
			
			\end{axis}
			\end{tikzpicture}
		}
		\only<2>
		{
			\begin{tikzpicture}[scale=0.8]
			\begin{axis}[
			legend entries={s3836-0 standard, s3836-0 trie par $\mathcal{T}$,s3836-0 par $\Gamma$,s3836-0 pmdaNoMaj,s3836-0 tabou, s3836-0 all},
			legend style={at={(1.4,1)},anchor=north}, legend columns=1
			xlabel={Caractérisation de taille k},
			ylabel={Temps d'éxécution en seconde},
			xmin={14},
			xmax={26},
			%ymax={1000000000}
			%title={Résolution sans heuristique de rch10 et s3836-0 sur les comparaisons}
			]
			\addplot +[mark=none] table[x=k,y=temps]{./resultats/sh_s3836.dat};
			\addplot +[mark=none] table[x=k,y=temps]{./resultats/tau_s3836.dat};
			\addplot +[mark=none] table[x=k,y=temps]{./resultats/gamma_s3836.dat};
			\addplot +[mark=none, color=green] table[x=k,y=temps]{./resultats/pmda_noMaj_s3836.dat};
			\addplot +[mark=none, color=cyan] table[x=k,y=temps]{./resultats/tabou_s3836.dat};
			\addplot +[mark=none, color=violet] table[x=k,y=temps]{./resultats/all_s3836.dat};
			\end{axis}
			\end{tikzpicture}
		}
	\end{overprint} 

%\end{frame}

\subsection{Recherche approchée}
\begin{frame}{Algorithme de recherche approchée par roulette proportionelle}
% 	\begin{itemize}
% 		\item Roulette proportionelle (non adaptative) favorisant la sélection des gènes présentant de faible taux de similarité $\cal{T}$
% 	\end{itemize}
	\begin{algorithm}
 	\scriptsize
	\textbf{roulette ($taux$, $Groupes$, $k$, $n$)}\\
	\tcp{Initialisation des probabilités pour chaque gène en fonction de leurs taux de similarité}
	\tcp{Les gènes présentant de faible taux de similarité $\cal{T}$ auront plus de chance d'être sélectionné}
	Réel $proba[] \leftarrow$ initialisation\_proba($taux$)\\
	\pause
	\Tq{$VRAI$ \tcp{Boucle 1}}
	{	
		\pause
		\tcp{Génération d'une combinaison à partir des probabilités de $proba$}
		Entier $combinaison \leftarrow genereCombinaisonAvecProba(proba)$\\
		
% 		\Tq{$|combinaison| < k $ \tcp{Boucle 2}}
% 		{
% 			Réel $alea \leftarrow $ nombre aléatoire entre 0 et 1 \\
% 			\PourTous{$i \in [1,\ldots,n]$ \tcp{Boucle 3: génération de la combinaison}}
% 			{
% 				\Si{$proba[i] \geq alea$}
% 				{
% 					$combinaison \leftarrow combinaison \cup i$\\
% 					Sortir de la boucle 3
% 				}
% 			}
% 		}
		\pause
		\Si{caractérise\_instance($I$,$combinaison$) \tcp{Test de la combinaison}}
		{
			afficher("La combinaison ",$combinaison$," permet de caractériser l'instance.")\\
			$k \leftarrow k-1$\\
			$combinaison \leftarrow \{\}$		
		}
	}
	\label{algoRoulette}
\end{algorithm}
\end{frame}

\begin{frame}{Recherche approchée}
	\begin{block}{Résultats sur instances réels}
		\begin{center}
			\begin{tabular}{|c|c|c|c|c|c|}
			\hline 
			\multirow{2}*{Instances} & \multirow{2}*{Entités} & \multirow{2}*{Gènes}& \multirow{2}*{B. Min} & \multicolumn{2}{c|}{Roulette proportionelle} \\
			\cline{5-6} 
			 & & & & k & temps \\
			\hline 
			rch8 & 56 & 27 & \textbf{9} & \textcolor{blue}9 & 0.031 \\ 
			\hline 
			raphv & 108 & 68 & \textbf{6} & \textcolor{blue}{6} & 0.657 \\ 
			\hline 
			raphy & 112 & 70 & \textbf{6} & \textcolor{blue}{6} & 0.873 \\ 
			\hline 
			rarep & 112 & 72 & \textbf{12} & 14 & 36.627 \\ 
			\hline 
			rch10 & 112 & 86 & \textbf{10} & 12 & 65.615 \\ 
			\hline 
			\end{tabular} 
		\end{center}
	\end{block}
\end{frame}

\subsection{Conclusions  et perspective}
\begin{frame}{Conclusions}
	\begin{itemize}
		\item Définitions de critères permettant de déterminer la difficulté d'une instance.
		\pause
		\item Mise en place d'heuristiques permettant des résolutions beaucoup plus efficaces sur des instances réelles.
%		\pause
%		\item Preuve d'absence d' heuristique utilsant le critère $\cal{T}$ permettant d'améliorer significativement la résolution d'une instance aléatoire.
		\pause
		\item Mise en place d'une recherche approchée fournissant d'excellents résultats.
	\end{itemize}
\end{frame}

\begin{frame}{Perspectives}
	\begin{itemize}
		\item Améliorer la recherche approchée.
		\pause
		\item Développer un générateur d'instance simulant des instances réelles.
		\pause 
		\item Travailler sur des instances réelles issues d'autres disciplines que celle de la biologie végétale (médecine, ...).
%		\pause
%		\item Chercher des formules à caractères sémantique ($a \lor \lnot b \equiv a \Rightarrow b$).
	\end{itemize}
\end{frame}

\begin{frame}{}
\centering
Merci pour votre attention.
\end{frame}

\end{document} 
